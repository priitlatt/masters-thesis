%!TEX root = ../../thesis.tex

\subsection{\texorpdfstring{$0|2$}{0|2}-dimensionaalsed
    \texorpdfstring{$3$}{3}-Lie superalgebrad}

Olgu meil $3$-Lie superalgebra $\g$, mille supervektorruumi dimensioon
on $0|2$ ja tema vektorruumi baas on $\{ f_1, f_2 \}$. Siis avaldub
kommutaator baasielementidel järgmiselt:
\begin{align*}
    \begin{matrix*}[l]
        & [f_1, f_1, f_1] = \m_1 f_1 + \m_2 f_2, \quad
        & [f_1, f_2, f_2] = \m_5 f_1 + \m_6 f_2, \\
        & [f_1, f_1, f_2] = \m_3 f_1 + \m_4 f_2, \quad
        & [f_2, f_2, f_2] = \m_7 f_1 + \m_8 f_2, \\
    \end{matrix*}
\end{align*}
kus $\m_1, \m_2, \dots, \m_8 \in \C$ on suvalised skalaarid.

Rakendades neile kommutaatoritele nüüd meie algoritmi saame $24$
mittetriviaalset kitsendust kitsendust\footnote{Need on leitavad veebilehelt
\url{https://www.example.com/2_0_restrictions.pdf}},
mida Lie superalgebra $\g$ struktuurikonstandid rahuldama peavad. Kitsenduste
lahenditeks sobivad struktuurikonstantide komplektid
\renewcommand\arraystretch{1}
\begin{align*}
    \begin{pmatrix}
        m_1 \\
        m_2 \\
        m_3 \\
        m_4 \\
        m_5 \\
        m_6 \\
        m_7 \\
        m_8
    \end{pmatrix}
    =
    \begin{pmatrix*}[r]
        -c_1 \\
        \frac{c_1^2}{c_2} \\
        -c_2 \\
        c_1 \\
        -\frac{c_2^2}{c_1} \\
        c_2 \\
        -\frac{c_2^3}{c_1^2} \\
        \frac{c_2^2}{c_1}
    \end{pmatrix*},
    \quad
    \begin{pmatrix}
        m_1 \\
        m_2 \\
        m_3 \\
        m_4 \\
        m_5 \\
        m_6 \\
        m_7 \\
        m_8
    \end{pmatrix}
    =
    \begin{pmatrix*}[r]
        0 \\
        c_3 \\
        0 \\
        0 \\
        0 \\
        0 \\
        0 \\
        0
    \end{pmatrix*},
    \quad
    \begin{pmatrix}
        m_1 \\
        m_2 \\
        m_3 \\
        m_4 \\
        m_5 \\
        m_6 \\
        m_7 \\
        m_8
    \end{pmatrix}
    =
    \begin{pmatrix*}[r]
        0 \\
        0 \\
        0 \\
        0 \\
        0 \\
        0 \\
        c_4 \\
        0
    \end{pmatrix*},
\end{align*}
\renewcommand\arraystretch{1}
kus $c_1, c_2, c_3, c_4 \in \C \setminus \{0\}$ on vabad muutujad.

\begin{lau}
    $3$-Lie superalgebra, mille supervektorruumi dimensioon on $0|2$, on Abeli
    Lie superalgebra, või tema mittetriviaalsed kommutatsiooniseosed on
    \renewcommand\arraystretch{1.5}
    \begin{align}
        & \left\{
            \begin{matrix*}[l]
                [f_1, f_1, f_1] = -c_1 f_1 + \frac{c_1^2}{c_2} f_2, \\
                [f_1, f_1, f_2] = -c_2 f_1 + c_1 f_2, \\
                [f_1, f_2, f_2] = -\frac{c_2^2}{c_1} f_1 + c_2 f_2, \\
                [f_2, f_2, f_2] = -\frac{c_2^3}{c_1^2} f_1 +
                    \frac{c_2^2}{c_1} f_2, \\
            \end{matrix*}
        \right. \\[0.2cm]
        %
        &\ \ \ [f_1, f_1, f_1] = c f_2, \\[0.2cm]
        %
        &\ \ \ [f_2, f_2, f_2] = c f_1,
    \end{align}
    \renewcommand\arraystretch{1}
    kus $c, c_1, c_2 \in \C \setminus \{ 0 \}$ on vabad muutujad.
\end{lau}
