%!TEX root = ../thesis.tex

\begin{center}
    {\large \textbf{\title}} \\
    Magistritöö \\
    \author
\end{center}

\paragraph{Lühikokkuvõte.}
    Käesolevas magistritöös tuletame meelde mõned Lie algebrate teooria
    põhitõed ja vaatame selle klassikalise struktuuri üldistusi. Filippov
    konstrueeris artiklis \cite{filippov1985} $n$-Lie algebra, kus binaarne
    kommutaator on asendatud $n$-aarse analoogiga. Meie kombineerime viimase
    Lie superalgebra struktuuriga, mis üldistab Lie algebraid kasutades
    $\Z_2$-gradueeritud vektorruumi ning gradueeringu iseärasusi kommutaatoril
    tavalise vektorruumi asemel, et saada \emph{$n$-Lie superalgebra}, nagu seda
    on tehtud artiklis \cite{Abramov:2014}. Me uurime $n$-Lie superalgebra,
    ehk $n$-aarse gradueeritud Filippovi samasust rahuldava tehtega
    superalgebra omadusi, ning rakendades ideid artiklitest
    \cite{Abramov:2014,AKMS:2014} indutseerime $n$-Lie superalgebratest
    $(n+1)$-Lie superalgebrad. Viimaks uurime me ternaarseid Lie superalgebraid
    üle $\C$, kus algebra aluseks oleva supervektorruumi dimensioon on
    $m|n$, $m+n \leq 4$. Me teeme kindlaks kui palju on erinevaid võimalikke
    kommutatsioonieeskirju, mida neile tingimustele vastavad $3$-Lie
    superalgebrad omada võivad.
\paragraph{Märksõnad.}
    $n$-Lie algebra, Lie superalgebra, $n$-Lie superalgebra,
    Filippovi samasus, supervektorruum.

\vfill

\begin{center}
    {\large \textbf{\engtitle}} \\
    Master’s Thesis \\
    \author
\end{center}

\paragraph{Abstract.}
    In this master’s thesis we remind the theory of Lie algebras and
    investigate some generalizations of this structure. Filippov constructed
    $n$-Lie algebras in \cite{filippov1985} where he replaced the binary
    commutator relation with $n$-ary analogue. We combine it with
    Lie superalgebra structure that emerged from theoretical
    physics in early 70s, and which generalizes Lie algebras by using
    $\Z_2$-graded vector space and grading restrictions on the commutator
    instead of classical vector space, to yield a \emph{$n$-Lie superalgebra}
    as introduced in \cite{Abramov:2014}.
    We study the properties of $n$-Lie superalgebras that are essentially
    superalgebras equipped with $n$-ary commutator relation obeying graded
    Filippov identity. Next we apply ideas from
    \cite{Abramov:2014,AKMS:2014} to induce $(n+1)$-Lie superalgebra from
    $n$-Lie superalgebra. Finally, we set an upper bound for the number of
    different $3$-Lie superalgebras over $\C$ with super vector space of
    dimension $m|n$, where $m+n \leq 4$.
\paragraph{Keywords.}
    $n$-Lie algebra, Lie superalgebra, $n$-Lie superalgebra,
    Filippov identity, super vector space.