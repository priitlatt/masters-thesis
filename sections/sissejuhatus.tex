%!TEX root = ../thesis.tex

\section*{Sissejuhatus}
\addcontentsline{toc}{section}{Sissejuhatus}

Matemaatika haru, mida me täna tunneme kui \emph{Lie teooriat} kerkis esile
geomeetria ja lineaaralgebra uurimisest. Lie teooria üheks keskseks mõisteks
on \emph{Lie algebra} -- vektorruum, mis on varustatud
mitteassotsiatiivse korrutamisega ehk nõndanimetatud \emph{Lie sulu} või
\emph{kommutaatoriga}.
Lie algebrad ja nende uurimine on tihedalt seotud teise Lie teooria keskse
mõistega, milleks on \emph{Lie rühm}. Viimased on
struktuurid, mis on korraga nii algebralised rühmad kui ka diferentseeruvad
muutkonnad, kusjuures rühma korrutamine ja selle pöördtehe on mõlemad
diferentseeruvad. Osutub, et igale Lie rühmale saab vastavusse seada Lie
algebra, kuid üldjuhul kahjuks vastupidine väide ei kehti. Samas on võimalik
näidata pisut nõrgem tulemus: suvalise lõplikumõõtmelise
reaalse või kompleksse Lie algebra jaoks leidub temale üheselt
vastav sidus Lie rühm \cite{kirillov2008introduction}. Just selle viimase,
nõndanimetatud \emph{Lie kolmanda teoreemi} tõttu on võimalik
Lie rühmasid vaadelda Lie algebrate kontekstis ja see teebki Lie algebrad
äärmiselt oluliseks ja efektiivseks tööriistaks.

Käesolevas magistritöös uurime me Lie algebrate ühte võimalikku üldistust,
milleks on $n$-Lie superalgebra. Ühelt poolt on sellise konstruktsiooni
inspiratsiooniks Filippovi poolt artiklis \cite{filippov1985} tutvustatud ja
uuritud \emph{$n$-Lie algebra} mõiste, kus binaarne kommutaator on asendatud
$n$-aarse analoogiga. Teiselt poolt võtame aluseks \emph{Lie superalgebra},
kus aluseks olev vektorruum on vahetatud \emph{supervektorruumi}, või
teisiti öelduna $\Z_2$-gradueeritud vektorruumiga, mis ei ole tegelikult
muud kui vektorruumide otsesumma,
$\V = \V_{\overline{0}} \oplus \V_{\overline{1}}$.

Lie superalgebrate uurimine sai alguse 70. aastate esimeses pooles praktilisest
vajadusest teoreetilises füüsikas kui tekkisid supersümmeetrilised
väljateooriad, kus ühes formalismis kirjeldatakse nii välja kui ainet.
Sellise füüsikalise nähtuse uurimiseks sobis hästi $\Z_2$-gradueeritud
matemaatika, kus ühes tervikstruktuuris peitub tegelikult kaks erinevate
omaduste alamstruktuuri. \cite{abramov1994} Koos sellega selgus ka Lie
superalgebrate olulisus, ning neid uuriti üsna põhjalikult. Näiteks
klassifitseeris Kac\footnote{Victor Gershevich Kac (1943), vene ja ameerika
matemaatik} 77. aastal Lie superalgebrad üle algebraliselt kinniste
nullkarakteristikaga korpuste. Samas pärast seda ei ole pea 30 aasta jooksul
selles vallas suuri edusamme tehtud ja Lie superalgebrate esituste teooria
pole siiani täielikult välja arendatud \cite{musson2012lie}.

Tuginedes $n$-Lie algebra ja Lie superalgebra mõistetele defineerime
\emph{$n$-Lie superalgebra} nagu seda on tehtud artiklis \cite{Abramov:2014}.
Edasi uurime tema olulisemaid omadusi ning vaatame erijuhul $n = 3$ kuni
kolme generaatoriga algebrate klassifikatsiooni üle korpuse $\C$.

\bigskip

Käesolev magistritöö koosneb neljast peatükist ning ühest lisast.

Esimeses peatükis tuletame meelde klassikalise Lie algebrate teooria ning
rikastame seda mitmete näidetega. Selleks vaatleme kõigepealt tuntud
maatriksrühmi ning toome sisse üldistatud eksponentsiaalkujutuse mõiste,
mille abil jõuame Lie algebrateni. Lõpetuseks meenutame mõned klassikalised
definitsioonid algebraliste struktuuride vallast Lie algebrate seades, mida
on tarvis edasise mõistmisel.

Teises peatükis rakendame Filippovi konstruktsiooni ning defineerime $n$-Lie
algebra ja tutvustame artiklitest \cite{AKMS:2014, AMS:2011}
pärinevat konstruktsiooni $n$-Lie algebrast $(n+1)$-Lie algebra indutseerimiseks.
Samuti tõestame mõned tulemused artiklist \cite{AKMS:2014}, millele autorid
ei olnud põhjendusi kaasa andnud. Peatüki lõpus toome Nambu mehaanikast
konkreetse näite $n$-Lie algebra kohta.

Kolmas peatükk algab Lie superalgebra ja tema omaduste tutvustamisega. Seejärel
anname artikli \cite{Abramov:2014} põhjal $n$-Lie superalgebra definitsiooni,
mida rikastame näitega, kus näeme, et kasutades teatud gradueeritud
kommutaatorit moodustavad supervektorruumi endomorfismid $n$-Lie superalgebra.
Pärast olulisemate omadustega tutvumist rakendame artiklite
\cite{Abramov:2014,AKMS:2014} ideid ning indutseerime $n$-Lie superalgebra
põhjal $(n+1)$-Lie superalgebra.

Töö lõpetab peatükk, kus uuritakse ternaarseid Lie superalgebraid, millel
on kuni $4$ generaatorit ja algebrale aluseks olev supervektorruum on
üle kompleksarvude korpuse $\C$. Kirjeldame algortimi, mille abil on võimalik
$3$-Lie superalgebraid klassifitseerida ning rakendame seda algoritmi
$3$-Lie superalgebrate korral, mille supervektorruumi dimensioon on $m|n$, kus
$m + n \leq 4$.

\bigskip

Tähistagu kõikjal järgnevas $K$ nullkarakteristikaga korpust ning $\V$
vektorruumi üle korpuse $K$. Ruumi kokkuhoiu ja mugavuse mõttes
kasutame edaspidi vajaduse korral summade tähistamisel
Einsteini summeerimiskokkulepet. Teisiti öeldes, kui meil on indeksid
$i$ ja $j$, mis omavad väärtusi $1, \dots, n$, kus $n \in \N$, siis jätame
vahel summeerimisel summa märgi kirjutamata, ning säilitame summeerimise
tähistamiseks vaid indeksid. Einsteini summeeruvuskokkulepet arvestades
kehtivad näiteks järgmised võrdused:
\begin{align*}
    x^{i} e_{i} &= \sum_{a=1}^{n} x^{i} e_{i} = 
        x^{1} e_{1} + x^{2} e_{2} + \dots + x^{n} e_{n}, \\
    \lambda{^i_j} x^{j} &= \sum_{j=1}^{n} 
        \lambda{^i_j} x^{j} = \lambda{^i_1} x^{1} + 
        \lambda{^i_2} x^{2} + 
        \dots +\lambda{^i_n} x^{n},\\
    \eta_{ij} u^{i} v^{j} &= \eta_{11} u^{1} v^{1} + 
        \eta_{12} u^{1} v^{2} + \dots + \eta_{1n} u^{1} v^{n} + 
        \eta_{21} u^{2} v^{1} + \dots + \eta_{nn} u^{n} v^{n},
\end{align*}
ja nii edasi.
