\documentclass[12pt,a4paper]{article}

\usepackage[T1]{fontenc}
\usepackage[utf8]{inputenc}
\usepackage[estonian]{babel}
\usepackage{geometry}
\usepackage{hyperref}

\usepackage{amsmath}
\usepackage{amsfonts}
\usepackage{amssymb}
\usepackage{amsthm}
\usepackage{amsopn}

\everymath{\displaystyle}

\author{Priit Lätt}
\title{3-Lie superalgebrad}

\theoremstyle{plain}
\newtheorem{thm}{Teoreem}[section]
\newtheorem{lau}[thm]{Lause}
\newtheorem{lemma}[thm]{Lemma}
\newtheorem{jar}[thm]{Järeldus}

\theoremstyle{definition}
\newtheorem{dfn}{Definitsioon}[section]
\newtheorem{naide}{Näide}[section]

\newcommand{\K}{\mathbb{K}}
\newcommand{\G}{\mathcal{G}}

\DeclareMathOperator{\gl}{gl}

\newcommand{\col}{\colon}
\newcommand{\arr}{\rightarrow}

\newcommand{\brac}[2]{\ensuremath{\left[ #1, #2 \right]}}
\newcommand{\nbrac}[2]{\ensuremath{\left[ #1, \dots, #2 \right]}}

\begin{document}

%%%%%%%%%%%%%%%%%%%%%%%%%%%%%%%%%%%
%%  Indutseeritud n-Lie algebra  %%
%%%%%%%%%%%%%%%%%%%%%%%%%%%%%%%%%%%

\section{Indutseeritud \texorpdfstring{$n$}\ -Lie algebra}

See peatükk tugineb artiklile \cite{AKMS:2014}

Edasises eeldame, et kõik vektoruumid on üle vaadeldud üle
$0$-karakteristikaga korpuse $\K$.

\begin{dfn}[Lie algebra]\label{def:lie_algebra}
    Vektorruumi $A$ nimetatakse \emph{Lie algebraks}, kui on
    määratud bilineaarvorm
    $\brac{\cdot}{\cdot} \colon A \times A \arr A$, mis
    suvaliste $x, y, z \in A$ korral rahuldab tingimusi
    \begin{itemize}
        \item $\brac{x}{y} = -\brac{y}{x}$,
        \item $\brac{x}{\brac{y}{z}} + 
               \brac{z}{\brac{x}{y}} +
               \brac{y}{\brac{z}{x}} = 0$.
    \end{itemize}
\end{dfn}

Bilineaarvormi $\brac{\cdot}{\cdot}$ Lie algebra definistioonis
nimetatakse selle Lie algebra suluks. Edaspidi tähistame konkreetsuse
mõttes sageli Lie suluga $\brac{\cdot}{\cdot}$ varustatud vektorruumi
$A$ paarina $(A, \brac{\cdot}{\cdot})$.

\begin{dfn}[$n$-Lie algebra]
    Vektorruumi $A$ nimetatakse \emph{$n$-Lie algebraks}, kui on
    määratud $n$-lineaarne kaldsümmeetriline kujutus
    $\nbrac{\cdot}{\cdot} \colon A^n \times A \arr A$, mis
    suvaliste
    \[ x_1, \dots, x_{n-1}, y_1, \dots, y_n \in A \]
    korral rahuldab tingimust
    \[
        \left[ x_1, \dots, x_{n-1}, \nbrac{y_1}{y_n} \right] =
        \sum_{i=1}^n \left[
            y_1, \dots, \left[ x_1, \dots, x_{n-1}, y_i \right], \dots, y_n
        \right].
    \]
\end{dfn}

\begin{dfn}[Jälg]
    Olgu $A$ vektorruum ning olgu $\phi \col A^n \arr A$. Me
    ütleme, et lineaarkujutus $\tau \col A \arr \K$ on
    \emph{$\phi$-jälg}, kui suvaliste $x_1, \dots, x_n \in A$ korral
    $\tau \left( \phi \left( x_1, \dots, x_n \right) \right) = 0$.
\end{dfn}

Olgu $\phi \col A^n \arr A$ $n$-lineaarne ja
$\tau \col A \arr \K$ lineaarne kujutus. Defineerime nende
kujutuste abil uue $(n+1)$-lineaarse kujutuse
$\phi_\tau \col A^{n+1} \arr A$ valemiga
\begin{align}\label{eq:phi_tau}
    \phi_\tau \left( x_1, \dots, x_{n+1} \right) =
    \sum_{i=1}^{n+1} (-1)^{i-1} \tau(x_i)
        \phi(x_1, \dots, \hat{x_i}, \dots, x_{n+1}),
\end{align}
kus $\hat{x_i}$ tähistab kõrvalejäätavat elementi, see tähendab
$\phi(x_1, \dots, \hat{x_i}, \dots, x_{n+1})$ arvutatakse elementidel
$x_1, \dots, x_{i-1}, x_{i+1}, \dots, x_{n+1}$.

Rikastame defineeritud kujutust ühe näitega. Võttes $n = 2$ saame
valemi~\ref{eq:phi_tau} põhjal kirjutada
\[
    \phi_\tau (x_1, x_2, x_3) =
        \tau(x_1) \phi(x_2, x_3) -
        \tau(x_2) \phi(x_1, x_3) +
        \tau(x_3) \phi(x_1, x_2).
\]

Edasises toome ära mõningad kujutuse $\phi_\tau$ tähtsamad omadused.

\begin{lemma}
    Olgu $A$ vektorruum ning $\phi \col A^n \arr A$ $n$-lineaarne kaldsümmeetriline kujutus
    ja $\tau \col A \arr \K$ lineaarne. Siis kujutus
    $\phi_\tau \col A^{n+1} \arr A$ on samuti kaldsümmeetriline. Lisaks,
    kui $\tau$ on $\phi$-jälg, siis $\tau$ on ka $\phi_\tau$-jälg.
\end{lemma}

\begin{thm}\label{thm:n+1_lie_alg}
    Olgu $(A, \phi)$ $n$-Lie algebra ning olgu $\tau$ lineaarkujutuse
    $\phi$-jälg. Siis $(A, \phi_\tau)$ on $(n+1)$-Lie algebra.
\end{thm}

Teoreemis kirjeldatud viisil saadud $(n+1)$-Lie algebrat $(A, \phi_\tau)$
nimetatakse $n$-Lie algebra $(A, \phi)$ poolt \emph{indutseeritud}
$(n+1)$-Lie algebraks.

Teoreemist~\ref{thm:n+1_lie_alg} saame teha olulise järlduse:

\begin{jar}
    Olgu $(A, \brac{\cdot}{\cdot})$ Lie algebra ning olgu antud
    $\brac{\cdot}{\cdot}$ jälg $\tau \col A \arr \K$. Siis ternaarne sulg
    $[ \cdot, \cdot, \cdot ] \col A^3 \arr A$, mis on defineeritud
    valemiga
    \[
        [x, y, z] = \tau(x)[y, z] + \tau(y)[z, x] + \tau(z)[x, y],
    \]
    määrab $3$-Lie algebra struktuuri $A_\tau$ vektorruumil $A$.
\end{jar}

%%%%%%%%%%%%%%%%%%%%%%%%%
%% n-Lie superalgebra  %%
%%%%%%%%%%%%%%%%%%%%%%%%%

\section{\texorpdfstring{$n$}\ -Lie superalgebra}

See peatükk tugineb artiklile \cite{Abramov:2014}

Järgnevas eeldame, et meil on antud supervektorruum
ehk supervektorruum $\G = \G_{\overline{0}} \oplus \G_{\overline{1}}$
ning $n$-lineaarne kujutus $\phi \col \G^n \arr \G$, mis rahuldab
tingimusi
\begin{itemize}
    \item $| \phi(x_1, \dots, x_n) | = \sum_{i=1}^n |x_i|$,
    \item $ \phi \left( x_1, \dots, x_i, x_{i+1}, \dots, x_n \right) =
            -(-1)^{ |x_i| |x_{i+1}| } \phi \left(
                x_1, \dots, x_{i+1}, x_i, \dots, x_n \right), $
\end{itemize}
kus $|x| \in \left\{ \overline{0}, \overline{1} \right\}$
tähistab elemendi $x$ paartust. Samuti eeldame, et $S \col \G \arr \K$
on lineaarne kujutus, mis rahuldab
\begin{itemize}
    \item $S \left( \phi \left( x_1, \dots, x_n \right) \right) = 0$,
    \item $S(x) = 0$ iga $x \in \G_{\overline{1}}$.
\end{itemize}

Selge, et siin sisse toodud kujutused $\phi$ ja $S$ on eelnevas
kirjeldatu analoogid supervektorruumis. Seejuures kujutust $S \col \G \arr \K$
nimetatakse \emph{superjäljeks}.

Kasutades kujutusi $\phi$ ja $S$ defineerime analoogiliselt
vektorruumide situatsioonile, kuid nüüd juba supervektorruumi iseärasusi
arvesse võttes, see tähendab paarsusi arvestades, uue kujutuse
$\phi_S \col \G^{n+1} \arr G$ valemiga
\[
    \phi_S (x_1, \dots, x_{n+1}) =
    \sum_{i=1}^{n+1} (-1)^{i-1}(-1)^{|x_i| \sum_{j=1}^{i-1} |x_j| }
        S(x_i) \phi \left(
            x_1, \dots, \hat{x_i}, \dots, x_{n+1}
        \right).
\]

Saadud kujutuse tähtsamad omadused võtab kokku järgmine oluline lemma:

\begin{lemma}
    $(n+1)$-lineaarne kujutus $\phi_S \col \G^{n+1} \arr \G$
    rahuldab tingimusi
    \begin{enumerate}
        \item $ | \phi_S \left(x_1, \dots, x_{n+1} \right) | =
               \sum_{i=1}^{n+1} |x_i| $,
        \item $ \phi_S \left(x_1, \dots, x_i, x_{i+1}, \dots, x_{n+1} \right) =
               -(-1)^{|x_i| |x_{i+1}|} \phi_S \left(
                    x_1, \dots, x_{i+1}, x_i, \dots, x_{n+1}
                \right) $,
        \item $S \left( \phi_S \left( x_1, \dots, x_{n+1} \right) \right)$.
    \end{enumerate}
\end{lemma}

Üldistame nüüd definitsiooni~\ref{def:lie_algebra} supervektorruumi jaoks
ning defineerime \emph{$n$-Lie superalgebra}.

\begin{dfn}[$n$-Lie superalgebra]
    Olgu $\G = \G_{\overline{0}} \oplus \G_{\overline{1}}$ 
    supervektorruum. Me ütleme, et $\G$ on
    \emph{$n$-Lie superalgebra}, kui $\G$ on varustatud
    gradueeritud $n$-Lie suluga $\nbrac{\cdot}{\cdot} \col \G^n \arr \G$,
    mis rahuldab tingimusi
    \begin{enumerate}
        \item $\left| \nbrac{x_1}{x_n} \right| = \sum_{i=1}^n |x_i| $,
        \item $\left[ x_1, \dots, x_i, x_{i+1}, \dots, x_n \right] =
            -(-1)^{|x_i| |x_{i+1}|} \left[
                x_1, \dots, x_{i+1}, x_i, \dots, x_n
            \right]$,
        \item $\left[ y_1, \dots, y_{n-1}, \nbrac{x_1}{x_n} \right] = $ \\
            $ = \sum_{i=1}^n (-1)^{\tau_x (i-1) \tau_y (n-1)}
            \left[
                x_1, \dots, x_{i-1},
                \left[ y_1, \dots, y_{n-1}, x_i \right],
                x_{i+1}, \dots, x_n
            \right] $,
    \end{enumerate}
    kus $x = (x_1, \dots, x_n)$ ja $y = (y_1, \dots, y_{n-1})$ ning
    $\tau_x (k) = \sum_{j=1}^{k-1} |x_i|$.
\end{dfn}

Võttes arvesse $n$-Lie superalgebra definitsiooni saame sõnastada 
teoreemi \ref{thm:n+1_lie_alg} superanaloogi järgmiselt:

\begin{thm}
    Olgu $\G = \G_{\overline{0}} \oplus \G_{\overline{1}}$
    $n$-Lie superalgebra suluga $\nbrac{\cdot}{\cdot} \col \G^n \arr \G$,
    ning $V$ lõplikumõõtmeline vektorruum ja olgu
    antud $\G$ esitus $\phi \col \G \arr \gl V$. Defineerides
    $\nbrac{\cdot}{\cdot} \col \G^{n+1} \arr \G$ valemiga
    \[
        \nbrac{x_1}{x_{n+1}} = \sum_{i=1}{n+1}
        (-1)^{i-1} (-1)^{|x_i| \tau_x (i-1)} S(\phi(x_i))
        \left[ x_1, \dots, \hat{x_i}, \dots, x_{n+1} \right],
    \]
    on supervektorruum $\G$, varustatuna suluga
    $\nbrac{\cdot}{\cdot} \col \G^{n+1} \arr \G$ $(n+1)$-Lie
    superalgebra.
\end{thm}

\bibliographystyle{plain}
\bibliography{references}

\end{document}