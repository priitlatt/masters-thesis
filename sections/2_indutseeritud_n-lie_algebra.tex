%!TEX root = ../thesis.tex

%%%%%%%%%%%%%%%%%%%%%%%%%%%%%%%%%%%
%%  Indutseeritud n-Lie algebra  %%
%%%%%%%%%%%%%%%%%%%%%%%%%%%%%%%%%%%

\section{Indutseeritud \texorpdfstring{$n$}\ -Lie algebra}

See peatükk tugineb artiklile \cite{AKMS:2014}

\begin{dfn}[$n$-Lie algebra]
    Vektorruumi $A$ nimetatakse \emph{$n$-Lie algebraks}, kui on
    määratud $n$-lineaarne kaldsümmeetriline kujutus
    $\nbrac{\cdot}{\cdot} \colon A^n \times A \to A$, mis
    suvaliste
    \[ x_1, \dots, x_{n-1}, y_1, \dots, y_n \in A \]
    korral rahuldab tingimust
    \[
        \left[ x_1, \dots, x_{n-1}, \nbrac{y_1}{y_n} \right] =
        \sum_{i=1}^n \left[
            y_1, \dots, \left[ x_1, \dots, x_{n-1}, y_i \right], \dots, y_n
        \right].
    \]
\end{dfn}

\begin{dfn}[Jälg]
    Olgu $A$ vektorruum ning olgu $\phi \col A^n \to A$. Me
    ütleme, et lineaarkujutus $\tau \col A \to \K$ on
    \emph{$\phi$-jälg}, kui suvaliste $x_1, \dots, x_n \in A$ korral
    $\tau \left( \phi \left( x_1, \dots, x_n \right) \right) = 0$.
\end{dfn}

Olgu $\phi \col A^n \to A$ $n$-lineaarne ja
$\tau \col A \to \K$ lineaarne kujutus. Defineerime nende
kujutuste abil uue $(n+1)$-lineaarse kujutuse
$\phi_\tau \col A^{n+1} \to A$ valemiga
\begin{align}\label{eq:phi_tau}
    \phi_\tau \left( x_1, \dots, x_{n+1} \right) =
    \sum_{i=1}^{n+1} (-1)^{i-1} \tau(x_i)
        \phi(x_1, \dots, \hat{x_i}, \dots, x_{n+1}),
\end{align}
kus $\hat{x_i}$ tähistab kõrvalejäätavat elementi, see tähendab
$\phi(x_1, \dots, \hat{x_i}, \dots, x_{n+1})$ arvutatakse elementidel
$x_1, \dots, x_{i-1}, x_{i+1}, \dots, x_{n+1}$.

Rikastame defineeritud kujutust ühe näitega. Võttes $n = 2$ saame
valemi~\ref{eq:phi_tau} põhjal kirjutada
\[
    \phi_\tau (x_1, x_2, x_3) =
        \tau(x_1) \phi(x_2, x_3) -
        \tau(x_2) \phi(x_1, x_3) +
        \tau(x_3) \phi(x_1, x_2).
\]

Edasises toome ära mõningad kujutuse $\phi_\tau$ tähtsamad omadused.

\begin{lemma}
    Olgu $A$ vektorruum ning $\phi \col A^n \to A$ $n$-lineaarne kaldsümmeetriline kujutus
    ja $\tau \col A \to \K$ lineaarne. Siis kujutus
    $\phi_\tau \col A^{n+1} \to A$ on samuti kaldsümmeetriline. Lisaks,
    kui $\tau$ on $\phi$-jälg, siis $\tau$ on ka $\phi_\tau$-jälg.
\end{lemma}

\begin{thm}\label{thm:n+1_lie_alg}
    Olgu $(A, \phi)$ $n$-Lie algebra ning olgu $\tau$ lineaarkujutuse
    $\phi$-jälg. Siis $(A, \phi_\tau)$ on $(n+1)$-Lie algebra.
\end{thm}

Teoreemis kirjeldatud viisil saadud $(n+1)$-Lie algebrat $(A, \phi_\tau)$
nimetatakse $n$-Lie algebra $(A, \phi)$ poolt \emph{indutseeritud}
$(n+1)$-Lie algebraks.

Teoreemist~\ref{thm:n+1_lie_alg} saame teha olulise järlduse:

\begin{jar}
    Olgu $(A, \brac{\cdot}{\cdot})$ Lie algebra ning olgu antud
    $\brac{\cdot}{\cdot}$ jälg $\tau \col A \to \K$. Siis ternaarne sulg
    $[ \cdot, \cdot, \cdot ] \col A^3 \to A$, mis on defineeritud
    valemiga
    \[
        [x, y, z] = \tau(x)[y, z] + \tau(y)[z, x] + \tau(z)[x, y],
    \]
    määrab $3$-Lie algebra struktuuri $A_\tau$ vektorruumil $A$.
\end{jar}
