%!TEX root = ../thesis.tex

%%%%%%%%%%%%%%%%%%%%%%%%%
%% n-Lie superalgebra  %%
%%%%%%%%%%%%%%%%%%%%%%%%%

\section{\texorpdfstring{$n$}\ -Lie superalgebra}

Järgnevas toome sisse \emph{supermatemaatika} põhimõsted ning defineerime
nende abil \emph{Lie superalgebra}. Edasi ühendame Lie superalgebra
konstruktsiooni Filippovi $n$-Lie algebraga, ning konstrueerime
\emph{$n$-Lie superalgebra}.

\subsection{Lie superalgebra}

Eelmises peatükis nägime, kuidas Filippov leidis viisi Lie algebra
üldistamiseks kasutades binaarse operatsiooni asemel $n$-aarset
operatsiooni, kuid jättes aluseks oleva algebra samaks. Teine viis
Lie algebra mõistet üldistada on vaadelda algebra asemel \emph{superalgebrat}.
Tuletame selleks kõigepealt meelde mõned baasdefinitsioonid.

\begin{dfn}
    Vektorruumi $\V$ nimetatakse \emph{$\Z_2$-gradueeritud vektorruumiks}
    ehk \emph{supervektorruumiks}, kui ta on esitatav vektorruumide
    otsesummana $\V = \V_{\overline{0}} \oplus \V_{\overline{1}}$.
    Vektorruumi $\V_{\overline{0}}$ elemente nimetatakse seejuures
    \emph{paarisvektoriteks} ja $\V_{\overline{1}}$ elemente
    \emph{paarituteks} vektoriteks.
\end{dfn}

Kui supervektorruumi element $x \in \V$ korral
$x \in \V_{\overline{0}}$ või $x \in \V_{\overline{1}}$, siis ütleme, et
vektor $x$ on \emph{homogeenne}. Homogeensete vektorite korral on
otstarbekas vaadelda kujutust $|\cdot| \col \V \to \Z_2$, mis on defineeritud
võrdusega
\begin{align*}
    |x| = \begin{cases}
        \overline{0}, &\text{kui}\ x \in \V_{\overline{0}}, \\
        \overline{1}, &\text{kui}\ x \in \V_{\overline{1}}.
    \end{cases}
\end{align*}
Homogeense elemendi $x$ korral nimetame suurust $|x| \in \Z_2$ tema
paarsuseks. Märgime, et kasutades edaspidises kirjutist $|x|$ eeldame
vaikimisi, et element $x \in \V$ on homogeenne.

Supervektorruumi $\V = \V_{\overline{0}} \oplus \V_{\overline{1}}$
\emph{alamruumiks} nimetatakse supervektorruumi
$\W = \W_{\overline{0}} \oplus \W_{\overline{1}} \subseteq \V$, kui
vastavate otseliidetavate gradueeringud ühtivad, see tähendab
$\W_i \subset \V_i$, kus $i \in \Z_2$.

\begin{dfn}
    Olgu supervektorruumil $\A$ antud bilineaarne algebraline tehe
    $\phi \col \A \times \A \to \A$. Supervektorruumi $\A$ nimetatakse
    \emph{superalgebraks}, kui tehe $\phi$ rahuldab suvaliste homogeensete
    vektorite $x, y \in \A$ korral tingimust
    \begin{align}\label{def:superalgebra-parity}
        \left|\phi(x, y)\right| = |x| + |y|.
    \end{align}
\end{dfn}

Paneme tähele, et superalgebra definitsioonis nõutav kujutus
$\phi \col \A \times \A \to \A$ nõuab taustal, et supervektorruum
$\A$ oleks tegelikult algebra. Supervektorruumi ühikelement ja
assotsiatiivsus defineeritakse tavapärasel viisil.

Sellega on meil olemas piisvad vahendid, et defineerida
\emph{Lie superalgebra}.

\begin{dfn}
    Olgu $\g = \g_{\overline{0}} \oplus \g_{\overline{1}}$ superalgebra,
    millel on määratud bilineaarne kujutus
    $[\cdot, \cdot] \col \g \times \g \to \g$, mis on suvaliste
    homogeensete elementide $x, y, z \in \g$ korral
    \begin{enumerate}
        \item kaldsümmeetriline gradueeritud mõttes:
            \begin{align}\label{def:super-antisymm}
                [x, y] = -(-1)^{|x||y|}[y, x],
            \end{align}
        \item rahuldab gradueeritud Jacobi samasust
            \begin{align}\label{id:super-jacobi-id}
                [x, [y, z]] = [[x, y], z] + (-1)^{|x||y|}[y, [x, z]].
            \end{align}
    \end{enumerate}
\end{dfn}

Supermatemaatikas on standardseks võtteks elementide $x$ ja $y$ järjekorra
vahetamisel seda operatsiooni balansseerida korrutades saadav tulemus
läbi suurusega $(-1)^{|x||y|}$, see tähendab võetakse arvesse elementide
gradueeringuid.
Seega on definitsiooni tingimus \eqref{def:super-antisymm} põhjendatud.
Gradueeritud Jacobi samasuse \eqref{id:super-jacobi-id} selline kuju võib
esmapilgul tunduda mõnevõrra üllatav, kuid tuletades meelde, et me võime
klassikalise Jacobi samasuse kirja panna ka kujul \eqref{id:jacobi-modified},
siis on selge, et selline üldistus omab mõtet. Enamgi veel, kui $x$ ja $y$ on
paarisvektorid, siis saamegi täpselt samasuse \eqref{id:jacobi-modified}.

\begin{markus}
    Prefiks "`\emph{super}"' kõikide mõistete ees pärineb teoreetilise füüsika
    harust, mida nimetatakse supersümmeetriaks. Meie vaadeldavad
    "`super"'-struktuurid loovad algebralised vahendid, milles on
    võimalik supersümmeetrilisi füüsikalisi teooriaid formuleerida.
    Täpsemalt on superruumis võimalik ühes konstruktsioonis siduda oma
    iseloomult täiesti erinevad fermionid ja bosonid, see on aine- ja
    väljaosakesed.

    Nii avaldub ka Lie superalgebrate tähtsus ja rakendus teoreetilises
    füüsikas, kus neid kasutatakse supersümmeetriate matemaatilises
    kirjelduses. Tavaliselt vaadeldakse neis teooriates superalgebra
    paarisosa kui bosoneid, ja paaritule osale seatakse vastavusse fermionid.
\end{markus}

\begin{lau}
    Olgu $(\g = \g_{\overline{0}} \oplus \g_{\overline{1}}, [\cdot, \cdot])$
    Lie superalgebra. Sel juhul,
    \begin{enumerate}[label=\arabic*)]
        \item kui $\g_{\overline{1}} = 0$, siis $\g$ on Lie algebra;
        \item kui $\g_{\overline{0}} = 0$, siis $\g$ on Abeli Lie
            superalgebra, see tähendab $[\g, \g] = 0$.
    \end{enumerate}
\end{lau}

\begin{proof}
    Olgu $(\g = \g_{\overline{0}} \oplus \g_{\overline{1}}, [\cdot, \cdot])$
    Lie superalgebra.
    \begin{enumerate}[label=\arabic*)]
        \item Eeldame, et $\g_{\overline{1}} = 0$. Siis suvaliste
            $x, y \in \g$ korral $|x| = |y| = \overline{0}$ ja
            $|x||y| = \overline{0}$, ning tingimusest
            \eqref{def:super-antisymm} saab tavaline kaldsümmeetrilisus,
            ning nagu eelnevalt juba märgitud, on sel juhul gradueeritud
            Jacobi samasus täpselt Jacobi samasus.
        \item Olgu nüüd $\g_{\overline{0}} = 0$, see tähendab $\g$ koosneb
            ainult paaritutest elementidest ehk iga $x, y \in \g$ korral
            $|x| = |y| = \overline{1}$. Kuna sulg $[\cdot, \cdot]$
            määrab $\g$ superalgebra struktuuri, siis kehtib
            $\left| [x, y] \right| = |x| + |y|$, mis tähenab, et
            kahe elemendi sulg on alati paariselement. Teisalt
            $\g_{\overline{0}} = 0$, ja järelikult suvaliste $x, y \in \g$
            korral $[x, y] = 0$, mis tähendabki, et $\g$ on Abeli
            Lie superalgebra.
    \end{enumerate}
\end{proof}

Meenutame, et näites \ref{naide:lie-algebra-konstrueerimine} andsime eeskirja,
kuidas suvalisest assotsiatiivsest algebrast on võimalik konstrueerida
Lie algebra. Osutub, et analoogiline situatsioon kehtib ka Lie superalgebrate
korral.

\begin{naide}
    Olgu $\A = \A_{\overline{0}} \oplus \A_{\overline{1}}$ assotsiatiivne
    superalgebra, kus elementide $x, y \in \A$ korrutis on tähistatud $xy$.
    Me saame anda supervektorruumile $\A$ Lie superalgebra struktuuri,
    kui defineerime homogeensete elementide jaoks sulu võrdusega
    \begin{align}\label{naide:assot-super-lie-alg}
        [x, y] = xy - (-1)^{|x||y|} yx, \quad x, y \in \A,
    \end{align}
    ning mittehomogeensete elementide jaoks rakendame korrutamise
    bilineaarsust.

    Sel juhul
    \begin{align*}
        [x, y] &= xy - (-1)^{|x||y|} yx = \\
        &= -(-1)^{|x||y|} \left[
            -(-1)^{|x||y|}xy + yx
        \right] = \\
        &= -(-1)^{|x||y|} [y, x],
    \end{align*}
    ehk nõue \eqref{def:super-antisymm} on täidetud.

    Gradueeritud Jacobi samasuse jaoks rakendame kaks korda valemit
    \eqref{naide:assot-super-lie-alg} ning arvutame võrduses
    \eqref{id:super-jacobi-id} olevad liikmed, kusjuures assotsiatiivsust
    arvesse võttes jätame korrutistes sulud kirjutamata.
    \begin{flalign*}
        &-[x, [y, z]] = -xyz + (-1)^{|x||y| + |x||z|} yzx +
        (-1)^{|y||z|} xzy - (-1)^{|x||y| + |x||z| + |y||z|} zyx, \\[0.25cm]
        %
        &[[x, y], z] = xyz - (-1)^{|x||z| + |y||z|} zxy -
        (-1)^{|x||y|} yxz + (-1)^{|x||y| + |x||z| + |y||z|} zyx, \\[0.25cm]
        %
        &(-1)^{|x||y|}[y, [x, z]] = (-1)^{|x||y|} yxz -
            (-1)^{|x||y| + |x||y| + |y||z|} xzy - \\
        &\phantom{(-1)^{|x||y|}[y, [x, z]] = }\
            (-1)^{|x||y| + |x||z|} yzx +
            (-1)^{|x||y| + |x||y| + |x||z| + |y||z|} zxy.
    \end{flalign*}
    Arvestades, et $(-1)^{2k + l} = (-1)^{l}$ mistahes
    $k, l \in \N \cup \{0\}$ korral, siis saame tulemusi liites kokku
    täpselt nulli, ehk sulg \eqref{naide:assot-super-lie-alg} rahuldab
    gradueeritud Jacobi samasust. Kokkuvõttes oleme saanud eeskirja, mille
    abil on suvaline assotsiatiivne superalgebra võimalik varustada
    Lie superalgebra struktuuriga.
\end{naide}

Loomulikult räägitakse ka Lie superalgebrate puhul homomorfismidest. Need
defineeritakse klassikalisel viisil, see tähendab kui
$\g = \g_{\overline{0}} \oplus \g_{\overline{1}}$ ja
$\h = \h_{\overline{0}} \oplus \h_{\overline{1}}$
on Lie superalgebrad, siis $f \col \g \to \h$ nimetatakse
Lie superalgebrate homomorfismiks, kui
$f([x, y]_\g) = [f(x), f(y)]_\h$
suvaliste $x, y \in \g$ korral. Homomorfismi $f \col \g \to \h$
gradueering $|f| \in \Z_2$ määratakse seejuures seosest
\begin{align*}
    f(\g_i) \subseteq \h_{i + |f|}, \quad i \in \Z_2.
\end{align*}
Niisiis, $f$ on paaris, kui $f(\g_i) \subseteq \h_i$, ja $f$ on paaritu,
kui $f(\g_i) \subseteq \h_{i+\overline{1}}$. Samamoodi võib kujutuse
gradueeringust rääkida ka supervektorruumi või superalgebra korral.
Homomorfismi, mille lähte- ja sihtalgebra ühtivad nimetame endomorfismiks
ja supervektorruumi $\V$ kõigi endomorfismide hulka tähistame $\End \V$.

\begin{lemma}
    Olgu $\V$ supervektorruum ja $f, g \in \End \V$. Kui $|f| = |g|$,
    siis $f \circ g$ on paaris, ning vastasel korral paaritu.
\end{lemma}

\begin{proof}
    Olgu $\V = \V_{\overline{0}} \oplus \V_{\overline{1}}$ supervektorruum,
    $f, g \in \End \V$ ja $i \in \Z_2$. Siis
    \begin{align*}
        (f \circ g)(\V_i) = f(g(\V_i)) \subseteq
        f(\V_{i + |g|}) \subseteq \V_{i + |g| + |f|}.
    \end{align*}
    Kui $|f| = |g|$, siis $i + |g| + |f| = i$ ehk $f \circ g$ on paaris.
    Vastasel korral $i + |g| + |f| = i + \overline{1}$ ja seega
    $f \circ g$ on paaritu.
\end{proof}

\begin{jar}\label{lemma:kompositsiooni-paarsus}
    Olgu $\V = \V_{\overline{0}} \oplus \V_{\overline{1}}$
    supervektorruum, $f, g \in \End \V$ ja $i \in \Z_2$. Siis
    \[ |f \circ g| = |f| + |g|. \]
\end{jar}

Järelduse põhjal on selge, et supervektorruumis $\End \V$ kujutuste
kompositsioon $\circ$ rahuldab tingimust
\eqref{def:superalgebra-parity}, ning seega on $\End \V$
superalgebra. Ühes eelmise näitega saame nüüd kirja panna järgmise
teoreemi.

\begin{thm}
    Olgu $\V$ supervektorruum. Assotsiatiivne superalgebra
    $\End \V$, mis on varustatud suluga
    \begin{align}\label{eq:end_v_sulg}
        [f, g] = f \circ g - (-1)^{|f||g|} g \circ f, \quad f, g \in \End \V,
    \end{align}
    on Lie superalgebra. \hfill \qed
\end{thm}

\begin{lau}\label{lause:kompositsioon sulus}
    Olgu $\V$ supervektorruum. Lie superalgebra $\End \V$ elementide
    $f, g$ ja $h$ korral kehtivad võrdused
    \begin{align}
        \label{eq:end_v_komp_brac_1}
        [f, g \circ h] =
            [f, g] \circ h + (-1)^{|f||g|} g \circ [f, h], \\
        \label{eq:end_v_komp_brac_2}
        [f \circ g, h] =
            f \circ [g, h]+ (-1)^{|g||h|} [f, h] \circ g.
    \end{align}
\end{lau}

\begin{proof}
    Olgu $f, g, h \in \End \V$. Vastavalt valemile \eqref{eq:end_v_sulg}
    saame siis kirjutada
    \begin{align*}
        &[f, g \circ h] = \\
        &= f \circ g \circ h +
            (-1)^{1+|g \circ h| |f|} g \circ h \circ f = \\
        &= f \circ g \circ h -
            (-1)^{|f||g| + |f||h|} g \circ h \circ f = \\
        &= f \circ g \circ h - (-1)^{|f||g|} g \circ f \circ h +
            (-1)^{|f||g|} g \circ f \circ h -
            (-1)^{|f||g| + |f||h|} g \circ h \circ f = \\
        &= \left[ f \circ g - (-1)^{|f||g|} g \circ f \right] \circ h +
            g \circ \left[ (-1)^{|f||g|} f \circ h -
            (-1)^{|f||g| + |f||h|} h \circ f \right] = \\
        &= [f, g] \circ h + (-1)^{|f||g|} g \circ [f, h],
    \end{align*}
    ehk võrdus \eqref{eq:end_v_komp_brac_1} kehtib.

    Veendumaks, et kehtib ka \eqref{eq:end_v_komp_brac_2} paneme
    tähele, et
    \begin{align*}
        &[f \circ g, h] = \\
        &= -(-1)^{|f \circ g||h|} [h, f \circ g] = \\
        &= -(-1)^{|f \circ g||h|} \left(
            [h, f] \circ g + (-1)^{|f|h|} f \circ [h, g] \right) = \\
        &= -(-1)^{|f \circ g||h|} [h, f] \circ g -
            (-1)^{|f \circ g||h|+|f||h|} f \circ [h, g] = \\
        &= (-1)^{|f||h|+|h||g|+|h||f|} [f, h] \circ g +
            (-1)^{|f||h|+|g||h|+|f||h|+|h||g|} f \circ [g, h] = \\
        &= (-1)^{|h||g|} [f, h] \circ g + f \circ [g, h] = \\
        &= f \circ [g, h] + (-1)^{|g||h|} [f, h] \circ g,
    \end{align*}
    mida oligi tarvis.
\end{proof}

\subsection{\texorpdfstring{$n$}\ -Lie superalgebra}

Kombineerime nüüd Filippovi $n$-Lie algebra ning eelnevas alapeatükis
tutvustatud Lie superalgebra üheks \emph{$n$-Lie superalgebraks}, nagu
seda on tehtud artiklis \cite{Abramov:2014}.

\begin{dfn}[$n$-Lie superalgebra]\label{def:n-Lie-superalgebra}
    Olgu $\g = \g_{\overline{0}} \oplus \g_{\overline{1}}$
    supervektorruum. Me ütleme, et $\g$ on
    \emph{$n$-Lie superalgebra}, kui $\g$ on varustatud
    gradueeritud $n$-Lie suluga $\nbrac{\cdot}{\cdot} \col \g^n \to \g$,
    mis rahuldab järgmisi tingimusi:
    \begin{enumerate}
        \item $n$-aarne sulg $\nbrac{\cdot}{\cdot}$ on $n$-lineaarne ja on
            kooskõlas gradueeringutega, see tähendab suvaliste
            homogeensete elementide $x_1, x_2, \dots, x_n \in \g$ korral
            \begin{align}\label{def:n-lie-superalg-brac-grading}
                \left| \nbrac{x_1}{x_n} \right| = \sum_{i=1}^n |x_i|,
            \end{align}
        \item $\nbrac{\cdot}{\cdot}$ on kaldsümmeetriline gradueeritud mõttes,
            see tähendab iga $i \in \{1, 2, \dots, n-1\}$ ja suvaliste
            homogeensete elementide $x_1, x_2, \dots, x_n \in \g$ korral
            \begin{align}
                \left[ x_1, \dots, x_i, x_{i+1}, \dots, x_n \right] =
                -(-1)^{|x_i| |x_{i+1}|} \left[
                    x_1, \dots, x_{i+1}, x_i, \dots, x_n
                \right],
            \end{align}
        \item kõikide homogeensete elementide
            $x_1, x_2, \dots, x_n, y_1, y_2, \dots, y_{n-1} \in \g$ korral
            on täidetud gradueeritud Filippovi samasus
            \begin{align}\begin{split}
                &\left[ y_1, \dots, y_{n-1}, \nbrac{x_1}{x_n} \right] = \\
                &= \sum_{i=1}^n (-1)^{|\mathbf{x}|_{i-1} |\mathbf{y}|_{n-1}}
                \left[
                    x_1, \dots, x_{i-1},
                    \left[ y_1, \dots, y_{n-1}, x_i \right],
                    x_{i+1}, \dots, x_n
                \right],
            \end{split}\end{align}
            kus $\mathbf{x} = (x_1, \dots, x_n)$ ja
            $\mathbf{y} = (y_1, \dots, y_{n-1})$ ning
            $|\mathbf{x}|_i = \sum_{j=1}^{i} |x_i|$.
    \end{enumerate}
\end{dfn}

Paneme tähele, et võttes definitsioonis \ref{def:n-Lie-superalgebra}
$n = 2$, saame Lie superalgebra, mis tähenab, et üldistus
sellisel kujul omab mõtet. Nagu klassikalise Lie algebra, ja tegelikult
ka $n$-Lie algebra või Lie superalgebra korral, on meil võimalik vaadelda
struktuurikonstante. Selle tarbeks peab loomulikult fikseerima
supervektorruumi $\g = \g_{\overline{0}} \oplus \g_{\overline{1}}$
baasi. Olgu selleks
\begin{align}\label{n-lie-superalg-basis}
    \B = \{ e_1, e_2, \dots, e_p, f_1, f_2, \dots, f_q \},
\end{align}
kusjuures $\g_{\overline{0}} = \spn \{e_1, e_2, \dots, e_p\}$
ja $\g_{\overline{1}} = \spn \{f_1, f_2, \dots, f_q\}$.

\begin{dfn}
    Olgu $n$-Lie superalgebra $\g$ vektorruumi baas $\B$ võrdusest
    \eqref{n-lie-superalg-basis}. Baasile $\B$ vastavateks
    \emph{struktuurikonstantideks} nimetatakse arve
    $K_{A_1 A_2 \dots A_n}^{B}$, mis on määratud võrranditega
    \begin{align*}
        [z_{A_1}, z_{A_2}, \dots, z_{A_n}] =
        K_{A_1 A_2 \dots A_n}^{B} z_B,
    \end{align*}
    kus $z_{A_1}, z_{A_2}, \dots, z_{A_n}, z_B \in \B$.
\end{dfn}

Ilmselt on meil baasi $\B$ elementidel sulu $[\cdot, \dots, \cdot]$
arvutamiseks kolm võimalust: kõik argumendid on paaris, kõik
argumendid on paaritud, või on nii paaris- kui ka paarituid elemente.
Kuid arvestades gradueeritud kaldsümmeetrilisust võime me
argumendid alati sellisesse järjekorda viia, et väiksema indeksiga
baasielement eelneb suurema indeksiga elemendile, ja
paarisgradueeringuga vektorid eelnevad sulus paaritutele.
Kokkuvõttes jäävad huvipakkuvate struktuurikonstantidena alles ainult:
\begin{enumerate}[label=\arabic*)]
    \item $[e_{\alpha_1}, e_{\alpha_2}, \dots, e_{\alpha_n}] =
        K_{\alpha_1 \alpha_2 \dots \alpha_n}^{\lambda} e_{\lambda}$,
    \item $[f_{i_1}, f_{i_2}, \dots, f_{i_n}] =
        K_{i_1 i_2 \dots i_n}^{B} z_{B}$,
    \item $[e_{\alpha_1}, e_{\alpha_2}, \dots, e_{\alpha_k},
            f_{i_1}, f_{i_2}, \dots, f_{i_{n-k}}] =
        K_{
            \alpha_1 \alpha_2 \dots \alpha_k i_1 i_2 \dots i_{n-k}
        }^{B} z_{B}$,
\end{enumerate}
kus $k < n$, $\alpha_1, \alpha_2, \dots, \alpha_n,
i_1, i_2, \dots, i_n \in \{1, 2, \dots, n\}$ ja $k < l$ korral
$\alpha_k < \alpha_l$ ning $i_k < i_l$.

Struktuurikonstantide kasutamise, seda eelkõige just ternaarsete
Lie superalgebra korral, juurde naaseme hiljem peatüki \textbf{???}
juures.
% TODO: viita siia kui klassifikatsiooni meetodi kirjeldusest
% kirjutatud on..

Vaatame järgnevalt lähemalt ühte $n$-Lie superalgebra näidet.
\cite{Abramov:2014}

\begin{thm}\label{teoreem:endomorphism-lie-algebra}
    Olgu $\V = \V_{\overline{0}} \oplus \V_{\overline{1}}$
    supervektorruum ning $\End \V$ tema endomorfismide
    supervektorruum. Defineerime kujutuse
    $[\cdot, \dots, \cdot] \col (\End \V)^n \to \End \V$ valemiga
    \begin{align}\label{naide:end-v-kommutaator}
        [\phi_1, \phi_2, \dots, \phi_n] =
        \sum_\sigma (-1)^{|\sigma|+|\phi_\sigma|}
            \phi_{i_1} \circ \phi_{i_2} \circ \ldots \circ \phi_{i_n},
    \end{align}
    kus $\phi = (\phi_1, \phi_2, \dots, \phi_n)$ on ruumi
    $\End \V$ endomorfismide ennik,
    $\mathrm{\sigma} = (i_1, i_2, \dots, i_n)$ on arvude
    $(1, 2, \dots, n)$ permutatsioon, ja $|\sigma|$
    tähistab selle permutatsiooni paarsust. Seejuures
    $|\phi_\sigma|$ arvutatakse valemiga
    \begin{align}\label{eq:permutation-parity}
        |\phi_\sigma| = \sum_{k=1}^{n} |\phi_{i_k}|
            \left(
                |\phi_{i_{k_1}}| + |\phi_{i_{k_2}}| + \dots +
                |\phi_{i_{k_r}}|
            \right),
    \end{align}
    kus $(i_{k_1}, i_{k_2}, \dots, i_{k_r})$ on arvud, mis
    permutatsioonis $\sigma$ moodustavad arvuga $i_k$ inversiooni,
    see tähendab iga $l = 1, 2, \dots, r$ korral $i_{k_l} > i_k$ ja
    $i_{k_l}$ eelneb elemendile $i_k$ permutatsioonis $\sigma$.

    Sel juhul on $\End \V$, varustatuna suluga
    \eqref{naide:end-v-kommutaator}, $n$-Lie superalgebra.
\end{thm}

Teoreemi täielikul tõestusel me ei peatu, kuna see nõuab väga palju
tehnilist arvutamist ning vahendeid kombinatoorikast. Küll aga vaatame
me siinkohal teoreemi tõestuse ideed, milleks läheb meil esmalt tarvis
järgnevat lemmat.

\begin{lemma}\label{lemma:n-sulg to (n-1)-sulg}
    Valemis \eqref{naide:end-v-kommutaator} esitatud $n$-aarne
    kommutaator on võimalik avaldada $(n-1)$-aarsete
    kommutaatorite abil järgmiselt:
    \begin{align}\label{eq:n komm by n-1 komm}
        [\phi_1, \phi_2, \dots, \phi_n] =
        \sum_{k=1}^{n} (-1)^{
            (k-1) + |\phi_k| \sum_{l < k} |\phi_l|
        } \phi_k \circ [
            \phi_1, \dots, \phi_{k-1}, \phi_{k+1}, \dots, \phi_n
        ].
    \end{align}
\end{lemma}

\begin{proof}
    Et üldise juhu tõestus paremini jälgitav oleks tõestame väite
    esmalt juhul $n=3$. Olgu selleks
    $\phi_1, \phi_2, \phi_3 \in \End \V$ ning kasutame samu
    tähistusi nagu teoreemi sõnastuses. Vahetu arvutuse tulemusel
    näeme, et

    \begin{align*}
        &[\phi_1, \phi_2, \phi_3] = \\
        &= (-1)^{|\phi_{123}|} \phi_1 \circ \phi_2 \circ \phi_3 +
        (-1)^{1 + |\phi_{132}|} \phi_1 \circ \phi_3 \circ \phi_2 + \\
        &\ \quad (-1)^{1 + |\phi_{213}|}
            \phi_2 \circ \phi_1 \circ \phi_3 +
        (-1)^{|\phi_{231}|} \phi_2 \circ \phi_3 \circ \phi_1 + \\
        &\ \quad (-1)^{|\phi_{312}|} \phi_3 \circ \phi_1 \circ \phi_2 +
        (-1)^{1 + |\phi_{321}|} \phi_3 \circ \phi_2 \circ \phi_1 = \\
        %
        &= \phi_1 \circ \phi_2 \circ \phi_3 + \\
        &\ \quad (-1)^{1 + |\phi_2||\phi_3|}
            \phi_1 \circ \phi_3 \circ \phi_2 + \\
        &\ \quad (-1)^{1 + |\phi_1||\phi_2|}
            \phi_2 \circ \phi_1 \circ \phi_3 + \\
        &\ \quad (-1)^{|\phi_1||\phi_2|+|\phi_1||\phi_3|}
            \phi_2 \circ \phi_3 \circ \phi_1 + \\
        &\ \quad (-1)^{|\phi_1||\phi_3|+|\phi_2||\phi_3|}
            \phi_3 \circ \phi_1 \circ \phi_2 + \\
        &\ \quad (-1)^{
                1 + |\phi_2||\phi_3|+|\phi_1||\phi_3|+|\phi_1||\phi_2|
            } \phi_3 \circ \phi_2 \circ \phi_1, = \\
        %
        &= \phi_1 \circ \left(
            \phi_2 \circ \phi_3 -
            (-1)^{|\phi_2||\phi_3|} \phi_3 \circ \phi_2
        \right) + \\
        &\ \quad (-1)^{1 + |\phi_1||\phi_2|} \phi_2 \circ \left(
            \phi_1 \circ \phi_3 -
            (-1)^{|\phi_1||\phi_3|} \phi_3 \circ \phi_1
        \right) + \\
        &\ \quad (-1)^{|\phi_1||\phi_3|+|\phi_2||\phi_3|} \phi_3 \circ
        \left(
            \phi_1 \circ \phi_2 -
            (-1)^{|\phi_1||\phi_2|} \phi_2 \circ \phi_1
        \right) = \\
        %
        &= \phi_1 \circ [\phi_2, \phi_3] +
            (-1)^{1 + |\phi_1||\phi_2|} \phi_2 \circ [\phi_1, \phi_3] +
            (-1)^{|\phi_1||\phi_3|+|\phi_2||\phi_3|} \phi_3 \circ [\phi_1, \phi_2].
    \end{align*}
    Seega kehtib võrdus
    \begin{align*}
        &[\phi_1, \phi_2, \phi_3] =
        \phi_1 \circ [\phi_2, \phi_3] -
        (-1)^{|\phi_1||\phi_2|} \phi_2 \circ [\phi_1, \phi_3] +
        (-1)^{|\phi_1||\phi_3| + |\phi_2||\phi_3|}
            \phi_3 \circ [\phi_1, \phi_2],
    \end{align*}
    mis on täpselt \eqref{eq:n komm by n-1 komm}, kui $n = 3$.

    Vaatame edasi üldist situatsiooni. Olgu $n > 2$ suvaline
    naturaalarv ja olgu $\phi_1, \phi_2, \dots, \phi_n \in \End \V$.
    Rakendades valemit \eqref{naide:end-v-kommutaator} saame kirjutada
    \begin{align*}
        &[\phi_1, \phi_2, \dots, \phi_n] = \\
        &= \sum_\sigma (-1)^{|\sigma| + |\phi_\sigma|} \phi_{i_1} \circ
            \phi_{i_2} \circ \ldots \circ \phi_{i_n} = \\
        &= \sum_{k=1}^{n} (-1)^{k-1} (-1)^{|\phi_k| \sum_{l<k} |\phi_l|}
            \phi_k \circ \sum_{\sigma'}
            (-1)^{|\sigma'| + |\phi_{\sigma'}|}\phi_{i_1} \circ
            \phi_{i_2} \circ \ldots \circ \phi_{i_{n-1}} = \\
        &= \sum_{k=1}^{n} (-1)^{k-1} (-1)^{|\phi_k| \sum_{l<k} |\phi_l|}
            \phi_k \circ [
                \phi_1, \dots, \phi_{k-1}, \phi_{k+1}, \dots, \phi_n
            ],
    \end{align*}
    nagu tarvis.
\end{proof}

Naaseme nüüd uuesti teoreemi \ref{teoreem:endomorphism-lie-algebra} juurde.
Olgu selleks
\[
    \psi_1, \psi_2, \dots, \psi_{n-1},
    \phi_1, \phi_2, \dots, \phi_n \in \End \V.
\]
Siis järleduse \ref{lemma:kompositsiooni-paarsus} järgi
\[
    | \phi_{i_1} \circ \phi_{i_2} \circ \dots \circ \phi_{i_n} | =
    \sum_{i=1}^{n} |\phi_i|.
\]
Seega on valemi \eqref{naide:end-v-kommutaator} paremal pool kõigi
liidetavate (kompositsioonide) paarsus sama ja seejuures võrdne arvuga
$\sum_{i=1}^{n} |\phi_i|$. Ilmselt on ka nende summa paarsus siis
täpselt selline, ehk kokkuvõttes
$ [\phi_1, \phi_2, \dots, \phi_n] = \sum_{i=1}^{n} |\phi_i| $,
nagu tarvis. Gradueeritud kaldsümmeetrilisuse kehtivuseks
märgime, et permutatsioonis kahe elemendi äravahetamisel
permutatsiooni paarsus muutub, ning sellega on põhjendatud
"`$-$"'. Teguri $(-1)^{|\phi_i||\phi_{i+1}|}$ saame vahetult
valemi \eqref{eq:permutation-parity} struktuurist.

Nagu tavaliselt selliste tulemuste puhul ikka, lasub põhiline keerukus
gradueeritud Filippovi samasuse kehtivuse näitamisel.
Antud juhul on seda võimalik teha matemaatilise induktsiooni abil.
Baasjuhu $n = 3$ puhul peab kommutaatori definitsiooni
\eqref{naide:end-v-kommutaator} järgi kehtima võrdus
\begin{align}\begin{split}\label{eq:tern endomorphism filippov}
    [\psi_1, \psi_2, [\phi_1, \phi_2, \phi_3]] =\
    %
    &[[\psi_1, \psi_2, \phi_1], \phi_2, \phi_3] + \\
    %
    (-1)^{|\phi_1||\psi_1| + |\phi_1||\psi_2|}
        &[\phi_1, [\psi_1, \psi_2, \phi_2], \phi_3] + \\
    %
    (-1)^{
        |\phi_1||\psi_1| + |\phi_1||\psi_2| +
        |\phi_2||\psi_1| + |\phi_2||\psi_2|
    } &[\phi_1, \phi_2, [\psi_1, \psi_2, \phi_3]].
\end{split}\end{align}
Vaatame kõigepealt selle võrduse vasakut poolt. Rakendades sisemisele
kommutaatorile lemmat \ref{lemma:n-sulg to (n-1)-sulg} saame kolme
ternaarse sulu summa, kus üheski ei esine enam argumendina ternaarset sulgu:
\begin{align*}
    [\psi_1, \psi_2, [\phi_1, \phi_2, \phi_3]] =\
    &[\psi_1, \psi_2, \phi_1 \circ [\phi_2, \phi_3]] + \\
    (-1)^{1 + |\phi_1||\phi_2|}
        &[\psi_1, \psi_2, \phi_2 \circ [\phi_1, \phi_3]] + \\
    (-1)^{|\phi_1||\phi_3| + |\phi_2||\phi_3|}
        &[\psi_1, \psi_2, \phi_3 \circ [\phi_1, \phi_2]].
\end{align*}
Rakendame nüüd liidetavatele uuesti lemmat \ref{lemma:n-sulg to (n-1)-sulg},
ning saame
\begin{align*}
    [\psi_1, \psi_2, [\phi_1, \phi_2, \phi_3]] =\
    %
    &\psi_1 \circ [\psi_2, \phi_1 \circ [\phi_2, \phi_3]] + \\
    (-1)^{ 1 + |\psi_1||\psi_2| }
        &\psi_2 \circ [\psi_1, \phi_1 \circ [\phi_2, \phi_3]] + \\
    (-1)^{
        |\psi_1||\phi_1 \circ [\phi_2, \phi_3]| +
        |\psi_2||\phi_1 \circ [\phi_2, \phi_3]|
    } &\phi_1 \circ [\phi_2, \phi_3] \circ [\psi_1, \psi_2] + \\
    %
    (-1)^{1 + |\phi_1||\phi_2|}
    &\psi_1 \circ [\psi_2, \phi_2 \circ [\phi_1, \phi_3]] + \\
    (-1)^{1 + |\phi_1||\phi_2|}
    (-1)^{ 1 + |\psi_1||\psi_2| }
        &\psi_2 \circ [\psi_1, \phi_2 \circ [\phi_1, \phi_3]] + \\
    (-1)^{1 + |\phi_1||\phi_2|} 
    (-1)^{
        |\psi_1||\phi_2 \circ [\phi_1, \phi_3]| +
        |\psi_2||\phi_2 \circ [\phi_1, \phi_3]|
    } &\phi_2 \circ [\phi_1, \phi_3] \circ [\psi_1, \psi_2] + \\
    %
    (-1)^{|\phi_1||\phi_3| + |\phi_2||\phi_3|} 
    &\psi_1 \circ [\psi_2, \phi_3 \circ [\phi_1, \phi_2]] + \\
    (-1)^{|\phi_1||\phi_3| + |\phi_2||\phi_3|} 
    (-1)^{ 1 + |\psi_1||\psi_2| }
        &\psi_2 \circ [\psi_1, \phi_3 \circ [\phi_1, \phi_2]] + \\
    (-1)^{|\phi_1||\phi_3| + |\phi_2||\phi_3|} 
    (-1)^{
        |\psi_1||\phi_3 \circ [\phi_1, \phi_2]| +
        |\psi_2||\phi_3 \circ [\phi_1, \phi_2]|
    } &\phi_3 \circ [\phi_1, \phi_2] \circ [\psi_1, \psi_2].
\end{align*}
Viimaks saame sellistele liidetavatele, kus sulu argumendis esineb
endomorfismide kompositsioon, rakendada lauset \ref{lause:kompositsioon sulus},
mis annab meile
\begin{align}\begin{split}\label{eq:tern endomorph filippov proof lfs}
    [\psi_1, \psi_2, [\phi_1, \phi_2, \phi_3]] =\
    %
    & \psi_1 \circ [\psi_2, \phi_1] \circ [\phi_2, \phi_3] + \\
    (-1)^{|\psi_2||\phi_1|}
        & \psi_1 \circ \phi_1 \circ [\psi_2, [\phi_2, \phi_3]] + \\
    (-1)^{ 1 + |\psi_1||\psi_2| }
        & \psi_2 \circ [\psi_1, \phi_1] \circ [\phi_2, \phi_3] + \\
    (-1)^{ 1 + |\psi_1||\psi_2| } (-1)^{|\psi_1||\phi_1|}
        & \psi_2 \circ \phi_1 \circ [\psi_1, [\phi_2, \phi_3]] + \\
    (-1)^{
        |\psi_1||\phi_1 \circ [\phi_2, \phi_3]| +
        |\psi_2||\phi_1 \circ [\phi_2, \phi_3]|
    } &\phi_1 \circ [\phi_2, \phi_3] \circ [\psi_1, \psi_2] + \dots,
\end{split}\end{align}
kusjuures kokku saame sel viisil viisteist liidetavat.

Rakendame nüüd eelnevalt kirjeldatud algoritmi ka võrrandi
\eqref{eq:tern endomorphism filippov} paremale poolele. Seda tehes saame
45 liidetavat, mis on analoogilisel kujul nagu võrduse
\eqref{eq:tern endomorph filippov proof lfs} paremal poolel olevad liidetavad.
Kommutaatori antisümmeetrilisust ja kompositsiooni assotsiatiivsust sobivalt
kasutades saame need liidetavad viia kujule, kus tekivad sarnaste liidetavate
grupid, millele on võimalik rakendada endomorfismide Lie superalgebra
binaarse sulu gradueeritud Jacobi samasust, või viia liidetavad
kujul $a \circ b \circ c$ ning $b \circ a \circ c$ üheks liikmeks
$[a, b] \circ c$, seejuures märke arvestades. Rekursiivselt sama mustrit
rakendades saame lõpuks nii võrrandi \eqref{eq:tern endomorphism filippov}
vasakule kui ka paremale poolele samad elemendid, ning järelikult juhul
$n = 3$ on $\End \V$, varustatuna suluga \eqref{naide:end-v-kommutaator},
tõepoolest $3$-Lie superalgebra.

Eeldades, et suvalise $k \in \N$, $k > 2$, korral iga $l < k$ määrab
sulguga \eqref{naide:end-v-kommutaator} supervektorruumil $l$-Lie
superalgebra struktuuri, saame tänu lemmale
\ref{lemma:n-sulg to (n-1)-sulg}, et sama sulg annab supervektorruumile
$\End \V$ ka $k$-Lie superalgebra struktuuri. Tõepoolest, analoogiliselt
eelnevalt kirjeldatule saame $k$-sulu gradueeritud Filippovi samasuse
lahti kirjutada ning seal kõigepealt liidetavate argumentidele ning
seejärel ka liidetavatele endile rakendada lemmat
\ref{lemma:n-sulg to (n-1)-sulg}. Nii saame juba $(k-1)$-sulud, millele
kehtib gradueeritud Filippovi samasus. Kasutame seda teadmist ja
grupeerime need liidetavad kokku. Edasi toimime rekursiivselt kuni
vasak ja parem pool ühtivad.

\begin{markus}
    Tekib loomulik küsimus, kas oleks võimalik lauset
    \ref{lause:kompositsioon sulus} tõestada ka üldisel juhul, see tähendab
    $n$-Lie superalgebra korral, mille sulg on määratud valemiga
    \eqref{naide:end-v-kommutaator}. Intuitiivselt peaks selline üldistus
    väitma, et
    \begin{align*}
        &[\psi \circ \phi_1, \phi_2, \dots, \phi_n] = \\
        &\psi \circ [\phi_1, \phi_2, \dots, \phi_n] +
        (-1)^{|\phi_1| \left(
            |\phi_2| + |\phi_3| + \dots |\phi_n|
        \right)}
        [\psi, \phi_2, \dots, \phi_n] \circ \phi_1.
    \end{align*}
    Paraku see nii ei ole. Meil piisab vaadata vaid
    klassikalisi endomorfismide $3$-Lie algebraid, kus me kujutuste
    paarsusi ei arvesta. Sel juhul võtab see valem kuju
    \begin{align*}
        [\psi \circ \phi_1, \phi_2, \phi_3] =
        \psi \circ [\phi_1, \phi_2, \phi_3] +
        [\psi, \phi_2, \phi_3] \circ \phi_1.
    \end{align*}

    Kirjutades viimases võrduse kommutaatorid definitsiooni järgi lahti saame,
    et mistahes $x \in \V$ korral peaks kehtima võrdus
    \begin{align*}
        &\psi \left( \phi_1 \left( \phi_2 \left( \phi_3
            \left( x \right) \right) \right) \right) -
        \psi \left( \phi_1 \left( \phi_3 \left( \phi_2
            \left( x \right) \right) \right) \right) -
        \phi_2 \left( \psi \left( \phi_1 \left( \phi_3
            \left( x \right) \right) \right) \right) + \\
        &\phi_2 \left( \phi_3 \left( \psi \left( \phi_1
            \left( x \right) \right) \right) \right) +
        \phi_3 \left( \psi \left( \phi_1 \left( \phi_2
            \left( x \right) \right) \right) \right) -
        \phi_3 \left( \phi_2 \left( \psi \left( \phi_1
            \left( x \right) \right) \right) \right) = \\
        %
        =\ &\psi \left( \phi_1 \left( \phi_2 \left(
            \phi_3 \left( x \right) \right) \right) \right) -
        \psi \left( \phi_1 \left( \phi_3 \left(
            \phi_2 \left( x \right) \right) \right) \right) -
        \psi \left( \phi_2 \left( \phi_1 \left(
            \phi_3 \left( x \right) \right) \right) \right) + \\
        &\psi \left( \phi_2 \left( \phi_3 \left(
            \phi_1 \left( x \right) \right) \right) \right) +
        \psi \left( \phi_3 \left( \phi_1 \left(
            \phi_2 \left( x \right) \right) \right) \right) -
        \psi \left( \phi_3 \left( \phi_2 \left(
            \phi_1 \left( x \right) \right) \right) \right) + \\
        %
        &\psi \left( \phi_2 \left( \phi_3 \left(
            \phi_1 \left( x \right) \right) \right) \right) -
        \psi \left( \phi_3 \left( \phi_2 \left(
            \phi_1 \left( x \right) \right) \right) \right) -
        \phi_2 \left( \psi \left( \phi_3 \left(
            \phi_1 \left( x \right) \right) \right) \right) + \\
        &\phi_2 \left( \phi_3 \left( \psi \left(
            \phi_1 \left( x \right) \right) \right) \right) +
        \phi_3 \left( \psi \left( \phi_2 \left(
            \phi_1 \left( x \right) \right) \right) \right) -
        \phi_3 \left( \phi_2 \left( \psi \left(
            \phi_1 \left( x \right) \right) \right) \right).
    \end{align*}

    Samas, võttes
    \[
        \psi(x) = x, \quad
        \phi_1 (x) = 1, \quad
        \phi_2 (x) = 2, \quad
        \phi_3 (x) = 3,
    \]
    jääb eelnevas vasakule poolele
    \[ \psi(1) - \psi(1) - 2 + 2 + 3 - 3 = 1 - 1 = 0, \]
    ning paremale poolele
    \begin{align*}
        &\psi(1) - \psi(1) - \psi(2) + \psi(2) + \psi(3) - \psi(3) +
            \psi(2) - \psi(3) - 2 + 2 + 3 - 3 = \\
        =\ &\psi(2) - \psi(3) = 2 - 3 = -1.
    \end{align*}
    $0 \neq -1$, ja seega sellisel kujul üldistus ei kehti.
\end{markus}


\subsection{Tuletatud rda ja keskne kahanev rida}

\textcolor{red}{
    TODO: korralik tõlge tuletatud rea (derived series) ja
    kahaneva rea (central descending series) jaoks
}


Märgime, et $n$-Lie superalgebra korral defineeritakse tema
\emph{ideaal} ning \emph{alamalgebra} täpselt nagu $n$-Lie algebra
korral (vt definitsioone \ref{def:n-lie-alamalgebra} ja
\ref{def:n-lie-algebra-ideaal}).

\begin{lemma}
    Olgu $(\g, [\cdot, \dots, \cdot])$ $n$-Lie superalgebra ja
    $\h_1, \h_2, \dots, \h_n \subseteq \g$ tema ideaalid. Siis
    $\h = [\h_1, \h_2, \dots, \h_n]$ on samuti $\g$ ideaal.
\end{lemma}

\begin{proof}
    Olgu $\h_1, \h_2, \dots, \h_n$ $n$-Lie superalgebra
    $(\g, [\cdot, \dots, \cdot])$ ideaalid ning olgu
    $x_1, x_2, \dots, x_{n-1} \in \g$. Tähistame
    $\h = [\h_1, \h_2, \dots, \h_n]$ ja valime suvalise $h \in \h$.
    Siis leiduvad $h_i \in \h_i$, $i = 1, 2, \dots, n$ nii, et
    $h = [h_1, h_2, \dots, h_n]$ ja gradueeritud Filippovi samasuse
    järgi saame arvutada
    \begin{align*}
        &[x_1, x_2, \dots, x_{n-1}, h] = \\
        &=[x_1, x_2, \dots, x_{n-1}, [h_1, h_2, \dots, h_n]] = \\
        &=\sum_{i = 1}^{n} (-1)^{|\mathrm{h}|_{i-1} |\mathrm{x}|_{n-1}}
            [h_1, \dots, h_{i-1}, [x_1, \dots, x_{n-1}, h_i],
            h_{i+1}, \dots, h_n],
    \end{align*}
    kus $\mathrm{h} = (h_1, h_2, \dots, h_n)$ ja
    $\mathrm{x} = (x_1, x_2, \dots, x_{n-1})$.

    Paneme tähele, et iga $i = 1, 2, \dots, n$ korral $h_i \in \h_i$.
    Kuna $\h_i$ on ideaal siis järeldub sellest, et
    $[x_1, \dots, x_{n-1}, h_i] \in \h_i$. Kokkuvõttes on seega
    $[x_1, x_2, \dots, x_{n-1}, h]$ ruumi $\h$ elementide lineaarkombinatsioon ehk $[x_1, x_2, \dots, x_{n-1}, h] \in \h$.
\end{proof}

Arvestades lemmat saame defineerida järgmised mõisted.

% TODO: vaata siin definitsioonis mõistete tõlked üle
\begin{dfn}
    Olgu $(\g, [\cdot, \dots, \cdot])$ $n$-Lie superalgebra ja $\h$
    tema ideaal. Ideaali $\h$ poolt \emph{tuletatud reaks} nimetatakse
    alamruumide jada $D^p(\h)$, $p \in \N$, kus
    \[
        D^0(\h) = \h \quad \text{ja} \quad
        D^{p+1}(\h) = [D^p(\h), \dots, D^p(\h)].
    \]
    Ideaali $\h$ \emph{keskseks kahanevaks reaks} nimetatakse
    alamruumide jada $C^p(\h)$, $p \in \N$, kus
    \[
        C^0(\h) = \h \quad \text{ja} \quad
        C^{p+1}(\h) = [C^p(\h), \h, \dots, \h].
    \]
\end{dfn}

\begin{dfn}
    Olgu $(\g, [\cdot, \dots, \cdot])$ $n$-Lie superalgebra ja $\h$
    tema ideaal. Me ütleme, et ideaal $\h$ on \emph{lahenduv}, kui
    leidub $p \in \N$ nii, et $D^p(\h) = \{0\}$. Ideaali $\h$
    nimetatakse \emph{nilpotentseks}, kui leidub $p \in \N$
    nii, et $C^p(\h) = \{0\}$.
\end{dfn}

\begin{dfn}
    Me ütleme, et $n$-Lie superalgebra $(\g, [\cdot, \dots, \cdot])$
    on \emph{lihtne}, kui tal ei ole muid ideaale peale $\{0\}$
    ja iseenda ning $D^1(\g) \neq \{0\}$.
\end{dfn}

\begin{lemma}
    Olgu $\g = \g_{\overline{0}} \oplus \g_{\overline{1}}$
    $n$-Lie superalgebra ja $g \in \g_{\overline{0}}$. Siis
    mistahes $x_1, x_2, \dots, x_{n-2} \in \g$ korral on
    Lie sulg arvutatuna elementide $g, g, x_1, x_2, \dots, x_{n-2}$
    suvalisel järjestusel alati null.
\end{lemma}

\begin{proof}
    Olgu $\g = \g_{\overline{0}} \oplus \g_{\overline{1}}$
    $n$-Lie superalgebra ja $g \in \g_{\overline{0}}$ ning olgu
    $x_1, x_2, \dots, x_{n-2} \in \g$ suvalised. Oletame, et me
    tahame arvutada Lie sulgu elementide
    \[ g, g, x_1, x_2, \dots, x_{n-2} \]
    mingil ümberjärjestusel. Kujutuse $[\cdot, \dots, \cdot]$
    kaldsümmeetrilisuse tõttu on meil lõpliku arvu elementide
    järjekorra ümbervahetamisel võimalik jõuda olukorrani, kus
    $g$ ja $g$ on argumentidena üksteise kõrval. Selle protsessi
    tulemusena saame Lie sulu
    \begin{align*}
        (-1)^p [x_{i_1}, \dots, x_{i_k}, g, g,
            x_{i_{k+1}}, \dots, x_{i_{n-2}}],
    \end{align*}
    kus $p \in \N$ tähistab sooritatud ümbervahetamiste arvu
    ja indeksid $i_1, i_2, \dots, i_{n-2} \in \{1, 2, \dots, n-2\}$ on
    paarikaupa erinevad. Seejuures rohkem kordajat $(-1)$ ei esine,
    sest $g \in \g_{\overline{0}}$ ja seega $|g| = \overline{0}$.
    Lõpetuseks võime veel omakorda $g$ ja $g$ positsioonid ära
    vahetada, ning saame
    \begin{align*}
        &(-1)^p [x_{i_1}, \dots, x_{i_k}, g, g,
            x_{i_{k+1}}, \dots, x_{i_{n-2}}] = \\
        &= -(-1)^p (-1)^{|g||g|} [x_{i_1}, \dots,
            x_{i_k}, g, g, x_{i_{k+1}}, \dots, x_{i_{n-2}}] = \\
        &= (-1)^{p+1} [x_{i_1}, \dots,
            x_{i_k}, g, g, x_{i_{k+1}}, \dots, x_{i_{n-2}}],
    \end{align*}
    ehk teisi sõnu
    \[
        [x_{i_1}, \dots, x_{i_k}, g, g, x_{i_{k+1}},
        \dots, x_{i_{n-2}}] =
        -[x_{i_1}, \dots, x_{i_k}, g, g, x_{i_{k+1}},
        \dots, x_{i_{n-2}}],
    \]
    mis saab võimalik olla vaid juhul, kui
    \begin{align*}
        [x_{i_1}, \dots, x_{i_k}, g, g, x_{i_{k+1}},
        \dots, x_{i_{n-2}}] = 0.
    \end{align*}
    Siis on aga ka esialgne sulg null.
\end{proof}

\begin{lau}
    Olgu $\g = \g_{\overline{0}} \oplus \g_{\overline{1}}$ varustatuna
    sulug $[\cdot, \dots, \cdot] \col \g^n \to \g$ $n$-Lie
    superalgebra ja olgu $g \in \g_{\overline{0}}$. Varustades
    supervektorruumi $\g$ sulga
    $[\cdot, \dots, \cdot]_g \col \g^{n-1} \to \g$, mis on
    defineeritud valemiga
    \[
        [x_1, x_2, \dots, x_{n-1}]_g = [g, x_1, x_2, \dots, x_{n-1}],
        \quad x_1, x_2, \dots, x_n \in \g,
    \]
    saame $(n-1)$-Lie superalgebra.
\end{lau}

\begin{proof}
    Olgu lause eeldused täidetud ning olgu
    $x_1, x_2, \dots, x_{n-1} \in \g$. Kuna
    $g \in \g_{\overline{0}}$, siis $|g| = \overline{0}$, ja seega
    \[
        |[x_1, x_2, \dots, x_{n-1}]_g =
        |[g, x_1, x_2, \dots, x_{n-1}]| =
        \sum_{i=1}^{n-1} |x_i| + |g| =
        \sum_{i=1}^{n-1} |x_i|.
    \]
    Kaldsümmeetrilisuseks märgime, et
    \begin{align*}
        &[x_1, \dots, x_i, x_{i+1}, \dots, x_{n-1}]_g = \\
        &=[g, x_1, \dots, x_i, x_{i+1}, \dots, x_{n-1}] = \\
        &=-(-1)^{|x_i||x_{i+1}|}
            [g, x_1, \dots, x_{i+1}, x_i, \dots, x_{n-1}] = \\
        &=-(-1)^{|x_i||x_{i+1}|}
            [x_1, \dots, x_{i+1}, x_i, \dots, x_{n-1}]_g.
    \end{align*}

    Veendumaks, et kehtib ka gradueeritud Filippovi samasus
    eeldame veel, et $y_1, y_2, \dots, y_{n-2} \in \g$ ja tähistame
    $\mathrm{x} = (g, x_1, \dots, x_{n-1})$ ning
    $\mathrm{y} = (g, y_1, \dots, y_{n-2})$. Siis
    \begin{align*}
        & [y_1, \dots, y_{n-2}, [x_1, \dots, x_{n-1}]_g]_g = \\
        &= [g, y_1, \dots, y_{n-2}, [g, x_1, \dots, x_{n-1}]] = \\
        &= \sum_{i=1}^{n-1} (-1)^{
            |\mathrm{x}|_{i-1} |\mathrm{y}|_{n-1}
        } [g, x_1, \dots, x_{i-1}, [g, y_1, \dots, y_{n-2}, x_i],
            x_{i+1}, \dots, x_{n-1}] + \\
        &\quad\ [[g, y_1, \dots, y_{n-2}, g], x_1, \dots, x_{n-1}] = \\
        &= \sum_{i=1}^{n-1} (-1)^{
            |\mathrm{x}|_{i-1} |\mathrm{y}|_{n-1}
        } [x_1, \dots, x_{i-1}, [y_1, \dots, y_{n-2}, x_i]_g,
            x_{i+1}, \dots, x_{n-1}]_g,
    \end{align*}
    sest lemma järgi $[g, y_1, \dots, y_{n-2}, g] = 0$.
\end{proof}

\subsection{Indutseeritud \texorpdfstring{$n$}\ -Lie superalgebra}

Tuginedes artiklile \cite{Abramov:2014} rakendame punktis \ref{subsec:indutseeritud-n-lie-alg} tutvustatud eeskirja $n$-Lie superalgebra
abil $(n+1)$-Lie superalgebra konstrueerimiseks.

\begin{dfn}
    Olgu $\V = \V_{\overline{0}} \oplus \V_{\overline{1}}$
    supervektorruum ning olgu $\phi \col \V^n \to \V$.
    Lineaarkujutust $S \col \V \to K$ nimetatakse $\phi$
    \emph{superjäljeks}, kui
    \begin{enumerate}[label=\arabic*)]
        \item iga $x_1, x_2, \dots, x_n \in \V$ korral
            $S\left(\phi(x_1, x_2, \dots, x_n)\right) = 0$,
        \item $S(x) = 0$ iga $x \in \V_{\overline{1}}$.
    \end{enumerate}
\end{dfn}

\begin{markus}
    Superjälje definitsioonis antud teine tingimus on ajendatud
    asjaolust, et blokkmaatriksi
    \[
        A = \begin{pmatrix}
            A_{00} & A_{01} \\
            A_{10} & A_{11}
        \end{pmatrix}
    \]
    korral arvutatakse superjälg valemiga
    $S(A) = \Tr(A_{00}) - \Tr(A_{11})$, kus $\Tr$ on tavaline
    maatriksi jälg, ja $A_{00}, A_{11}$ moodustavad paarisosa
    ning $A_{01}, A_{10}$ moodustavad paaritu
    osa. Selge, et kui sellise maatriksi paarisosa on null, siis
    on tegu paaritu maatriksiga, ning tema superjälg on null.
\end{markus}

Vaatleme $n$-lineaarset kujutus $\phi \col \V^n \to \V$, mis
suvaliste homogeensete elementide $x_1, x_2, \dots, x_n \in \V$ korral
rahuldab tingimusi
\begin{gather}
    |\phi(x_1, \dots, x_n)| = \sum_{i=1}^n |x_i|,
        \label{eq:phi-grading}\\
    \phi \left( x_1, \dots, x_i, x_{i+1}, \dots, x_n \right) =
        -(-1)^{ |x_i| |x_{i+1}| } \phi \left(
            x_1, \dots, x_{i+1}, x_i, \dots, x_n
        \right), \label{eq:phi-graded-anticomm}
\end{gather}
see tähenab $\phi$ on kooskõlas gradueeringutega ning
kaldsümmeetriline gradueeritud mõttes. Lisaks olgu meil antud
kujutusele $\phi$ vastav superjälg $S \col \V \to K$.
Kasutades kujutusi $\phi$ ja $S$ defineerime uue kujutuse
$\phi_S \col \G^{n+1} \to G$, mis homogeensete elementide
jaoks on määratud valemiga
\begin{align}\label{def:phi-s}
    \phi_S (x_1, \dots, x_{n+1}) =
    \sum_{i=1}^{n+1} (-1)^{i-1}(-1)^{|x_i| |\mathbf{x}|_{i-1} }
        S(x_i) \phi \left(
            x_1, \dots, \hat{x_i}, \dots, x_{n+1}
        \right),
\end{align}
kus $|\mathbf{x}|_{i} = \sum_{j=1}^{i} |x_j|$.
Mittehomogeensete elementide korral vaatleme eraldi tema paaris- ja
paaritut osa, ning rakendame kujutuse \eqref{def:phi-s} lineaarsust
ja arvutame $\phi_S$ sel viisil.

Saadud kujutuse tähtsamad omadused võtab kokku järgmine lemma.

\begin{lemma}\label{lemma:indutseeritud-sulu-omadused}
    Olgu $x_1, x_2, \dots, x_{n+1} \in \V$ homogeensed elemendid.
    Siis $(n+1)$-lineaarne kujutus $\phi_S \col \G^{n+1} \to \g$
    rahuldab järgmisi tingimusi:
    \begin{enumerate}[label=\arabic*)]
        \item $ |\phi_S \left(x_1, \dots, x_{n+1} \right)| =
                \sum_{i=1}^{n+1} |x_i|$,
        \item $ \phi_S \left(x_1, \dots, x_i, x_{i+1}, \dots, x_{n+1} \right) =
               -(-1)^{|x_i| |x_{i+1}|} \phi_S \left(
                    x_1, \dots, x_{i+1}, x_i, \dots, x_{n+1}
                \right) $,
        \item $S \left( \phi_S \left( x_1, \dots, x_{n+1} \right) \right) = 0$.
    \end{enumerate}
\end{lemma}

\begin{proof}
    Olgu $\V$ supervektorruum, rahuldagu $n$-lineaarne
    kujutus $\phi \col \V^n \to \V$ tingimusi
    \eqref{eq:phi-grading} ja \eqref{eq:phi-graded-anticomm}, ning
    olgu $S \col \V \to K$ kujutuse $\phi$ superjälg. Kui
    Lisaks olgu $x_1, x_2, \dots, x_{n+1} \in \V$ suvalised
    homogeensed elemendid ja tähistame
    $\mathrm{x} = (x_1, x_2, \dots, x_{n+1})$.
    \begin{enumerate}[label=\arabic*)]
        % TODO: add proof for the first statement
        \item \dots

        \item Rakendame valemit \eqref{def:phi-s} ja arvutame
            \begin{align*}
                &\phi_S(x_1, \dots, x_i, x_{i+1}, \dots, x_{n+1}) = \\
                &= \sum_{j=1}^{n+1}
                    (-1)^{j-1} (-1)^{|x_j||\mathrm{x}|_{j-1}}
                    S(x_j) \phi_j(
                        x_1, \dots, x_i, x_{i+1}, \dots, x_{n+1}
                    ) = \\
                &= \sum_{\mathclap{
                        \substack{j=1\\j \ne i,\ j \ne i+1}}
                    }^{n+1}
                    (-1)^{j-1} (-1)^{|x_j||\mathrm{x}|_{j-1}}
                    S(x_j) \phi_j(
                        x_1, \dots, x_i, x_{i+1}, \dots, x_{n+1}
                    ) + \\
                &\quad\, +
                    (-1)^{i-1} (-1)^{|x_i||\mathrm{x}|_{i-1}}
                    S(x_i) \phi(
                        x_1, \dots, \hat{x_i}, x_{i+1}, \dots, x_{n+1}
                    ) + \\
                &\quad\, +
                    (-1)^{i} (-1)^{|x_{i+1}||\mathrm{x}|_{i}}
                    S(x_{i+1}) \phi(
                        x_1, \dots, x_i, \hat{x_{i+1}}, \dots, x_{n+1}
                    ) = \\
                &= -(-1)^{|x_i||x_{i+1}|} \sum_{\mathclap{
                        \substack{j=1\\j \ne i,\ j \ne i+1}}
                    }^{n+1}
                    (-1)^{j-1} (-1)^{|x_j||\mathrm{x}|_{j-1}}
                    S(x_j) \phi_j(
                        x_1, \dots, x_{i+1}, x_i, \dots, x_{n+1}
                    ) - \\
                &\quad
                    -1(-1)^i (-1)^{
                        |x_i||\mathrm{x}|_{i-1} + |x_i||x_{i+1}|
                    } (-1)^{|x_i||x_{i+1}|}
                    S(x_i) \phi(
                        x_1, \dots, x_{i+1}, \hat{x_i}, \dots, x_{n+1}
                    ) - \\
                &\quad
                    -(-1)^{i-1} (-1)^{|x_{i+1}||\mathrm{x}|_{i-1}}
                    (-1)^{|x_{i+1}||x_i|}
                    S(x_{i+1}) \phi(
                        x_1, \dots, \hat{x_{i+1}}, x_i, \dots, x_{n+1}
                    ) = \\
                &= -(-1)^{|x_i||x_{i+1}|} \phi_S(
                    x_1, \dots, x_{i+1}, x_i, \dots, x_{n+1}
                )
            \end{align*}

        \item Kuna $S$ on kujutuse $\phi$ superjälg, siis $S$ on
            lineaarne ja $S(\phi(y_1, \dots, y_n)) = 0$ mistahes
            elementide $y_1, y_2, \dots, y_n \in \V$ korral. Seega
            \begin{align*}
                & S\left(\phi_S(x_1, x_2, \dots, x_{n+1})\right) = \\
                &= S\left( \sum_{i=1}^{n+1}
                    (-1)^{i-1} (-1)^{|x_i||\mathrm{x}|_{i-1}}
                    S(x_i) \phi(
                        x_1, \dots, \hat{x_i}, \dots, x_{n+1}
                    )
                    \right) = \\
                &= \sum_{i=1}^{n+1}
                    (-1)^{i-1} (-1)^{|x_i||\mathrm{x}|_{i-1}}
                    S(x_i) S(\phi(
                        x_1, \dots, \hat{x_i}, \dots, x_{n+1}
                    )) = \\
                &= \sum_{i=1}^{n+1} 0 = 0,
            \end{align*}
            nagu tarvis. \qedhere
    \end{enumerate}
\end{proof}

Artiklis \cite{Abramov:2014} on toodud teoreem, mis ütleb, kuidas
$n$-Lie superalgebrast superjälje abil $(n+1)$-Lie superalgebra
indutseerida saab.

\begin{thm}
    Olgu $\g = \g_{\overline{0}} \oplus \g_{\overline{1}}$
    $n$-Lie superalgebra suluga
    $\nbrac{\cdot}{\cdot} \col \g^n \to \g$ ning olgu
    $S \col \g \to K$ sulu $[\cdot, \dots, \cdot]$ superjälg.
    Defineerides $(n+1)$-lineaarse sulu
    $[\cdot, \dots, \cdot]_S \col \g^{n+1} \to \g$ valemiga
    \begin{align}\label{eq:indutseeritud-n-lie-superalgebra}
        \nbrac{x_1}{x_{n+1}}_S = \sum_{i=1}^{n+1}
        (-1)^{i-1} (-1)^{|x_i| |\mathrm{x}|_{i-1}} S(x_i)
        \left[ x_1, \dots, \hat{x_i}, \dots, x_{n+1} \right],
    \end{align}
    on supervektorruum $\g$, varustatuna suluga
    \eqref{eq:indutseeritud-n-lie-superalgebra} $(n+1)$-Lie
    superalgebra.
\end{thm}

Selle teoreemi valguses on selge, et lausetega
\ref{lause:indutseeritud-alamalgebra} ja
\ref{lause:indutseeritud-ideaal} analoogilised tulemused
kehtivad ka $n$-Lie superalgebrate korral.

\begin{lau}
    Olgu $(\g, [\cdot, \dots, \cdot])$ $n$-Lie superalgebra ning
    olgu $\h \subset \g$ alamalgebra. Kui $S$ on
    sulu $[\cdot, \dots, \cdot]$ superjälg,
    siis $\h$ on ka $(\g, [\cdot, \dots, \cdot]_S)$ alamalgebra.
    \hfill \qed
\end{lau}

\begin{lau}
    Olgu $(\g, [\cdot, \dots, \cdot])$ ideaal $\h$ ja olgu
    $S$ sulu $[\cdot, \dots, \cdot]$ superjälg. Siis $\h$ on
    $(\g, [\cdot, \dots, \cdot]_S)$ ideaal parajasti siis, kui
    $[\g, \g, \dots, \g] \subseteq \h$ või $\h \subseteq \ker S$.
    \hfill \qed
\end{lau}

\begin{lau}
    Olgu $(\g, [\cdot, \dots, \cdot])$ $n$-Lie superalgebra ja
    $S$ sulu $[\cdot, \dots, \cdot]$ superjälg. Siis indutseeritud
    $(n+1)$-Lie superalgebra $(\g_S, [\cdot, \dots, \cdot]_S)$
    on lahenduv.
\end{lau}

\begin{proof}
    Eeldame, et $(\g, [\cdot, \dots, \cdot])$ on $n$-Lie
    superalgebra ja $S$ tema sulu $[\cdot, \dots, \cdot]$ superjälg.
    Olgu
    \[
        x_1, x_2, \dots, x_{n+1} \in
        [\g, \g, \dots, \g]_S = D^1(\g_S).
    \]
    Siis iga $i = 1, 2, \dots, n+1$ korral leiduvad
    $x_i^1, x_i^2, \dots, x_i^{n+1} \in \g$ nii, et
    \[
        x_i = [x_i^1, x_i^2, \dots, x_i^{n+1}]_S.
    \]
    Sel juhul
    \begin{align*}
        &[x_1, x_2, \dots, x_{n+1}]_S = \\
        &= \sum_{i=1}^{n+1} (-1)^{i-1} (-1)^{|x_i||x|_{i-1}}
            S(x_i) [x_1, \dots, \hat{x_i}, \dots, x_{n+1}] = \\
        &= \sum_{i=1}^{n+1} (-1)^{i-1} (-1)^{|x_i||x|_{i-1}}
            S([x_i^1, x_i^2, \dots, x_i^{n+1}]_S)
            [x_1, \dots, \hat{x_i}, \dots, x_{n+1}] = \\
        &= 0,
    \end{align*}
    sest lemma \ref{lemma:indutseeritud-sulu-omadused} järgi
    iga $i = 1, 2, \dots, n+1$ korral
    $S([x_i^1, x_i^2, \dots, x_i^{n+1}]_S) = 0$.
\end{proof}

Lause tõestusest võime teha järelduse.

\begin{jar}
    Kui $S$ on $n$-Lie superalgebra $(\g, [\cdot, \dots, \cdot])$
    Lie sulu superjälg ja $(\g_S, [\cdot, \dots, \cdot]_S)$ on
    vastav indutseeritud $(n+1)$-Lie superalgebra, siis
    $D^p(\g_S) = \{0\}$, kui $p \geq 2$. \hfill \qed
\end{jar}

\begin{lau}
    Olgu $n$-Lie superalgebra $(\g, [\cdot, \dots, \cdot])$
    kommutaatori jälg $S$ ja olgu $(\g_S, [\cdot, \dots, \cdot]_S)$
    neile vastav indutseeritud $(n+1)$-Lie superalgebra.
    Tähistagu $\left(C^p(\g)\right)_{p=0}^\infty$ algebra
    $(\g, [\cdot, \dots, \cdot])$
    keskset kahanevat jada, ning tähistagu
    $\left(C^p(\g_S)\right)_{p=0}^\infty$ algebra
    $(\g, [\cdot, \dots, \cdot]_S)$
    keskset kahanevat jada. Siis iga $p \in \N$ korral
    \begin{align*}
        C^p(\g_S) \subseteq C^p(\g).
    \end{align*}
    Enamgi veel, kui leidub $g \in \g$ nii, et iga
    $x_1, x_2, \dots, x_n \in \g$ korral
    $[g, x_1, \dots, x_n]_S = [x_1, \dots, x_n]$, siis
    kõikide $p \in \N$ jaoks kehtib võrdus
    \begin{align*}
        C^p(\g_S) = C^p(\g).
    \end{align*}
\end{lau}

\begin{proof}
    Viime tõestuse läbi kasutades matemaatilist induktsiooni.
    Baasjuht $p = 0$ on triviaalne, sest $C^0(\g) = \g$ ja
    samuti $C^0(\g_S) = \g$.

    Vaatleme juhtu $p = 1$. Siis iga
    $x = [x_1, x_2, \dots, x_{n+1}]_S \in C^1(\g_S)$ korral
    \[
        x = \sum_{i=1}^{n+1} (-1)^{i-1}(-1)^{|x_i||\mathrm{x}|_{i-1}}
            S(x_i) [x_1, \dots, \hat{x_i}, \dots, x_{n+1}],
    \]
    kus $x = (x_1, x_2, \dots, x_{n+1})$. See aga tähendab, et
    $x$ on $C^1(\g)$ elementide lineaarkombinatsioon ja järelikult
    $x \in C^1(\g)$.

    Oletame nüüd, et leidub $g \in \g$ nii, et suvaliste
    $y_1, y_2, \dots, y_n \in \g$ korral
    \[
        [g, y_1, \dots, y_n]_S = [y_1, \dots, y_n].
    \]
    Siis aga võib $x = [x_1, \dots, x_n] \in C^1(\g)$
    esitada kujul $[g, x_1, \dots, x_n]_S$, mis tähendab, et
    $x \in C^1(\g_S)$.

    Oletame nüüd, et väide kehtib mingi $p \in \N$ korral ja
    vaatleme elementi $x \in C^{p+1}(\g_S)$. Siis leiduvad
    $x_1, x_2, \dots, x_n \in \g$ ja $g \in C^p(\g_S)$ nii, et
    \begin{align*}
        x &= [g, x_1, x_2, \dots, x_n]_S = \\
        &= -(-1)^{|g||x_1|} [x_1, g, x_2, \dots, x_n]_S = \\
        &= \dots = \\
        &= (-1)^n (-1)^{|a||\mathrm{x}|_n}
            [x_1, x_2, \dots, x_n, g]_S = \\
        &= (-1)^{n + |a||\mathrm{x}|_n}
            \sum_{i=1}^{n} (-1)^{i-1} (-1)^{|x_i||\mathrm{x}|_{i-1}}
                S(x_i) [x_1, \dots, \hat{x_i}, \dots, x_n, a] + \\
        &\quad\ (-1)^{n + |a||\mathrm{x}|_n}
            (-1)^n (-1)^{|a||\mathrm{x}|_n} S(g) [x_1, \dots, x_n] = \\
        &= (-1)^{n + |a||\mathrm{x}|_n} \sum_{i=1}^{n}
            (-1)^{i-1} (-1)^{|x_i||\mathrm{x}|_{i-1}}
            S(x_i) [x_1, \dots, \hat{x_i}, \dots, x_n, g],
    \end{align*}
    sest $S(g) = 0$, kuna $g \in C^p(\g_S)$ ja seega on ta esitatav
    mingite elementide Lie suluna. Teisalt, et $g \in C^p(\g_S)$,
    siis meie induktiivse eelduse põhjal $g \in C^p(\g)$, ja
    seega $x \in C^{p+1}(\g)$.

    Tõestuse lõpetuseks eeldame, et leidub $g \in \g$ nii, et
    iga $y_1, y_2, \dots, y_n \in \g$ korral
    \[
        [g, y_1, \dots, y_n]_S = [y_1, \dots, y_n].
    \]
    Kui nüüd $x \in C^{p+1}(\g)$, siis
    $x = [h, x_1, \dots, x_{n-1}]$, kus
    $x_1, x_2, \dots, x_{n-1} \in \g$ ja $h \in C^p(\g)$.
    Kokkuvõttes saame, et
    \[
        x = [h, x_1, \dots, x_{n-1}] =
        [g, h, x_1, \dots, x_{n-1}]_S =
        -(-1)^{|g||h|} [h, g, x_1, \dots, x_{n-1}]_S.
    \]
    Samas $h \in C^p(\g) = C^p(\g_S)$ ja seega
    $[h, g, x_1, \dots, x_{n-1}]_S \in C^{p+1}(\g_S)$, millest
    saame, et $x \in C^{p+1}(\g_S)$.
\end{proof}
