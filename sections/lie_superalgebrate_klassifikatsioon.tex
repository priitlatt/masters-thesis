%!TEX root = ../thesis.tex

%%%%%%%%%%%%%%%%%%%%%%%
%% Klassifikatsioon  %%
%%%%%%%%%%%%%%%%%%%%%%%

\section[Madaladimensionaalsete \texorpdfstring{$3$}{3}-Lie superalgebrate
klassifikatsioon]{Madaladimensionaalsete \texorpdfstring{$3$}{3}-Lie
superalgebrate \\ klassifikatsioon}

\subsection{Meetodi kirjeldus}

Järgnevas peatükis uurime kui palju on erinevaid $3$-Lie superalgebraid üle
supervektorruumi, mille kombineeritud dimensioon on väiksem kui $4$.
Klassifikatsiooni koostamiseks kasutame Lie superalgebra struktuurikonstante
ja teeme seda järgmisel viisil.

Oletame, et meil on 3-Lie superalgebra $(\g, [\cdot, \cdot, \cdot])$,
kusjuures supervektorruum $\g$ on lõpik ning tema dimensioon on $m|n$. Olgu
supervektorruumil $\g$ fikseeritud mingi baas $\B$. Arvestades dimensiooni
$m|n$, saame baasi kirjutada kujul
\[
    \B = \left\{ e_1, e_2, \dots, e_m, f_1, f_2, \dots, f_n \right\} =
         \left\{ z_1, z_2, \dots, z_{m+n} \right\},
\]
kus $e_\alpha$, $1 \leq \alpha \leq m$, on paaris baasivektorid ja
$f_i$, $1 \leq i \leq n$, on paaritud baasivektorid, ning $z_A$,
$1 \leq A \leq m+n$, võib olla kas paaris- või paarituvektor.

Edasi, kasutades teadmist, et $|[z_1, z_2, z_3]| = |z_1| + |z_2| + |z_3|$,
avaldame kommutaatori $[\cdot, \cdot, \cdot]$ väärtused baasivektoritel
struktuurikonstantide $K_{ABC}^D$ abil:
\begin{align*}
    &[e_\alpha, e_\beta, e_\gamma] =
        K_{\alpha \beta \gamma}^\lambda e_\lambda, \\
    &[e_\alpha, e_\beta, f_i] = K_{\alpha \beta i}^j f_j, \\
    &[e_\alpha, f_i, f_j] = K_{\alpha i j}^\beta e_\beta, \\
    &[f_i, f_j, f_k] = K_{i j k}^l f_l,
\end{align*}
kus $\alpha \leq \beta \leq \gamma$ ja $i \leq j \leq  k$.
Seejuures paneme tähele, et rohkematel kui kirjutatud baasivektorite
kombinatsioonidel ei ole mõtet kommutaatori väärtusi arvutada kuna nad ei
sisalda uut informatsiooni, sest kaldsümmeetrilisuse abil on võimalik
sulg $[f_j, e_\alpha, f_i]$, $i < j$, viia alati kujule $[e_\alpha, f_i, f_j]$,
ning jõuda juba saadud tulemuseni.

Otsime nüüd välja millised sulud on tegelikult võrdsed nullvektoriga. Selleks
permuteerime kaldsümmeetrilisust kasutades kommutaatori argumente, ning kui
jõuame samasuguse tulemuseni kui oli esialgne sulg, kuid erineva märgiga,
peab kogu sulg võrduma nulliga, sest vektor on võrdne oma vastandvektoriga
siis ja ainult siis, kui tegu on nullvektoriga. Vaatame näiteks sulgu
$[e_1, e_1, f_i]$. Siis ilmselt
\[
    [e_1, e_1, f_i] = -(-1)^{|e_1||e_1|} [e_1, e_1, f_i] =
    -(-1)^0 [e_1, e_1, f_i] = - [e_1, e_1, f_i],
\]
ehk $[e_1, e_1, f_i] = 0$.
Sellega oleme ära kasutanud $n$-Lie superalgebra definitsioonis nõutud
gradueeringute kooskõla ehk tingimuse \eqref{def:n-lie-superalg-brac-grading}
ja samuti oleme juba tarvitusele võtnud kaldsümmeetrilisuse nõude ehk
tingimuse \eqref{def:n-lie-superalg-brac-antikomm}. Järele jääb veel kasutada
gradueeritud Filippovi samasus \eqref{id:graded-filippov}.