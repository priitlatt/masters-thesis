%!TEX root = ../thesis.tex

%%%%%%%%%%%%%%%%%%%%%%%%%
%% n-Lie superalgebra  %%
%%%%%%%%%%%%%%%%%%%%%%%%%

\section{\texorpdfstring{$n$}\ -Lie superalgebra}

See peatükk tugineb artiklile \cite{Abramov:2014}

Järgnevas eeldame, et meil on antud supervektorruum
ehk supervektorruum $\G = \G_{\overline{0}} \oplus \G_{\overline{1}}$
ning $n$-lineaarne kujutus $\phi \col \G^n \arr \G$, mis rahuldab
tingimusi
\begin{itemize}
    \item $| \phi(x_1, \dots, x_n) | = \sum_{i=1}^n |x_i|$,
    \item $ \phi \left( x_1, \dots, x_i, x_{i+1}, \dots, x_n \right) =
            -(-1)^{ |x_i| |x_{i+1}| } \phi \left(
                x_1, \dots, x_{i+1}, x_i, \dots, x_n \right), $
\end{itemize}
kus $|x| \in \left\{ \overline{0}, \overline{1} \right\}$
tähistab elemendi $x$ paartust. Samuti eeldame, et $S \col \G \arr \K$
on lineaarne kujutus, mis rahuldab
\begin{itemize}
    \item $S \left( \phi \left( x_1, \dots, x_n \right) \right) = 0$,
    \item $S(x) = 0$ iga $x \in \G_{\overline{1}}$.
\end{itemize}

Selge, et siin sisse toodud kujutused $\phi$ ja $S$ on eelnevas
kirjeldatu analoogid supervektorruumis. Seejuures kujutust $S \col \G \arr \K$
nimetatakse \emph{superjäljeks}.

Kasutades kujutusi $\phi$ ja $S$ defineerime analoogiliselt
vektorruumide situatsioonile, kuid nüüd juba supervektorruumi iseärasusi
arvesse võttes, see tähendab paarsusi arvestades, uue kujutuse
$\phi_S \col \G^{n+1} \arr G$ valemiga
\[
    \phi_S (x_1, \dots, x_{n+1}) =
    \sum_{i=1}^{n+1} (-1)^{i-1}(-1)^{|x_i| \sum_{j=1}^{i-1} |x_j| }
        S(x_i) \phi \left(
            x_1, \dots, \hat{x_i}, \dots, x_{n+1}
        \right).
\]

Saadud kujutuse tähtsamad omadused võtab kokku järgmine oluline lemma:

\begin{lemma}
    $(n+1)$-lineaarne kujutus $\phi_S \col \G^{n+1} \arr \G$
    rahuldab tingimusi
    \begin{enumerate}
        \item $ | \phi_S \left(x_1, \dots, x_{n+1} \right) | =
               \sum_{i=1}^{n+1} |x_i| $,
        \item $ \phi_S \left(x_1, \dots, x_i, x_{i+1}, \dots, x_{n+1} \right) =
               -(-1)^{|x_i| |x_{i+1}|} \phi_S \left(
                    x_1, \dots, x_{i+1}, x_i, \dots, x_{n+1}
                \right) $,
        \item $S \left( \phi_S \left( x_1, \dots, x_{n+1} \right) \right)$.
    \end{enumerate}
\end{lemma}

Üldistame nüüd definitsiooni~\ref{def:lie_algebra} supervektorruumi jaoks
ning defineerime \emph{$n$-Lie superalgebra}.

\begin{dfn}[$n$-Lie superalgebra]
    Olgu $\G = \G_{\overline{0}} \oplus \G_{\overline{1}}$ 
    supervektorruum. Me ütleme, et $\G$ on
    \emph{$n$-Lie superalgebra}, kui $\G$ on varustatud
    gradueeritud $n$-Lie suluga $\nbrac{\cdot}{\cdot} \col \G^n \arr \G$,
    mis rahuldab tingimusi
    \begin{enumerate}
        \item $\left| \nbrac{x_1}{x_n} \right| = \sum_{i=1}^n |x_i| $,
        \item $\left[ x_1, \dots, x_i, x_{i+1}, \dots, x_n \right] =
            -(-1)^{|x_i| |x_{i+1}|} \left[
                x_1, \dots, x_{i+1}, x_i, \dots, x_n
            \right]$,
        \item $\left[ y_1, \dots, y_{n-1}, \nbrac{x_1}{x_n} \right] = $ \\
            $ = \sum_{i=1}^n (-1)^{\tau_x (i-1) \tau_y (n-1)}
            \left[
                x_1, \dots, x_{i-1},
                \left[ y_1, \dots, y_{n-1}, x_i \right],
                x_{i+1}, \dots, x_n
            \right] $,
    \end{enumerate}
    kus $x = (x_1, \dots, x_n)$ ja $y = (y_1, \dots, y_{n-1})$ ning
    $\tau_x (k) = \sum_{j=1}^{k-1} |x_i|$.
\end{dfn}

Võttes arvesse $n$-Lie superalgebra definitsiooni saame sõnastada 
teoreemi \ref{thm:n+1_lie_alg} superanaloogi järgmiselt:

\begin{thm}
    Olgu $\G = \G_{\overline{0}} \oplus \G_{\overline{1}}$
    $n$-Lie superalgebra suluga $\nbrac{\cdot}{\cdot} \col \G^n \arr \G$,
    ning $V$ lõplikumõõtmeline vektorruum ja olgu
    antud $\G$ esitus $\phi \col \G \arr \gl V$. Defineerides
    $\nbrac{\cdot}{\cdot} \col \G^{n+1} \arr \G$ valemiga
    \[
        \nbrac{x_1}{x_{n+1}} = \sum_{i=1}{n+1}
        (-1)^{i-1} (-1)^{|x_i| \tau_x (i-1)} S(\phi(x_i))
        \left[ x_1, \dots, \hat{x_i}, \dots, x_{n+1} \right],
    \]
    on supervektorruum $\G$, varustatuna suluga
    $\nbrac{\cdot}{\cdot} \col \G^{n+1} \arr \G$ $(n+1)$-Lie
    superalgebra.
\end{thm}
