%!TEX root = ../thesis.tex

%%%%%%%%%%%%%%%%%%%%%%%%%%%%%%%%%%%%%%%%%%%%%
%%  Klassikaline teooria ehk Lie algebrad  %%
%%%%%%%%%%%%%%%%%%%%%%%%%%%%%%%%%%%%%%%%%%%%%

\section{Lie algebra}

Matemaatika haru, mida me täna tunneme kui \emph{Lie teooriat} kerkis esile
geomeetria ja lineaaralgebra uurimisest. Lie teooria üheks keskseks mõisteks
on \emph{Lie algebra}, see on vektorruum, mis on varustatud
mitteassotsiatiivse korrutamisega ehk nõndanimetatud \emph{Lie suluga}.
Lie algebrad ja nende uurimine on tihedalt seotud teise Lie teooria keskse
mõistega, milleks on \emph{Lie rühm}\cite{johan1989survey}. Viimased on
struktuurid, mis on korraga nii algebralised rühmad kui ka topoloogilised
muutkonnad, kusjuures rühma korrutamine ja selle pöördtehe on mõlemad
pidevad. Osutub, et igale Lie rühmale saab vastavusse seada Lie algebra ja
kehtib ka vastupidine pisut nõrgem tulemus. Nimelt suvalise lõplikumõõtmelise
reaalsele või komplekssele Lie algebra jaoks leidub temale üheselt
vastav sidus Lie rühm.\cite{kirillov2008introduction} Just selle viimase,
\emph{Lie kolmanda teoreemi} tõttu on võimalik Lie rühmasid vaadelda
Lie algebrate kontekstis ja see teebki Lie algebrad äärmiselt oluliseks.

Tähistagu kõikjal järgnevas $\K$ nullkarakteristikaga korpust ning $V$
vektorruum üle korpuse $\K$.