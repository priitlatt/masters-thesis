%!TEX root = ../../thesis.tex

\subsection{\texorpdfstring{$0|2$}{0|2}-dimensionaalsed
    \texorpdfstring{$3$}{3}-Lie superalgebrad}

Olgu $3$-Lie superalgebra $\g$ supervektorruumi dimensioon
$0|2$ ja olgu tema generaatoriteks $\{ f_1, f_2 \}$. Siis
\begin{align*}
    \begin{matrix*}[l]
        & [f_1, f_1, f_1] = \m_1 f_1 + \m_2 f_2, \quad
        & [f_1, f_2, f_2] = \m_5 f_1 + \m_6 f_2, \\
        & [f_1, f_1, f_2] = \m_3 f_1 + \m_4 f_2, \quad
        & [f_2, f_2, f_2] = \m_7 f_1 + \m_8 f_2,
    \end{matrix*}
\end{align*}
kus $\m_1, \m_2, \dots, \m_8 \in \C$ on suvalised skalaarid.

Rakendades neile kommutaatoritele meie algoritmi saame $24$
mittetriviaalset kitsendust kitsendust\footnote{Need on leitavad veebilehelt
\url{http://priitlatt.github.io/masters-thesis/kitsendused/0_2.txt}},
mida $\g$ struktuurikonstandid rahuldama peavad. Kitsenduste
mittetriviaalseteks lahenditeks sobivad struktuurikonstantide komplektid
\renewcommand\arraystretch{1.3}
\begin{align*}
    \left\{ \begin{matrix*}[l]
        \m_1 = -c_1 \\
        \m_2 = \frac{c_1^2}{c_2} \\
        \m_3 = -c_2 \\
        \m_4 = c_1 \\
        \m_5 = -\frac{c_2^2}{c_1} \\
        \m_6 = c_2 \\
        \m_7 = -\frac{c_2^3}{c_1^2} \\
        \m_8 = \frac{c_2^2}{c_1} \\
    \end{matrix*} \right.,
    \quad
    \left\{ \begin{matrix*}[l]
        \m_1 = 0 \\
        \m_2 = c_3 \\
        \m_3 = 0 \\
        \m_4 = 0 \\
        \m_5 = 0 \\
        \m_6 = 0 \\
        \m_7 = 0 \\
        \m_8 = 0 \\
    \end{matrix*} \right.,
    \quad
    \left\{ \begin{matrix*}[l]
        \m_1 = 0 \\
        \m_2 = 0 \\
        \m_3 = 0 \\
        \m_4 = 0 \\
        \m_5 = 0 \\
        \m_6 = 0 \\
        \m_7 = c_4 \\
        \m_8 = 0 \\
    \end{matrix*} \right.,
\end{align*}
\renewcommand\arraystretch{1}
kus $c_1, c_2, c_3, c_4 \in \C \setminus \{0\}$ on vabad muutujad.

\begin{lau} \label{lause:0|2 seosed}
    Olgu $3$-Lie superalgebra $\g$ supervektorruumi dimensioon on $0|2$.
    Siis $\g$ on kas Abeli Lie superalgebra, või tema mittetriviaalsed
    kommutatsiooniseosed on kujul
    \renewcommand\arraystretch{1.5}
    \begin{align}
        & \left\{
            \begin{matrix*}[l]
                [f_1, f_1, f_1] = -c_1 f_1 + \frac{c_1^2}{c_2} f_2, \\
                [f_1, f_1, f_2] = -c_2 f_1 + c_1 f_2, \\
                [f_1, f_2, f_2] = -\frac{c_2^2}{c_1} f_1 + c_2 f_2, \\
                [f_2, f_2, f_2] = -\frac{c_2^3}{c_1^2} f_1 +
                    \frac{c_2^2}{c_1} f_2, \\
            \end{matrix*}
        \right. \label{samasus:0|2 - 1} \\[0.2cm]
        &\ \ \ [f_1, f_1, f_1] = c f_2, \label{samasus:0|2 - 2} \\[0.2cm]
        &\ \ \ [f_2, f_2, f_2] = c f_1, \label{samasus:0|2 - 3}
    \end{align}
    \renewcommand\arraystretch{1}
    kus $c, c_1, c_2 \in \C \setminus \{ 0 \}$ on vabad muutujad.
\end{lau}

Sooritades muutujavahetused $f_1 = f_2'$ ja $f_2 = f_1'$ annavad
eeskirjad \eqref{samasus:0|2 - 2} ja \eqref{samasus:0|2 - 3}
tegelikult samad kommutatsiooniseosed. Enamgi veel, sooritades muutujavahetuse
$f_2 = \frac{f_2'}{c}$, saame seose \eqref{samasus:0|2 - 2} viia kujule
\[ [f_1, f_1, f_1] = f_2'. \]
Vaadeldes muutujavahetusi $f_1 = \sqrt{c_1} f_1'$ ja
$f_2 = \frac{c_2}{\sqrt{c_1}} f_2'$ on vahetu kontrolli põhjal selge, et
samasused \eqref{samasus:0|2 - 1} on mistahes $c_1, c_2 \in \C \setminus \{0\}$
korral võimalik teisendada kujule
\begin{align*}
    \left\{
        \begin{matrix*}[l]
            [f_1', f_1', f_1'] = -f_1' + f_2', \\
            [f_1', f_1', f_2'] = -f_1' + f_2', \\
            [f_1', f_2', f_2'] = -f_1' + f_2', \\
            [f_2', f_2', f_2'] = -f_1' + f_2'. \\
        \end{matrix*}
    \right.
\end{align*}

Kokkuvõttes kehtib seega järgmine teoreem.
\begin{thm}
    Olgu $3$-Lie superalgebra $\g$ supervektorruumi dimensioon on $0|2$.
    Siis $\g$ on kas Abeli Lie superalgebra, või ta on isomorfne $3$-Lie
    superalgebraga $\h$, mille mittetriviaalsed kommutatsiooniseosed on kas
    \[ \left\{
        \begin{matrix*}[l]
            [f_1, f_1, f_1] = -f_1 + f_2, \\
            [f_1, f_1, f_2] = -f_1 + f_2, \\
            [f_1, f_2, f_2] = -f_1 + f_2, \\
            [f_2, f_2, f_2] = -f_1 + f_2, \\
        \end{matrix*}
    \right. \]
    või
    \[
        [f_1, f_1, f_1] = f_2,
    \]
    kus $f_1$ ja $f_2$ on $\h$ paaritud generaatorid.
\end{thm}

