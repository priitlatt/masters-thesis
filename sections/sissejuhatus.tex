%!TEX root = ../thesis.tex

\section*{Sissejuhatus}
\addcontentsline{toc}{section}{Sissejuhatus}

Matemaatika haru, mida me täna tunneme kui \emph{Lie teooriat} kerkis esile
geomeetria ja lineaaralgebra uurimisest. Lie teooria üheks keskseks mõisteks
on \emph{Lie algebra} -- vektorruum, mis on varustatud
mitteassotsiatiivse korrutamisega ehk nõndanimetatud \emph{Lie sulu} või
\emph{kommutaatoriga}.
Lie algebrad ja nende uurimine on tihedalt seotud teise Lie teooria keskse
mõistega, milleks on \emph{Lie rühm}. Viimased on
struktuurid, mis on korraga nii algebralised rühmad kui ka topoloogilised
muutkonnad, kusjuures rühma korrutamine ja selle pöördtehe on mõlemad
pidevad. Osutub, et igale Lie rühmale saab vastavusse seada Lie algebra, kuid
üldjuhul kahjuks vastupidine väide ei kehti. Samas on võimalik näidata
pisut nõrgem tulemus: suvalise lõplikumõõtmelise
reaalse või kompleksse Lie algebra jaoks leidub temale üheselt
vastav sidus Lie rühm \cite{kirillov2008introduction}. Just selle viimase,
nõndanimetatud \emph{Lie kolmanda teoreemi} tõttu on võimalik
Lie rühmasid vaadelda Lie algebrate kontekstis ja see teebki Lie algebrad
äärmiselt oluliseks ja efektiivseks tööriistaks.

Tähistagu kõikjal järgnevas $K$ nullkarakteristikaga korpust ning $\V$
vektorruumi üle korpuse $K$. Ruumi kokkuhoiu ja mugavuse mõttes
kasutame edaspidi vajaduse korral summade tähistamisel
Einsteini summeerimiskokkulepet. Teisi sõnu, kui meil on indeksid $i$ ja $j$,
mis omavad väärtusi $1, \dots, n$, kus $n \in \N$, siis jätame vahel
summeerimisel summa märgi kirjutamata, ning säilitame summeerimise
tähistamiseks vaid indeksid. Einsteini summeeruvuskokkulepet arvestades
kehtivad näiteks järgmised võrdused:
\begin{align*}
    x^{i} e_{i} &= \sum_{a=1}^{n} x^{i} e_{i} = 
        x^{1} e_{1} + x^{2} e_{2} + \dots + x^{n} e_{n}, \\
    \lambda{^i_j} x^{j} &= \sum_{j=1}^{n} 
        \lambda{^i_j} x^{j} = \lambda{^i_1} x^{1} + 
        \lambda{^i_2} x^{2} + 
        \dots +\lambda{^i_n} x^{n},\\
    \eta_{ij} u^{i} v^{j} &= \eta_{11} u^{1} v^{1} + 
        \eta_{12} u^{1} v^{2} + \dots + \eta_{1n} u^{1} v^{n} + 
        \eta_{21} u^{2} v^{1} + \dots + \eta_{nn} u^{n} v^{n},
\end{align*}
ja nii edasi.
