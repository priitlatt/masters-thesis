%!TEX root = ../../thesis.tex

\subsection{\texorpdfstring{$0|1$}{0|1}-dimensionaalsed
    \texorpdfstring{$3$}{3}-Lie superalgebrad}

Olgu $3$-Lie superalgebra $\g$ supervektorruumi dimensioon
$0|1$ ning olgu $\g$ generaatoriteks $\{ f \}$. Kommutaatori saame
siis generaatoritel välja arvutada kui
\[ [f, f, f] = \mu f, \]
kus $\mu \in \C$ on suvaline skalaar.

Rakendades nüüd eelmises punktis kirjeldatud algoritmi, saame ühelt poolt
\[
    [f, f, [f, f, f]] = \mu [f, f, f] = \mu^2 f,
\]
ning teisalt
\begin{align*}
    [f, f, [f, f, f]] =&\
        [[f, f, f], f, f] +
        (-1)^{|f|(|f| + |f|)} [f, [f, f, f], f] + \\
        &\ (-1)^{(|f| + |f|)^2} [f, f, [f, f, f]] = \\
    =&\ \mu^2 f +  \mu^2 f +  \mu^2 f = \\
    =&\ 3\mu^2 f.
\end{align*}

Kokkuvõttes saime seega täpselt ühe kitsenduse, mida $\g$ struktuurikonstandid
rahuldama peavad:
\[ \mu^2 = 3\mu^2. \]
Selle võrrandi ainsaks lahendiks on loomulikult $\mu = 0$.

\begin{thm}
    Kõik $3$-Lie superalgebraid, mille supervektorruumi dimensioon on $0|1$,
    on triviaalsed ehk Abeli Lie superalgebrad.
\end{thm}
