%!TEX root = ../thesis.tex

%%%%%%%%%%%%%%%%%%%%%%%%%
%% n-Lie superalgebra  %%
%%%%%%%%%%%%%%%%%%%%%%%%%

\section{\texorpdfstring{$n$}\ -Lie superalgebra}

Järgnevas toome sisse \emph{supermatemaatika} põhimõsted ning defineerime
nende abil \emph{Lie superalgebra}. Edasi ühendame Lie superalgebra
konstruktsiooni Filippovi $n$-Lie algebraga, ning konstrueerime
\emph{$n$-Lie superalgebra}.

\subsection{Lie superalgebra}

Eelmises peatükis nägime, kuidas Filippov leidis viisi Lie algebra
üldistamiseks kasutades binaarse operatsiooni asemel $n$-aarset
operatsiooni, kuid jättes aluseks oleva algebra samaks. Teine viis
Lie algebra mõistet üldistada on vaadelda algebra asemel \emph{superalgebrat}.
Tuletame selleks kõigepealt meelde mõned baasdefinitsioonid.

\begin{dfn}
    Vektorruumi $\V$ nimetatakse \emph{$\Z_2$-gradueeritud vektorruumiks}
    ehk \emph{supervektorruumiks}, kui ta on esitatav vektorruumide
    otsesummana $\V = \V_{\overline{0}} \oplus \V_{\overline{1}}$.
    Vektorruumi $\V_{\overline{0}}$ elemente nimetatakse seejuures
    \emph{paarisvektoriteks} ja $\V_{\overline{1}}$ elemente
    \emph{paarituteks} vektoriteks.
\end{dfn}

Kui supervektorruumi element $x \in \V$ korral
$x \in \V_{\overline{0}}$ või $x \in \V_{\overline{1}}$, siis ütleme, et
vektor $x$ on \emph{homogeenne}. Homogeensete vektorite korral on
otstarbekas vaadelda kujutust $|\cdot| \col \V \to \Z_2$, mis on defineeritud
võrdusega
\begin{align*}
    |x| = \begin{cases}
        \overline{0}, &\text{kui}\ x \in \V_{\overline{0}}, \\
        \overline{1}, &\text{kui}\ x \in \V_{\overline{1}}.
    \end{cases}
\end{align*}
Homogeense elemendi $x$ korral nimetame suurust $|x| \in \Z_2$ tema
paarsuseks. Märgime, et kasutades edaspidises kirjutist $|x|$ eeldame
vaikimisi, et element $x \in \V$ on homogeenne.

Supervektorruumi $\V = \V_{\overline{0}} \oplus \V_{\overline{1}}$
\emph{alamruumiks} nimetatakse supervektorruumi
$\W = \W_{\overline{0}} \oplus \W_{\overline{1}} \subseteq \V$, kui
vastavate otseliidetavate gradueeringud ühtivad, see tähendab
$\W_i \subset \V_i$, kus $i \in \Z_2$.

\begin{dfn}
    Olgu supervektorruumil $\A$ antud bilineaarne algebraline tehe
    $\phi \col \A \times \A \to \A$. Supervektorruumi $\A$ nimetatakse
    \emph{superalgebraks}, kui tehe $\phi$ rahuldab suvaliste homogeensete
    vektorite $x, y \in \A$ korral tingimust
    \begin{align}\label{def:superalgebra-parity}
        \left|\phi(x, y)\right| = |x| + |y|.
    \end{align}
\end{dfn}

Paneme tähele, et superalgebra definitsioonis nõutav kujutus
$\phi \col \A \times \A \to \A$ nõuab taustal, et supervektorruum
$\A$ oleks tegelikult algebra. Supervektorruumi ühikelement ja
assotsiatiivsus defineeritakse tavapärasel viisil.

Sellega on meil olemas piisvad vahendid, et defineerida
\emph{Lie superalgebra}.

\begin{dfn}
    Olgu $\g = \g_{\overline{0}} \oplus \g_{\overline{1}}$ superalgebra,
    millel on määratud bilineaarne kujutus
    $[\cdot, \cdot] \col \g \times \g \to \g$, mis on suvaliste
    homogeensete elementide $x, y, z \in \g$ korral
    \begin{enumerate}
        \item kaldsümmeetriline gradueeritud mõttes:
            \begin{align}\label{def:super-antisymm}
                [x, y] = -(-1)^{|x||y|}[y, x],
            \end{align}
        \item rahuldab gradueeritud Jacobi samasust
            \begin{align}\label{id:super-jacobi-id}
                [x, [y, z]] = [[x, y], z] + (-1)^{|x||y|}[y, [x, z]].
            \end{align}
    \end{enumerate}
\end{dfn}

Supermatemaatikas on standardseks võtteks elementide $x$ ja $y$ järjekorra
vahetamisel seda operatsiooni balanseerida korrutades tulemus läbi suurusega
$(-1)^{|x||y|}$, see tähendab võetakse arvesse elementide gradueeringuid.
Seega on definitsiooni tingimus \eqref{def:super-antisymm} põhjendatud.
Gradueeritud Jacobi samasuse \eqref{id:super-jacobi-id} selline kuju võib
esmapilgul tunduda mõnevõrra kohatu, kuid tuletades meelde, et me võime
klassikalise Jacobi samasuse kirja panna ka kujul \eqref{id:jacobi-modified},
siis on selge, et selline üldistus omab mõtet. Enamgi veel, kui $x$ ja $y$ on
paarisvektorid, siis saamegi täpselt samasuse \eqref{id:jacobi-modified}.

Meenutame, et näites \ref{naide:lie-algebra-konstrueerimine} andsime eeskirja,
kuidas suvalisest assotsiatiivsest algebrast on võimalik konstrueerida
Lie algebra. Osutub, et analoogiline situatsioon kehtib ka Lie superalgebrate
korral.

\begin{naide}
    Olgu $\A = \A_{\overline{0}} \oplus \A_{\overline{1}}$ assotsiatiivne
    superalgebra, kus elementide $x, y \in \A$ korrutise on tähistatud $xy$.
    Me saame anda supervektorruumile $\A$ Lie superalgebra struktuuri,
    kui defineerime homogeensete elementide jaoks sulu võrdusega
    \begin{align}\label{naide:assot-super-lie-alg}
        [x, y] = xy - (-1)^{|x||y|} yx, \quad x, y \in \A,
    \end{align}
    ning mittehomogeensete elementide jaoks rakendame $\phi$ bilineaarsust.

    Sel juhul
    \begin{align*}
        [x, y] &= xy - (-1)^{|x||y|} yx = \\
        &= -(-1)^{|x||y|} \left[
            -(-1)^{|x||y|}xy + yx
        \right] = \\
        &= -(-1)^{|x||y|} [y, x],
    \end{align*}
    ehk nõue \eqref{def:super-antisymm} on täidetud.

    Gradueeritud Jacobi samasuse jaoks rakendame kaks korda valemit
    \eqref{naide:assot-super-lie-alg} ning arvutame võrduses
    \eqref{id:super-jacobi-id} olevad liikmed, kusjuures assotsiatiivsust
    arvesse võttes jätame korrutistes sulud kirjutamata.
    \begin{flalign*}
        &-[x, [y, z]] = -xyz + (-1)^{|x||y| + |x||z|} yzx + 
        (-1)^{|y||z|} xzy - (-1)^{|x||y| + |x||z| + |y||z|} zyx, \\[0.25cm]
        %
        &[[x, y], z] = xyz - (-1)^{|x||z| + |y||z|} zxy -
        (-1)^{|x||y|} yxz + (-1)^{|x||y| + |x||z| + |y||z|} zyx, \\[0.25cm]
        %
        &(-1)^{|x||y|}[y, [x, z]] = (-1)^{|x||y|} yxz -
            (-1)^{|x||y| + |x||y| + |y||z|} xzy - \\
        &\phantom{(-1)^{|x||y|}[y, [x, z]] = }\ 
            (-1)^{|x||y| + |x||z|} yzx +
            (-1)^{|x||y| + |x||y| + |x||z| + |y||z|} zxy.
    \end{flalign*}
    Arvestades, et $(-1)^{2k + l} = (-1)^{l}$ mistahes
    $k, l \in \N \cup \{0\}$ korral, siis saame tulemusi liites kokku
    täpselt nulli, ehk sulg \eqref{naide:assot-super-lie-alg} rahuldab
    gradueeritud Jacobi samasust. Kokkuvõttes oleme saanud eeskirja, mille
    abil on suvaline assotsiatiivne superalgebra võimalik varustada
    Lie superalgebra struktuuriga.
\end{naide}


\subsection{\texorpdfstring{$n$}\ -Lie superalgebra}

Järgnevas eeldame, et meil on antud supervektorruum
ehk supervektorruum $\G = \G_{\overline{0}} \oplus \G_{\overline{1}}$
ning $n$-lineaarne kujutus $\phi \col \G^n \to \G$, mis rahuldab
tingimusi
\begin{itemize}
    \item $| \phi(x_1, \dots, x_n) | = \sum_{i=1}^n |x_i|$,
    \item $ \phi \left( x_1, \dots, x_i, x_{i+1}, \dots, x_n \right) =
            -(-1)^{ |x_i| |x_{i+1}| } \phi \left(
                x_1, \dots, x_{i+1}, x_i, \dots, x_n \right), $
\end{itemize}
kus $|x| \in \left\{ \overline{0}, \overline{1} \right\}$
tähistab elemendi $x$ paartust. Samuti eeldame, et $S \col \G \to \K$
on lineaarne kujutus, mis rahuldab
\begin{itemize}
    \item $S \left( \phi \left( x_1, \dots, x_n \right) \right) = 0$,
    \item $S(x) = 0$ iga $x \in \G_{\overline{1}}$.
\end{itemize}

Selge, et siin sisse toodud kujutused $\phi$ ja $S$ on eelnevas
kirjeldatu analoogid supervektorruumis. Seejuures kujutust $S \col \G \to \K$
nimetatakse \emph{superjäljeks}.

Kasutades kujutusi $\phi$ ja $S$ defineerime analoogiliselt
vektorruumide situatsioonile, kuid nüüd juba supervektorruumi iseärasusi
arvesse võttes, see tähendab paarsusi arvestades, uue kujutuse
$\phi_S \col \G^{n+1} \to G$ valemiga
\[
    \phi_S (x_1, \dots, x_{n+1}) =
    \sum_{i=1}^{n+1} (-1)^{i-1}(-1)^{|x_i| \sum_{j=1}^{i-1} |x_j| }
        S(x_i) \phi \left(
            x_1, \dots, \hat{x_i}, \dots, x_{n+1}
        \right).
\]

Saadud kujutuse tähtsamad omadused võtab kokku järgmine oluline lemma:

\begin{lemma}
    $(n+1)$-lineaarne kujutus $\phi_S \col \G^{n+1} \to \G$
    rahuldab tingimusi
    \begin{enumerate}
        \item $ | \phi_S \left(x_1, \dots, x_{n+1} \right) | =
               \sum_{i=1}^{n+1} |x_i| $,
        \item $ \phi_S \left(x_1, \dots, x_i, x_{i+1}, \dots, x_{n+1} \right) =
               -(-1)^{|x_i| |x_{i+1}|} \phi_S \left(
                    x_1, \dots, x_{i+1}, x_i, \dots, x_{n+1}
                \right) $,
        \item $S \left( \phi_S \left( x_1, \dots, x_{n+1} \right) \right)$.
    \end{enumerate}
\end{lemma}

Üldistame nüüd definitsiooni~\ref{def:lie-algebra} supervektorruumi jaoks
ning defineerime \emph{$n$-Lie superalgebra}.

\begin{dfn}[$n$-Lie superalgebra]
    Olgu $\G = \G_{\overline{0}} \oplus \G_{\overline{1}}$ 
    supervektorruum. Me ütleme, et $\G$ on
    \emph{$n$-Lie superalgebra}, kui $\G$ on varustatud
    gradueeritud $n$-Lie suluga $\nbrac{\cdot}{\cdot} \col \G^n \to \G$,
    mis rahuldab tingimusi
    \begin{enumerate}
        \item $\left| \nbrac{x_1}{x_n} \right| = \sum_{i=1}^n |x_i| $,
        \item $\left[ x_1, \dots, x_i, x_{i+1}, \dots, x_n \right] =
            -(-1)^{|x_i| |x_{i+1}|} \left[
                x_1, \dots, x_{i+1}, x_i, \dots, x_n
            \right]$,
        \item $\left[ y_1, \dots, y_{n-1}, \nbrac{x_1}{x_n} \right] = $ \\
            $ = \sum_{i=1}^n (-1)^{\tau_x (i-1) \tau_y (n-1)}
            \left[
                x_1, \dots, x_{i-1},
                \left[ y_1, \dots, y_{n-1}, x_i \right],
                x_{i+1}, \dots, x_n
            \right] $,
    \end{enumerate}
    kus $x = (x_1, \dots, x_n)$ ja $y = (y_1, \dots, y_{n-1})$ ning
    $\tau_x (k) = \sum_{j=1}^{k-1} |x_i|$.
\end{dfn}

Võttes arvesse $n$-Lie superalgebra definitsiooni saame sõnastada 
teoreemi \ref{thm:n+1_lie_alg} superanaloogi järgmiselt:

\begin{thm}
    Olgu $\G = \G_{\overline{0}} \oplus \G_{\overline{1}}$
    $n$-Lie superalgebra suluga $\nbrac{\cdot}{\cdot} \col \G^n \to \G$,
    ning $V$ lõplikumõõtmeline vektorruum ja olgu
    antud $\G$ esitus $\phi \col \G \to \gl V$. Defineerides
    $\nbrac{\cdot}{\cdot} \col \G^{n+1} \to \G$ valemiga
    \[
        \nbrac{x_1}{x_{n+1}} = \sum_{i=1}{n+1}
        (-1)^{i-1} (-1)^{|x_i| \tau_x (i-1)} S(\phi(x_i))
        \left[ x_1, \dots, \hat{x_i}, \dots, x_{n+1} \right],
    \]
    on supervektorruum $\G$, varustatuna suluga
    $\nbrac{\cdot}{\cdot} \col \G^{n+1} \to \G$ $(n+1)$-Lie
    superalgebra.
\end{thm}

\subsection{Indutseeritud \texorpdfstring{$n$}\ -Lie superalgebra}

See peatükk tugineb artiklile \cite{Abramov:2014} \dots
