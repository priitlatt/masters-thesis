%!TEX root = ../thesis.tex

%%%%%%%%%%%%%%%%%%%%%%%%%%%%%%%%%%%%%%%%%%%%%
%%  Klassikaline teooria ehk Lie algebrad  %%
%%%%%%%%%%%%%%%%%%%%%%%%%%%%%%%%%%%%%%%%%%%%%

\section{Lie algebra}

Matemaatika haru, mida me täna tunneme kui \emph{Lie teooriat} kerkis esile
geomeetria ja lineaaralgebra uurimisest. Lie teooria üheks keskseks mõisteks
on \emph{Lie algebra} - vektorruum, mis on varustatud
mitteassotsiatiivse korrutamisega ehk nõndanimetatud \emph{Lie suluga}.
Lie algebrad ja nende uurimine on tihedalt seotud teise Lie teooria keskse
mõistega, milleks on \emph{Lie rühm}. Viimased on
struktuurid, mis on korraga nii algebralised rühmad kui ka topoloogilised
muutkonnad, kusjuures rühma korrutamine ja selle pöördtehe on mõlemad
pidevad. Osutub, et igale Lie rühmale saab vastavusse seada Lie algebra ja
kehtib ka vastupidine, pisut nõrgem tulemus. Nimelt suvalise lõplikumõõtmelise
reaalsele või komplekssele Lie algebra jaoks leidub temale üheselt
vastav sidus Lie rühm.\cite{kirillov2008introduction} Just selle viimase,
\emph{Lie kolmanda teoreemi} tõttu on võimalik Lie rühmasid vaadelda
Lie algebrate kontekstis ja see teebki Lie algebrad äärmiselt oluliseks.

Tähistagu kõikjal järgnevas $K$ nullkarakteristikaga korpust ning $V$
vektorruumi üle korpuse $K$. Ruumi kokkuhoiu ja mugavuse mõttes
kasutame edaspidi vajaduse korral summade tähistamisel
Einsteini summeerimiskokkulepet. See tähendab, kui meil on indeksid $i$ ja $j$,
mis omavad väärtusi $1, \dots, n$, kus $n \in N$, siis jätame vahel
summeerimisel summamärgi kirjutamata ning säilitame summeerimise tähistamiseks
vaid indeksid. Einsteini summeeruvuskokkulepet arvestades kehtivad näiteks
järgmised võrdused:
\begin{align*}
    x^{i} e_{i} &= \sum_{a=1}^{n} x^{i} e_{i} = 
        x^{1} e_{1} + x^{2} e_{2} + \dots + x^{n} e_{n}, \\
    \lambda{^i_j} x^{j} &= \sum_{j=1}^{n} 
        \lambda{^i_j} x^{j} = \lambda{^i_1} x^{1} + 
        \lambda{^i_2} x^{2} + 
        \dots +\lambda{^i_n} x^{n},\\
    \eta_{ij} u^{i} v^{j} &= \eta_{11} u^{1} v^{1} + 
        \eta_{12} u^{1} v^{2} + \dots + \eta_{1n} u^{1} v^{n} + 
        \eta_{21} u^{2} v^{1} + \dots + \eta_{nn} u^{n} v^{n},
\end{align*}
ja nii edasi.

Järgnevas teeme raamatu \cite{johan1989survey} põhjal põgusa sissejuhatuse
klassikalisse Lie teooriasse. 

\subsection{Maatriksrühmade näiteid}

Meenutame, et \emph{lineaarkujutus}
$\phi \col V_1 \to V_2$ vektorruumist $V_1$ vektorruumi $V_2$ säilitab
vektorite liitmise ja skalaariga korrutamise, see tähendab
\[ \phi(x + y) = \phi(x) + \phi(y), \]
ning
\[ \phi(\lambda x) = \lambda \phi(x), \]
kus $x, y \in V_1$ ja $\lambda$ on skalaar. Kui vektorruumid $V_1$ ja $V_2$
langevad kokku, siis ütleme me kujutuse $\phi$ kohta ka \emph{lineaarteisendus}.

Algebrast on teada, et lineaarteisendusel eksisteerib pöördteisendus siis ja
ainult siis, kui ta on nii üks-ühene kui ka pealeteisendus. Kõigi vektorruumi $V$
pööratavate lineaarteisenduste rühma nimetatakse vektorruumi $V$
\emph{lineaarteisenduste rühmaks}\footnote{Inglise keeles
\emph{general linear group}.} ja tähistatakse $\GL(V)$. Selge, et selle
rühma korrutamiseks on tavaline lineaarteisenduste kompositsioon.

Et lõplikumõõtmelise vektorruumi lineaarteisendus on pööratav parajasti siis
kui tema determinant on nullist erinev, siis rühma $\GL(V)$ kuuluvad
need ja ainult need lineaarteisendused, mille determinant pole null.
Kui vaatleme vaid lineaarteisendusi, mille determinant on üks, saame olulise
alamrühma $\SL(V)$, mida nimetatakse vektorruumi $V$
\emph{spetsiaalsete lineaarteisenduste rühmaks}.

Et igal vektorruumil leidub baas, siis võime vektorruumi $V$ jaoks
fikseerida mingi baasi. Sel juhul
saame kõik lineaarteisendused esitada maatriksitena ning nõnda võime
edaspidi lineaarteisenduste rühmade asemel rääkida \emph{maatriksrühmadest}.
Kui $\{e_1, e_2, \dots, e_n\}$ moodustab vektorruumi $V$ baasi ning
$\phi \col V \to V$ on mingi lineaarteisendus, siis talle vastav maatriks
selle baasi suhtes on $(a^i_j)$, mis on määratud valemiga
\[ \phi(e_j) = \sum_{i=1}^{n} a^i_j e_i. \]

Selge, et vaadeldes rühmi $\GL(V)$ ja $\SL(V)$ maatriksrühmadena
on rühma tehteks juba tavaline maatriksite korrutamine. Ilmselt saab
nimetatud maatriksrühmad defineerida suvalise korpuste jaoks, ja nii ka
reaal- ning kompleksarvude korral. Sellest lähtuvalt kasutatakse sageli
nullist erineva determinandiga $n \times n$ maatriksrühmade tähistuseks
$\GL(n, \R)$ või $\GL(n, \C)$, ning neid rühmi nimetame vastavalt
\emph{reaalsete lineaarteisenduste rühmaks} ja \emph{komplekssete
lineaarteisenduste rühmaks}. Analoogiliselt on kasutusel tähistused
$\SL(n, \R)$ ja $\SL(n, \C)$.

Rühmal $\GL(n, \C)$ on palju tuntud alamrühmi. Klassikaliseks näiteks on
$n \times n$ \emph{ortogonaalsete maatriksite rühm} $\Ort(n, \C)$, kuhu kuuluvad
ortogonaalsed maatriksid, see tähendab sellised maatriksid $A$, mille korral
$A^T = A^{-1}$. Teise näitena võib tuua unitaarsete maatriksite rühma $\U(n)$,
mille elementideks on anti-Hermite'i maatriksid $A$, mis rahuldavad tingimust
$A^\dag = \overline{A}^T = -A$. Edasi on lihtne konstrueerida saadud alamrühmade
\emph{spetsiaalsed} variandid. \emph{Spetsiaalsete komplekssete ortogonaalmaatriksite rühm} on
\[ \SO(n, \C) = \Ort(n, \C) \cap \SL(n, \C), \]
ja \emph{spetsiaalsete unitaarsete maatriksite rühmaks} on
\[ \SU(n) = \U(n) \cap \SL(n, \C). \]

\subsection{Bilineaarvorm}

\begin{dfn}
    Olgu $V$ vektorruum üle korpuse $K$. Kujutust $\psi \col V^2 \to K$
    nimetatakse \emph{bilineaarvormiks}, kui iga $x, y, z \in V$
    ja skalaaride $\lambda, \mu \in K$ korral
    \begin{enumerate}[label=\roman*.]
        \item $\psi(\lambda x + \mu y, z) = \lambda \psi(x, z) + \mu \psi(y, z)$,
        \item $\psi(x, \lambda y + \mu z) = \lambda \psi(x, y) + \mu \psi(x, z)$.
    \end{enumerate}
\end{dfn}

Kui vetorruumis $V$ on antud baas $\{e_1, e_2, \dots, e_n\}$, siis saab
bilineaarvormi $\psi \col V^2 \to K$ esitada talle vastava maatriksi
$B = (b_{ij})$ abil, kus $b_{ij} = \psi(e_i, e_j)$. Tõepoolest, kui
meil on antud vektorid $x = \sum_i \lambda^i e_i$ ja $y = \sum_j \mu^j e_j$,
siis kasutades $\psi$ lineaarsust mõlema muutuja järgi võime arvutada
\[ \psi(x, y) = \sum_{i, j} b_{ij} \lambda^i \mu^j. \]

Me ütleme, et bilineaarvorm $\psi \col V^2 \to K$ on \emph{sümmeetriline} kui
iga $x, y \in V$ korral $\psi(x, y) = \psi(y, x)$. Selge, et sümmeetrilise
bilineaarvormi maatriksi $B$ korral kehtib võrdus $B = B^T$. Vormi $\psi$
nimetatakse \emph{kaldsümmeetriliseks} kui iga $x, y \in V$ korral kehtib
võrdus $\psi(x, y) = - \psi(y, x)$. Lihtne on veenduda, et kaldsümmeetrilise
bilineaarvormi korral rahuldab talle vastav maatriks $B$ seost
$B^T = -B$.

\begin{naide}
    Skalaarkorrutis on bilineaarvorm.
\end{naide}

\begin{dfn}[Jälg]
    Olgu $V$ vektorruum kus on fikseeritud mingi baas, olgu $\phi$
    vektorruumi $V$ lineaarteisendus ning olgu $A = (a^i_j)$ on
    lineaarteisenduse $\phi$ maatriks fikseeritud baasi suhtes.
    Lineaarteisenduse $\phi$ \emph{jäljeks} nimetatakse kujutust
    $\Tr_V \col \GL(V) \to K$, kus
    \[ \Tr_V(A) = \sum_i a^i_j. \]
\end{dfn}

\begin{naide}
    On hästi teada, et vektorruumi $V$ kõigi lineaarteisenduste hulk $\Lin V$
    on ise ka vektorruum, kusjuures kui $\dim(V) = n$, siis
    $\dim \left(\Lin V\right) = n^2$. Kasutades jälge $\Tr_V$ saame
    defineerida bilineaarvormi vektorruumil
    $\psi \col \Lin V \times \Lin V \to K$ järgmiselt:
    \[ \psi(A, B) = \Tr_V(AB), \]
    kus $A$ ja $B$ on maatriksid, mis vastavad vektorruumi $\Lin V$
    teisendustele mingi baasi suhtes.
\end{naide}

Kasutades bilineaarvormi sümmeetrilisuse või kaldsümmeetrilisuse mõistet
saame sisse tuua \emph{ortogonaalsuse}: me ütleme, et vektorid $x$ ja $y$
on bilineaarvormi $\psi$ suhtes ortogonaalsed, kui $\psi(x, y) = 0$. Selge, et
ortogonaalsuse tingimus ise on sümmeetriline, see tähendab kui $x$ on
ortogonaalne vektoriga $y$, siis kehtib ka vastupidine, $y$ on ortogonaalne
vektoriga $x$. Kui vektor $x \neq 0$ on iseenesega ortogonaalne, see tähendab
$\psi(x, x) = 0$, siis nimetatakse vektorit $x$ \emph{isotroopseks}. Selge, et
Eukleidilises geomeetrias selliseid vektoreid ei leidu, kuid üldisemas
käsitluses kohtab neid juba üsna madalal tasemel, näiteks
\emph{Minkowski ruumis}.

Edasises vaatleme ortogonaaseid ja sümplektilisi rühmi ning selleks
nõuame, et vaatluse all olevad bilineaarvormid oleksid mittesingulaarsed
ehk regulaarsed, see tähendab kui $\psi(x, y) = 0$ iga $y \in V$ korral, siis
järelikult $x = 0$. Osutub, et bilineaarvorm $\psi$ on regulaarne parajasti
siis, kui temale vastav maatiks $B = (b^i_j)$ on pööratav, mis tähendab,
et $\det B \neq 0$.

\begin{dfn}
    Me ütleme, et lineaarne operaator $\psi$ on \emph{ortogonaalne}
    regulaarse sümmeetrilise bilineaarvormi $\psi$ suhtes, kui
    \[ \psi(\phi(x), \phi(y)) = \psi(x, y) \]
    kõikide $x$ ja $y$ korral vektorruumist $V$.
\end{dfn}

Kui $x$ on ortogonaalse lineaarse operaatori $\phi$ tuumast, siis
kehtib $\phi(x) = 0$. Viimane aga tähendab, et iga $y \in V$ korral
$\psi(x, y) = \psi(\phi(x), \phi(y)) = \psi(0, \phi(y)) = 0$. Kokkuvõttes,
et $\psi$ on regulaarne, siis järelikult $x = 0$ ja $\phi$ on üks-ühene.
Kui nüüd veel $V$ on lõplikumõõtmeline, siis peab $\phi$ olema pööratav.
Seda arutelu silmas pidades võime öelda, et ortogonaalsed lineaarsed
operaatorid moodustavad rühma, mida me nimetame \emph{ortogonaalsed rühmaks}
bilineaarvormi $\psi$ suhtes.

Et saada eelnevalt defineeritud ortogonaalset maatriksrühma peame appi võtma
vektorruumi baasi...
