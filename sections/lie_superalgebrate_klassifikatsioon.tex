%!TEX root = ../thesis.tex

%%%%%%%%%%%%%%%%%%%%%%%
%% Klassifikatsioon  %%
%%%%%%%%%%%%%%%%%%%%%%%

\section[Madaladimensionaalsete \texorpdfstring{$3$}{3}-Lie superalgebrate
klassifikatsioon]{Madaladimensionaalsete \texorpdfstring{$3$}{3}-Lie
superalgebrate \\ klassifikatsioon}\label{section:klassifikatsioon}

\subsection{Meetodi kirjeldus}

Järgnevas peatükis uurime kui palju on maksimaalselt erinevaid $3$-Lie
superalgebraid üle supervektorruumi, dimensiooniga $m|n$, kus $m + n \leq 4$.
Klassifikatsiooni koostamiseks kasutame Lie superalgebra struktuurikonstante
ja teeme seda järgmisel viisil.

Oletame, et meil on 3-Lie superalgebra $(\g, [\cdot, \cdot, \cdot])$,
kusjuures supervektorruum $\g$ on lõpik ning tema dimensioon on $m|n$. Olgu
supervektorruumil $\g$ fikseeritud mingi baas $\B$. Arvestades dimensiooni
$m|n$, saame baasi kirjutada kujul
\[
    \B = \left\{ e_1, e_2, \dots, e_m, f_1, f_2, \dots, f_n \right\} =
         \left\{ z_1, z_2, \dots, z_{m+n} \right\},
\]
kus $e_\alpha$, $1 \leq \alpha \leq m$, on paaris baasivektorid ja
$f_i$, $1 \leq i \leq n$, on paaritud baasivektorid, ning $z_A$,
$1 \leq A \leq m+n$, võib olla kas paaris- või paarituvektor, kusjuures
$z_A = e_A$, kui $1 \leq A \leq m$, ning $z_A = f_{A-m}$, kui
$m < A \leq m+n$.

Edasi, kasutades teadmist, et $|[z_1, z_2, z_3]| = |z_1| + |z_2| + |z_3|$,
avaldame kommutaatori $[\cdot, \cdot, \cdot]$ väärtused baasivektoritel
struktuurikonstantide $K_{ABC}^D$ abil:
\begin{align*}
    &[e_\alpha, e_\beta, e_\gamma] =
        K_{\alpha \beta \gamma}^\lambda e_\lambda, \\
    &[e_\alpha, e_\beta, f_i] = K_{\alpha \beta i}^j f_j, \\
    &[e_\alpha, f_i, f_j] = K_{\alpha i j}^\beta e_\beta, \\
    &[f_i, f_j, f_k] = K_{i j k}^l f_l,
\end{align*}
kus $\alpha \leq \beta \leq \gamma$ ja $i \leq j \leq  k$.
Seejuures paneme tähele, et rohkematel kui kirjutatud baasivektorite
kombinatsioonidel ei ole mõtet kommutaatori väärtusi arvutada kuna nad ei
sisalda uut informatsiooni, sest kaldsümmeetrilisuse abil on võimalik
sulg $[f_j, e_\alpha, f_i]$, $i < j$, viia alati kujule $[e_\alpha, f_i, f_j]$,
ning jõuda juba saadud tulemuseni.

Otsime nüüd välja millised sulud on tegelikult võrdsed nullvektoriga. Selleks
permuteerime kaldsümmeetrilisust kasutades kommutaatori argumente, ning kui
jõuame samasuguse tulemuseni kui oli esialgne sulg, kuid erineva märgiga,
peab kogu sulg võrduma nulliga, sest vektor on võrdne oma vastandvektoriga
siis ja ainult siis, kui tegu on nullvektoriga. Vaatame näiteks sulgu
$[e_1, e_1, f_i]$. Siis ilmselt
\[
    [e_1, e_1, f_i] = -(-1)^{|e_1||e_1|} [e_1, e_1, f_i] =
    -(-1)^0 [e_1, e_1, f_i] = - [e_1, e_1, f_i],
\]
ehk $[e_1, e_1, f_i] = 0$.
Sellega oleme ära kasutanud $n$-Lie superalgebra definitsioonis nõutud
gradueeringute kooskõla ehk tingimuse \eqref{def:n-lie-superalg-brac-grading}
ja samuti oleme juba tarvitusele võtnud kaldsümmeetrilisuse nõude ehk
tingimuse \eqref{def:n-lie-superalg-brac-antikomm}. Järele jääb veel kasutada
gradueeritud Filippovi samasus \eqref{id:graded-filippov}.

Valime kõik nullist erinevad sulud
\[ [z_A, z_B, z_C] = K_{ABC}^D \ne 0, \]
$1 \leq A \leq B \leq C \leq m+n$, ja arvutame kommutaatori
\[ \left[ z_E, z_F, [z_A, z_B, z_C] \right] \]
välja kahel erineval viisil. Esmalt saame ära kasutada juba teadaolevat,
ning võime kirjutada
\[
    \left[ z_E, z_F, [z_A, z_B, z_C] \right] = K_{ABC}^D [z_E, z_F, z_D].
\]
Seejärel teisendame sulu $[z_E, z_F, z_D]$ kujule
$(-1)^{\circlearrowright_{DEF}} [z_{D'}, z_{E'}, z_{F'}]$, kus
\[ \{D, E, F\} = \{D', E', F'\}, \]
kuid $D' \leq E' \leq F'$, ja  $(-1)^{\circlearrowright_{DEF}}$ tähistab
kaldsümmeetrilisust arvestades permutatsiooni tulemusel saadavat märki.
Samas ka $[z_{D'}, z_{E'}, z_{F'}]$ on avaldatud struktuurikonstandide
abil baasivektorite lineaarkombinatsioonina, ehk võime kirjutada
\[
    [z_{D'}, z_{E'}, z_{F'}] =
    K_{D' E' F'}^H z_H.
\]
Seega ühelt poolt
\[
    \left[ z_E, z_F, [z_A, z_B, z_C] \right] =
    (-1)^{\circlearrowright_{DEF}} K_{ABC}^D K_{D' E' F'}^H z_H.
\]

Teiselt poolt on meil kommutaatori $\left[ z_E, z_F, [z_A, z_B, z_C] \right]$
arvutamiseks võimalik kasutada Filippovi samasust
\begin{align*}
    &\left[ z_E, z_F, [z_A, z_B, z_C] \right] = \\
    = &\left[[z_E, z_F, z_A], z_B, z_C\right] + \\
    (-1)^{|z_A|(|z_E|+|z_F|)}
        &\left[z_A, [z_E, z_F, z_B], z_C \right] + \\
    (-1)^{(|z_A|+|z_B|)(|z_E|+|z_F|)}
        &\left[z_A, z_B, [z_E, z_F, z_C] \right]
\end{align*}
Igas saadud liidetavas saame nüüd rakendada eelnevalt kirjeldatud algoritmi.
Toome selleks sisse tähistused
\[
    z_{AEF} := |z_A|(|z_E|+|z_F|)
    \quad \text{ja} \quad
    z_{ABEF} := (|z_A|+|z_B|)(|z_E|+|z_F|),
\]
ning järjestame kõigepealt sisemised kommutaatorid nii, et argumentvektorid
oleks kasvavas järjekorras ja asendame siis selle kommutaatori temale vastava
baasivektorite lineaarkombinatsiooniga. Edasi teeme täpselt sama allesjäänud
kommutaatoritega. Sedasi toimides saame
\begin{align*}
    &\left[ z_E, z_F, [z_A, z_B, z_C] \right] = \\
    %
    = (-1)^{\circlearrowright_{AEF}}
        &\left[ [z_{A'}, z_{E'}, z_{F'}], z_B, z_C \right] + \\
    (-1)^{z_{AEF} + \circlearrowright_{BEF}}
        &\left[ z_A, [z_{B'}, z_{E'}, z_{F'}], z_C \right] + \\
    (-1)^{z_{ABEF} + \circlearrowright_{CEF}}
        &\left[ z_A, z_B, [z_{C'}, z_{E'}, z_{F'}] \right] = \\
    %
    = (-1)^{\circlearrowright_{AEF}} K_{A' E' F'}^G
        &\left[ z_G, z_B, z_C \right] + \\
    (-1)^{z_{AEF} + \circlearrowright_{BEF}} K_{B' E' F'}^G
        &\left[ z_A, z_G, z_C \right] + \\
    (-1)^{z_{ABEF} + \circlearrowright_{CEF}} K_{C' E' F'}^G
        &\left[ z_A, z_B, z_G \right] = \\
    %
    = (-1)^{\circlearrowright_{AEF} + \circlearrowright_{B'C'G'}}
        K_{A' E' F'}^G
        &\left[ z_{B'}, z_{C'}, z_{G'} \right] + \\
    (-1)^{z_{AEF} + \circlearrowright_{BEF} + \circlearrowright_{A'C'G'}}
        K_{B' E' F'}^G
        &\left[ z_{A'}, z_{C'}, z_{G'} \right] + \\
    (-1)^{z_{ABEF} + \circlearrowright_{CEF} + \circlearrowright_{A'B'G'}}
        K_{C' E' F'}^G
        &\left[ z_{A'}, z_{B'}, z_{G'} \right] = \\
    %
    = (-1)^{\circlearrowright_{AEF} + \circlearrowright_{B'C'G'}}
        K_{A' E' F'}^G K_{B' C' G'}^H
        &\,z_H + \\
    (-1)^{z_{AEF} + \circlearrowright_{BEF} + \circlearrowright_{A'C'G'}}
        K_{B' E' F'}^G K_{A' C' G'}^H
        &\,z_H + \\
    (-1)^{z_{ABEF} + \circlearrowright_{CEF} + \circlearrowright_{A'B'G'}}
        K_{C' E' F'}^G K_{A' B' G'}^H
        &\,z_H.
\end{align*}

Teisisõnu, me saame kommutaatori $\left[ z_E, z_F, [z_A, z_B, z_C] \right]$
arvutada ka kujul
\begin{align*}
    \left[ z_E, z_F, [z_A, z_B, z_C] \right] = 
    &(-1)^{\circlearrowright_{AEF} + \circlearrowright_{B'C'G'}}
        K_{A' E' F'}^G K_{B' C' G'}^H
        \,z_H + \\
    &(-1)^{z_{AEF} + \circlearrowright_{BEF} + \circlearrowright_{A'C'G'}}
        K_{B' E' F'}^G K_{A' C' G'}^H
        \,z_H + \\
    &(-1)^{z_{ABEF} + \circlearrowright_{CEF} + \circlearrowright_{A'B'G'}}
        K_{C' E' F'}^G K_{A' B' G'}^H
        \,z_H,
\end{align*}
ehk meil on tekkinud võrrandid
\begin{align*}
    (-1)^{\circlearrowright_{DEF}} K_{ABC}^D K_{D' E' F'}^H z_H =\ 
    &(-1)^{\circlearrowright_{AEF} + \circlearrowright_{B'C'G'}}
        K_{A' E' F'}^G K_{B' C' G'}^H
        \,z_H + \\
    &(-1)^{z_{AEF} + \circlearrowright_{BEF} + \circlearrowright_{A'C'G'}}
        K_{B' E' F'}^G K_{A' C' G'}^H
        \,z_H + \\
    &(-1)^{z_{ABEF} + \circlearrowright_{CEF} + \circlearrowright_{A'B'G'}}
        K_{C' E' F'}^G K_{A' B' G'}^H
        \,z_H,
\end{align*}
kus otsitavateks on struktuurikonstandid $K_{ABC}^D$, ning baasivektorid
$z_H$ on teada. Siit saame iga $H \in \{ 1, 2, m+n \}$ jaoks veel omakorda
võrrandi
\begin{align*}
    (-1)^{\circlearrowright_{DEF}} K_{ABC}^D K_{D' E' F'}^H =\ 
    &(-1)^{\circlearrowright_{AEF} + \circlearrowright_{B'C'G'}}
        K_{A' E' F'}^G K_{B' C' G'}^H + \\
    &(-1)^{z_{AEF} + \circlearrowright_{BEF} + \circlearrowright_{A'C'G'}}
        K_{B' E' F'}^G K_{A' C' G'}^H + \\
    &(-1)^{z_{ABEF} + \circlearrowright_{CEF} + \circlearrowright_{A'B'G'}}
        K_{C' E' F'}^G K_{A' B' G'}^H,
\end{align*}
sest igas võrrandis on vasakul ja paremal pool sama vektor, ning
vektor avaldub baasivektorite lineaarkombinatsioonina üheselt.
Kokkuvõttes on meil tekkinud ruutvõrrandisüsteem, mille lahenditeks on
$m|n$ dimensiooniga Lie superalgebra $\g$ võimalikud struktuurikonstantide
komplektid.

Eelnevalt kirjeldatud algoritm on käesoleva magistriöö tarbeks
implementeeritud programmeerimiskeeles Python, ning see on kättesaadav
aadressilt \url{https://github.com/priitlatt/3-lie-superalgebras}.
Selle programmi abil tekkinud võrrandisüsteemide lahendamiseks on
kasutatud sümbolarvutustarkvara
Mathematica 10\footnote{\url{http://www.wolfram.com/mathematica/}}
funktsiooni \verb+Solve+\footnote{\url{https://reference.wolfram.com/language/ref/Solve.html}}, kus määramispiirkonnaks on antud
kompleksarvud.

Samas on oluline märkida, et sel viisil
saadud klassifikatsioon sõltub supervektorruumi baasi valikust ega ole seega
isomorfismi täpsusega. Invariantsete lahendite kõrvaldamist uurime juba iga
konkreetse situatsiooni juures juba eraldi.

Lisaks rõhutame, et me ei hakka siin uurima Lie superalgebraid,
mille supervektorruumi dimensioon on $m|0$, kus $m \in \N$, sest
analoogiliselt lausele \ref{lause:null-otseliidetav} on selline Lie
superalgebra tegelikult tavaline Lie algebra. Samas kui dimensioon on
kujul $0|m$, siis erinevalt lausest \ref{lause:null-otseliidetav}, ei ole
meil automaatselt tegu Abeli Lie superalgebraga, sest ternaarne kommutaator
annab paaritutel vektoritel tulemuseks uuesti paaritu vektori.
