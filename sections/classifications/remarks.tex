%!TEX root = ../../thesis.tex

\subsection{Täiendavaid märkuseid}

Eelnevas leidsime ülemise tõkke erinevate $3$-Lie superalgebrate arvule,
kui algebrale aluseks oleva supervektorruumi dimensioon on $m|n$, kus
$m + n \leq 3$. Saadud tulemused võtab kokku järgnev tabel.

\begin{center}
    \begin{tabular}{|c|c|} \hline
        \textbf{
            \begin{tabular}{c}
                Supervektorruumi \\
                dimensioon
            \end{tabular}
        } & \textbf{
            \begin{tabular}{c}
                Erinevaid \\
                algebraid
            \end{tabular}
        } \\ \hline
        $0|1$ & $1$ \\ \hline
        $0|2$ & $3$ \\ \hline
        $1|1$ & $1$ \\ \hline
        $1|2$ & $3$ \\ \hline
        $2|1$ & $4$ \\ \hline
    \end{tabular}
\end{center}

Kahjuks osutub, et selline meetod ei skaleeru ja seega ei sobi suurema
generaatorite arvuga $3$-Lie superalgebrate klassifitseerimiseks. Näiteks
kui supervektorruumi dimensioon on $0|3$ tekib meil $180$ võrrandiga süsteem
$30$ tundmatu suhtes\footnote{Võrrandid on leitavad veebilehelt
\url{http://priitlatt.github.io/masters-thesis/kitsendused/0_3.txt}}.
Selle süsteemi lahendeid Mathematica 10 tavalise
arvuti peal mõistliku ajaga enam leida ei suuda, rääkimata vabavaralistest
alternatiividest nagu SymPy\footnote{\url{http://www.sympy.org}}. Põhiliseks
takistuseks on siin loomulikult teist järku võrrandite süsteemi täpsete
lahendite leidmise suur arvutuslik keerukus.

Teisalt näiteks dimensioonide $3|1$ ja $2|2$ korral tekib kitsendusi vastavalt
$112$ ja $192$ ning Mathematica 10 suudab mõlemal juhul lahendid üle $\C$ leida,
kuid lahendikomplekte on liiga palju, et nende põhjal olulisi järeldusi teha.
Näiteks dimensiooni $3|1$ korral tekib $55$ lahendikomplekti, mille põhjal
saame sama palju võimalikke kommutatsiooniseoseid, nagu on näha lisas
\ref{lisa:3|1 kommutatsiooniseosed}. Samas on selgelt näha, et mitmed
kommutatsiooniseosed on seal tegelikult üksteisega isomorfsed, kui kasutame
vaid generaatorite ümbernimetamist. Sellisteks üksteisega samaväärseteks
on näiteks seosted $33$, $34$ ja $47$ või $41$, $46$, $50$. Nõnda on võimalik
vähese vaevaga vähendada erinevate kommutatsiooniseoste arvu rohkem kui kümne
võrra, kuid rohkemate isomorfismide leidmine nõuaks oluliselt põhjalikumat
analüüsi. Analoogiline on situatsioon ka $3$-Lie superalgebrate korral, mille
supervektorruumi dimensioon on $2|2$.
