%!TEX root = ../thesis.tex

%%%%%%%%%%%%%%%%%%%%%%%%%%%%%%%%%%%%%%%%%%%%%
%%  Klassikaline teooria ehk Lie algebrad  %%
%%%%%%%%%%%%%%%%%%%%%%%%%%%%%%%%%%%%%%%%%%%%%

\section{Lie algebra}

Matemaatika haru, mida me täna tunneme kui \emph{Lie teooriat} kerkis esile
geomeetria ja lineaaralgebra uurimisest. Lie teooria üheks keskseks mõisteks
on \emph{Lie algebra} - vektorruum, mis on varustatud
mitteassotsiatiivse korrutamisega ehk nõndanimetatud \emph{Lie suluga}.
Lie algebrad ja nende uurimine on tihedalt seotud teise Lie teooria keskse
mõistega, milleks on \emph{Lie rühm}. Viimased on
struktuurid, mis on korraga nii algebralised rühmad kui ka topoloogilised
muutkonnad, kusjuures rühma korrutamine ja selle pöördtehe on mõlemad
pidevad. Osutub, et igale Lie rühmale saab vastavusse seada Lie algebra, kuid
üldjuhul kahjuks vastupidine väide ei kehti. Samas on võimalik näidata
pisut nõrgem tulemus: suvalise lõplikumõõtmelise
reaalsele või komplekssele Lie algebra jaoks leidub temale üheselt
vastav sidus Lie rühm.\cite{kirillov2008introduction} Just selle viimase,
nõndanimetatud \emph{Lie kolmanda teoreemi} tõttu on võimalik
Lie rühmasid vaadelda Lie algebrate kontekstis ja see teebki Lie algebrad
äärmiselt oluliseks.

Tähistagu kõikjal järgnevas $K$ nullkarakteristikaga korpust ning $\V$
vektorruumi üle korpuse $K$. Ruumi kokkuhoiu ja mugavuse mõttes
kasutame edaspidi vajaduse korral summade tähistamisel
Einsteini summeerimiskokkulepet. Teisi sõnu, kui meil on indeksid $i$ ja $j$,
mis omavad väärtusi $1, \dots, n$, kus $n \in \N$, siis jätame vahel
summeerimisel summamärgi kirjutamata ning säilitame summeerimise tähistamiseks
vaid indeksid. Einsteini summeeruvuskokkulepet arvestades kehtivad näiteks
järgmised võrdused:
\begin{align*}
    x^{i} e_{i} &= \sum_{a=1}^{n} x^{i} e_{i} = 
        x^{1} e_{1} + x^{2} e_{2} + \dots + x^{n} e_{n}, \\
    \lambda{^i_j} x^{j} &= \sum_{j=1}^{n} 
        \lambda{^i_j} x^{j} = \lambda{^i_1} x^{1} + 
        \lambda{^i_2} x^{2} + 
        \dots +\lambda{^i_n} x^{n},\\
    \eta_{ij} u^{i} v^{j} &= \eta_{11} u^{1} v^{1} + 
        \eta_{12} u^{1} v^{2} + \dots + \eta_{1n} u^{1} v^{n} + 
        \eta_{21} u^{2} v^{1} + \dots + \eta_{nn} u^{n} v^{n},
\end{align*}
ja nii edasi.

Järgnevas anname minimaalse ülevaate klassikalisest Lie algebrate teooriast,
mida on tarvis edasiste peatükkide mõistmiseks.

\subsection{Maatriksrühmad ja bilineaarvorm}\label{subsec:mat-ryhmad-ja-bilinvorm}

Meenutame, et \emph{lineaarkujutus}
$\phi \col \V_1 \to \V_2$ vektorruumist $\V_1$ vektorruumi $\V_2$ säilitab
vektorite liitmise ja skalaariga korrutamise, see tähendab
\begin{align*}
    \phi(x + y) = \phi(x) + \phi(y),
\end{align*}
ning
\begin{align*}
    \phi(\lambda x) = \lambda \phi(x),
\end{align*}
kus $x, y \in \V_1$ ja $\lambda$ on skalaar. Kui vektorruumid $\V_1$ ja $\V_2$
langevad kokku, siis ütleme me kujutuse $\phi$ kohta \emph{lineaarteisendus}.

Algebrast on teada, et lineaarteisendusel eksisteerib pöördteisendus siis ja
ainult siis, kui ta on nii üks-ühene kui ka pealeteisendus. Kõigi vektorruumi $\V$
pööratavate lineaarteisenduste rühma nimetatakse vektorruumi $\V$
\emph{pööratavate lineaarteisenduste rühmaks}\footnote{Inglise keeles
\emph{general linear group}.} ja tähistatakse $\GL(\V)$. Selge, et selle
rühma korrutamiseks on tavaline lineaarteisenduste kompositsioon.

Et lõplikumõõtmelise vektorruumi lineaarteisendus on pööratav parajasti siis
kui tema determinant on nullist erinev, siis rühma $\GL(\V)$ kuuluvad
need ja ainult need lineaarteisendused, mille determinant pole null.
Kui vaatleme vaid lineaarteisendusi, mille determinant on üks, saame olulise
alamrühma $\SL(\V)$, mida nimetatakse vektorruumi $\V$
\emph{spetsiaalsete lineaarteisenduste rühmaks}.

Et igal vektorruumil leidub baas, siis võime vektorruumi $\V$ jaoks
fikseerida mingi baasi. Sel juhul
saame kõik lineaarteisendused esitada maatriksitena ning nõnda võime
edaspidi lineaarteisenduste rühmade asemel rääkida \emph{maatriksrühmadest}.
Kui $\{e_1, e_2, \dots, e_n\}$ moodustab vektorruumi $\V$ baasi ning
$\phi \col \V \to \V$ on mingi lineaarteisendus, siis talle vastav maatriks
selle baasi suhtes on $(a^i_j)$, mis on määratud valemiga
\begin{align*}
    \phi(e_j) = \sum_{i=1}^{n} a^i_j e_i.
\end{align*}

Selge, et vaadeldes rühmi $\GL(\V)$ ja $\SL(\V)$ maatriksrühmadena
on rühma tehteks juba tavaline maatriksite korrutamine. Ilmselt saab
nimetatud maatriksrühmad defineerida suvalise korpuste jaoks, ja nii ka
reaal- ning kompleksarvude korral. Sellest lähtuvalt kasutatakse sageli
nullist erineva determinandiga $n \times n$ maatriksrühmade tähistuseks
$\GL(n, \R)$ või $\GL(n, \C)$, ning neid rühmi nimetame vastavalt
\emph{reaalsete pööratavate lineaarteisenduste rühmaks} ja \emph{komplekssete
pööratavate lineaarteisenduste rühmaks}. Analoogiliselt on kasutusel
tähistused $\SL(n, \R)$ ja $\SL(n, \C)$.

Rühmal $\GL(n, \C)$ on palju tuntud alamrühmi. Klassikaliseks näiteks on
$n \times n$ \emph{ortogonaalsete maatriksite rühm} $\Ort(n, \C)$, kuhu kuuluvad
ortogonaalsed maatriksid, see tähendab sellised maatriksid $A$, mille korral
$A^T = A^{-1}$. Teise näitena võib tuua unitaarsete maatriksite rühma $\U(n)$,
mille elementideks on anti-Hermite'i maatriksid $A$, mis rahuldavad tingimust
$A^\dag = \overline{A}^T = -A$. Edasi on lihtne konstrueerida saadud alamrühmade
\emph{spetsiaalsed} variandid. \emph{Spetsiaalsete komplekssete ortogonaalmaatriksite rühm} on
\begin{align*}
    \SO(n, \C) = \Ort(n, \C) \cap \SL(n, \C),
\end{align*}
ja \emph{spetsiaalsete unitaarsete maatriksite rühmaks} on
\begin{align*}
    \SU(n) = \U(n) \cap \SL(n, \C).
\end{align*}

\begin{dfn}
    Olgu $\V$ vektorruum üle korpuse $K$. Kujutust
    $\blinv \col \V \times \V \to K$
    nimetatakse \emph{bilineaarvormiks}, kui iga $x, y, z \in \V$
    ja skalaaride $\lambda, \mu \in K$ korral
    \begin{enumerate}[label=\roman*.]
        \item $(\lambda x + \mu y, z) = \lambda (x, z) + \mu (y, z)$,
        \item $(x, \lambda y + \mu z) = \lambda (x, y) + \mu (x, z)$.
    \end{enumerate}
\end{dfn}

Kui vetorruumis $\V$ on antud baas $\{e_1, e_2, \dots, e_n\}$, siis saab
bilineaarvormi $\blinv \col \V \times \V \to K$ esitada talle vastava maatriksi
$B = (b_{ij})$ abil, kus $b_{ij} = (e_i, e_j)$. Tõepoolest, kui
meil on antud vektorid $x = \sum_i \lambda^i e_i$ ja $y = \sum_j \mu^j e_j$,
siis kasutades $\blinv$ lineaarsust mõlema muutuja järgi võime arvutada
\begin{align*}
    (x, y) = \sum_{i, j} b_{ij} \lambda^i \mu^j.
\end{align*}

Me ütleme, et bilineaarvorm $\blinv \col \V \times \V \to K$ on
\emph{sümmeetriline} kui
iga $x, y \in \V$ korral $(x, y) = (y, x)$. Selge, et sümmeetrilise
bilineaarvormi maatriksi $B$ korral kehtib võrdus $B = B^T$. Vormi $\blinv$
nimetatakse \emph{kaldsümmeetriliseks} kui iga $x, y \in \V$ korral kehtib
võrdus $(x, y) = - (y, x)$. Lihtne on veenduda, et kaldsümmeetrilise
bilineaarvormi korral rahuldab talle vastav maatriks $B$ seost
$B^T = -B$.

\begin{dfn}
    Olgu $\V$ vektorruum kus on fikseeritud mingi baas, olgu $\phi$
    vektorruumi $\V$ lineaarteisendus ning olgu $A = (a^i_j)$ on
    lineaarteisenduse $\phi$ maatriks fikseeritud baasi suhtes.
    Lineaarteisenduse $\phi$ \emph{jäljeks} nimetatakse kujutust
    $\Tr_\V \col \GL(\V) \to K$, kus
    \begin{align*}
        \Tr_\V(A) = \sum_i a^i_i.
    \end{align*}
\end{dfn}

Juhul kui maatriksi $A$ korral $\Tr_\V A = 0$, siis ütleme, et
maatriks $A$ on \emph{jäljeta}.

\begin{naide}
    On hästi teada, et vektorruumi $\V$ kõigi lineaarteisenduste hulk $\Lin \V$
    on ise ka vektorruum, kusjuures kui vektorruumi $\V$ dimensioon
    on $\dim(\V) = n$, siis ruumi $\Lin \V$ dimendsioon on
    $\dim \left(\Lin \V\right) = n^2$. Kasutades jälge $\Tr_\V$ saame
    defineerida bilineaarvormi
    $\blinv \col \Lin \V \times \Lin \V \to K$ järgmiselt:
    \begin{align*}
        (A, B) = \Tr_\V(AB),
    \end{align*}
    kus $A$ ja $B$ on maatriksid, mis vastavad vektorruumi $\Lin \V$
    teisendustele mingi baasi suhtes. Selge, et selliselt defineeritud
    bilineaarvorm sümmeetriline.
\end{naide}

Kasutades bilineaarvormi sümmeetrilisuse või kaldsümmeetrilisuse mõistet
saame sisse tuua \emph{ortogonaalsuse} mõiste. Me ütleme, et vektorid $x$ ja $y$
on bilineaarvormi $\blinv$ suhtes ortogonaalsed, kui $(x, y) = 0$. Selge, et
ortogonaalsuse tingimus ise on sümmeetriline, see tähendab kui $x$ on
ortogonaalne vektoriga $y$, siis kehtib ka vastupidine, $y$ on ortogonaalne
vektoriga $x$. Kui vektor $x \neq 0$ on iseenesega ortogonaalne, see tähendab
$(x, x) = 0$, siis nimetatakse vektorit $x$ \emph{isotroopseks}. Selge, et
Eukleidilises geomeetrias selliseid vektoreid ei leidu, kuid üldisemates
situatsioonides esinevad nad küllaltki sageli, näiteks
\emph{Minkowski aegruumis}.

Edasises vaatleme ortogonaaseid ja sümplektilisi rühmi ning selleks
nõuame, et vaatluse all olevad bilineaarvormid oleksid mittesingulaarsed
ehk regulaarsed, see tähendab kui $(x, y) = 0$ iga $y \in \V$ korral, siis
järelikult $x = 0$. Osutub, et bilineaarvorm $\blinv$ on regulaarne parajasti
siis, kui temale vastav maatiks $B = (b^i_j)$ on pööratav, mis tähendab,
et $\det B \neq 0$.

\begin{dfn}
    Me ütleme, et lineaarne operaator $\phi$ on \emph{ortogonaalne}
    regulaarse sümmeetrilise bilineaarvormi $\blinv$ suhtes, kui
    \begin{align*}
        (\phi(x), \phi(y)) = (x, y)
    \end{align*}
    kõikide $x$ ja $y$ korral vektorruumist $\V$.
\end{dfn}

Kui $x$ on ortogonaalse lineaarse operaatori $\phi$ tuumast, siis
kehtib $\phi(x) = 0$. Viimane aga tähendab, et iga $y \in \V$ korral
$(x, y) = (\phi(x), \phi(y)) = (0, \phi(y)) = 0$. Kokkuvõttes,
et $\blinv$ on regulaarne, siis järelikult $x = 0$ ja $\phi$ on üks-ühene.
Kui nüüd veel $\V$ on lõplikumõõtmeline, siis peab $\phi$ olema pööratav.
Seda arutelu silmas pidades võime öelda, et ortogonaalsed lineaarsed
operaatorid moodustavad rühma, mida me nimetame \emph{ortogonaalsete
lineaarteisenduste rühmaks} bilineaarvormi $\blinv$ suhtes.
Võttes tarvitusele vektorruumi $\V$ baasi saame konstrueerida ka
\emph{ortogonaalsete maatriksite rühma}, mida tähistatakse komplekssel juhul
kui $\Ort(n, \C)$, kus $n \in \N$ märgib, et tegu on $n \times n$ maatriksitega.

Sümplektiliste teisenduste tarvis tuleb vaadelda kaldsümmeetrilisi
bilineaarvorme.

\begin{dfn}
    Me ütleme, et lineaarne operaator $\phi$ on \emph{sümplektiline}
    regulaarse kaldsümmeetrilise bilineaarvormi $\blinv$ suhtes, kui
    \begin{align*}
        (\phi(x), \phi(y)) = (x, y)
    \end{align*}
    kõikide $x$ ja $y$ korral vektorruumist $\V$.
\end{dfn}

Märgime, et sümplektilised lineaarteisendused leiduvad ainult sellistes
vektorruumides, mille dimensioon on paarisarvuline, see tähendab
$\dim \V = 2n$, kus $n \in \N$. Sümplektilised teisendused moodustavad
\emph{sümplektiliste rühma}, mida tähistatakse kompleksel juhul
$\Sp(n, \C)$. Reaalsete sümplektiliste
teisenduste rühma saame kui vaatleme ühisosa rühmaga $\GL(2n, \R)$:
\begin{align*}
    \Sp(n, \R) = \Sp(n, \C) \cap \GL(2n, \R).
\end{align*}

\subsection{Eksponentsiaalkujutus}

Kõikide seni käsitluse all olnud maatriksrühmade esindajad peavad
vastavatesse rühmadesse kuulumiseks rahuldama mingeid algebralisi tingimusi.
Need tingimused võib kirja panna maatriksite elementide kaudu, mille
tulemusel saaame me mittelineaarseid võrrandeid, mis määravad rühma kuulumise.
Osutub, et need tingimused on võimalik asendada mingi hulga ekvivalentsete
lineaarsete võrranditega ja selline üleminek mittelineaarselt süsteemilt
lineaarsele ongi võtmetähtsusega idee üleminekul Lie rühmadest Lie
algebratele.\cite{johan1989survey}

Klassikaliseks viisiks kuidas sellist üleminekut realiseeritakse on
\emph{eksponentsiaalkujutuse} kasutuselevõtt. Nagu nimigi viitab, on
tegu analüüsist tuttava kujutuse üldistusega. Et meil oli siiani tegemist
vaid maatriksrühmadega, siis läheme siin ka edasi vaid eksponentsiaalkujutuse
ühe tähtsa erijuhuga, \emph{maatrikseksponentsiaaliga}, kuid olgu öeldud,
et järgnevad väited kehtivad tegelikult ka üldisemas seades,
nagu võib näha raamatus \cite{kirillov2008introduction}.

Olgu $A$ mingi $n \times n$ maatriks, $k \in \N$ ning olgu $I$
ühikmaatriksit. Tähistame $A^0 = I$ ning
$A^k = \underbrace{A \cdot A \cdot \ldots \cdot A}_{k\text{-korda}}$.

\begin{dfn}
    Olgu $X$ reaalne või kompleksne $n \times n$ maatriks. Maatriksi
    $X$ \emph{eksponendiks}, mida tähistatakse $e^X$ või $\exp X$, nimetatakse
    astmerida
    \begin{align}\label{eq:mat-exp}
        e^X = \sum_{k=0}^{\infty} \frac{X^k}{k!}.
    \end{align}
\end{dfn}

Ilmselt tuleks definitsiooni korrektsuses veendumaks näidata, et suvalise
maatriksi $X$ korral rida \eqref{eq:mat-exp} koondub. Selleks meenutame,
et $n \times n$ maatriksi $X = (X_{ij})$ normi arvutatakse valemi
\begin{align}\label{eq:mat-norm}
    \| X \| = \left( \sum_{i,j=1}^n |X_{ij}|^2 \right)^{\frac{1}{2}}
\end{align}
järgi. Arvestades, et $\| XY \| \le \|X\| \|Y\|$, siis $\|X^k\| \le \|X\|^k$.
Rakendades nüüd normi \eqref{eq:mat-norm} rea \eqref{eq:mat-exp} liikmetele
saame
\begin{align*}
    \sum_{k=0}^\infty \left\lVert \frac{X^k}{k!} \right\rVert \le
    \sum_{k=0}^\infty \frac{\|X\|^k}{k!} = e^{\|X\|} < \infty,
\end{align*}
mis tähendab, et rida \eqref{eq:mat-exp} koondub absoluutselt ja seega
ta ka koondub. Märkamaks, et $e^X$ on pidev funktsioon märgime esiteks,
et $X^k$ on argumendi $X$ suhtes pidev funktsioon ja seega on rea
\eqref{eq:mat-exp} osasummad pidevad. Teisalt paneme tähele, et
\eqref{eq:mat-exp} koondub ühtlaselt hulkadel, mis on kujul
$\{ \|X\| \le R \}$, ja seega on rida kokkuvõttes pidev.

Seega on maatrikseksponentsiaal korrektselt defineeritud ning ka pidev.
Järgmises lauses on toodud rida eksponentsiaalkujutuse põhilisi omadusi,
mille võrdlemisi lihtsad tõestused võib huvi korral võib leida näiteks
teosest \cite{hall2003lie}.

\begin{lau}\label{lau:mat-exp-om}
    Olgu $X$ ja $Y$ suvalised $n \times n$ maatriksid. Siis kehtivad järgmised
    väited:
    \begin{enumerate}[label=\arabic*)]
        \item\label{om:mat-exp-1} $e^0 = I$,
        \item\label{om:mat-exp-2} $\left(e^X\right)^T = e^{X^T}$,
        \item\label{om:mat-exp-3} $e^X$ on pööratav ning kehtib
            $\left(e^X\right)^{-1} = e^{X^{-1}}$,
        \item\label{om:mat-exp-4} $e^{(\lambda + \mu)X} = e^{\lambda X} e^{\mu X}$ suvaliste
            $\lambda, \mu \in \C$ korral,
        \item\label{om:mat-exp-5} kui $XY = YX$, siis $e^{X+Y} = e^X e^Y = e^Y e^X$,
        \item\label{om:mat-exp-6}  kui $C$ on pööratav, siis $e^{CXC^{-1}} = C e^X C^{-1}$,
        \item\label{om:mat-exp-7} $\det e^X = e^{\Tr_\V X}$.
    \end{enumerate}
\end{lau}

Ostutub, et rühma $\GL(n, \C)$ ühikelemendi mingis ümbruses on võimalik
suvaline maatriks esitada kujul $e^A$, kus $A$ on mingi $n \times n$
maatriks. Rühma $\GL(n, \R)$ korral on maatriks $A$ reaalne. Oluline on
tähele panna, et vaadeldes rühma $\GL(n, \R)$, ei ole eksponentfunktsiooni
kujutis terve rühm. Selles veenumiseks piisab võtta $n = 1$ ning näha, et
$\exp \left(\GL(1, \R)\right) = \R^+$, ehk kujutiseks on reaaltelje positiivne
osa, samas kui $\GL(1, \R) = \R \setminus \{0\}$ ehk reaaltelg ilma
nullpunktita.

Niisiis eksponentkujutust kasutades on oht kaotada rühma globaalne struktuur,
samas kui lokaalne struktuur säilib.

Et maatriksi $A$ korral kuuluks maatriks $e^A$ mõnda puntis
\ref{subsec:mat-ryhmad-ja-bilinvorm}. \nameref{subsec:mat-ryhmad-ja-bilinvorm} nimetatud rühma tuleb maatriksile
$A$ seada mingid lineaarsed kitsendused. Näiteks spetsiaalse lineaarse rühma
$\SL(n)$ korral võime mittelineaarse tingimuse $e^A$ determinandi kohta
asendada lineaarse tingimusega maatriksi $A$ jälje kohta kasutades lause
\ref{lau:mat-exp-om} punkti \ref{om:mat-exp-7}. Nii on näiteks
$\det e^A = 1$ parajasti siis, kui $\Tr_\V A = 0$.

Kokkuvõttes nägime, et eksponentsiaalkujutuse abil on võimalik asendada
klassikalised maatriksrühmad maatrikshulkadega, millele on seatud
teatud lineaarsed kitsendused. Selge, et need hulgad on kinnised
lineaarkombinatsioonide suhtes ja nii võib neid vaadelda kui vektorruume.
Tavalise maatriksite korrutamise osas kahjuks kinnisus säilida ei
pruugi. Samas kui meil on $n \times n$ maatriksid $A$ ja $B$, mis on vastavalt
kas kaldsümmeetrilised, rahuldavad anti-Hermite'i tingimust või neil
puudub jälg, siis maatriksil $C = AB - BA$ on samuti selline omadus.
Niisiis saadud maatrikshulgad ei moodusta ainuüksi vektorruumi,
vaid on kinnised ka teatud binaarse tehte suhtes.

\subsection{Lie algebra definitsioon}

Enne kui Lie algebra definitsiooni anname tuletame meelde, et \emph{algebraks}
üle korpuse $K$ nimetatakse vektorruumi $\V$ üle korpuse $K$, millel on
defineeritud bilineaarne korrutamine $\V \times \V \to \V$. Kui algebra
tehe rahuldab assotsiatiivsuse tingimust, siis nimetatakse seda algebrat
assotsiatiivsekt ning vastasel korral mitteassotsiatiivseks. Nii on näiteks
vektorruumi $\V$ lineaarteisenduste vektorruum $\Lin \V$ assotsiatiivne algebra,
mille tehteks on teisenduste kompositsioon: $f \circ g$. Samas võime
vektorruumi $\Lin \V$ varustada ka teistsuguse korrutamisega ning saada uue
algebralise struktuuri, kui võtame tehteks näiteks $f \circ g - g \circ f$.
Üldiselt selline korrutamine aga enam kommutatiivne ei ole.

\begin{dfn}[Lie algebra]\label{def:lie-algebra}
    Algebrat $\g$ üle korpuse $K$ nimetatakse \emph{Lie algebraks}, kui
    tema korrutamine $[\cdot, \cdot] \col \V \times \V \to \V$ rahuldab
    kõikide $x, y, z \in \g$ tingimusi
    \begin{gather}
        [x, x] = 0,\label{id:bracket-antisymm} \\
        \brac{x}{\brac{y}{z}} + \brac{y}{\brac{z}{x}} +
            \brac{z}{\brac{x}{y}} = 0.\label{id:jacobi}
    \end{gather}
\end{dfn}


Me ütleme definitsioonis toodud korrutise $[x, y]$ kohta elementide $x$ ja
$y$ \emph{Lie sulg}, ning bilineaarvormi $[\cdot, \cdot]$ kohta öeldakse ka
\emph{kommutaator}. Definitsioonis toodud samasust \eqref{id:jacobi}
nimetatakse \emph{Jacobi samasuseks}. Sageli on otstarbekas tähistada Lie
algebrat $\g$ paarina $(\g, [\cdot, \cdot])$.

\begin{markus}
    Paljudes käsitlustes antakse Lie algebrale veidi üldisem definitsioon,
    kui algebrat $\g$ ei vaaldeda mitte vektorruumina üle korpuse,
    vaid moodulina üle ringi, nagu seda on tehtud näiteks viites
    \cite{bourbaki1989lie}.
\end{markus}

Piltlikult öeldes mõõdab kommutaator
algebra elementide mittekommuteeruvust ja seda asjaolu kirjeldavat võrdust
\eqref{id:bracket-antisymm} võime kirjutada ka kujul
\begin{equation}\label{eq:bracket-antisymm}
    [x, y] = -[y, x].
\end{equation}
Tõepoolest, \eqref{id:bracket-antisymm} järgi kehtib $[x+y, x+y] = 0$,
millest saame bilineaarsuse abil $[x, x] + [x, y] + [y, x] + [y, y] = 0$,
ehk kehtibki $[x, y] = -[y, x]$.

Rakendades võrdust \eqref{eq:bracket-antisymm} saame Jacobi samasuse
kirjutada kui
\begin{equation}
    \brac{x}{\brac{y}{z}} = \brac{\brac{x}{y}}{z} + \brac{y}{\brac{x}{z}}.
\end{equation}

\begin{naide}\label{naide:lie-algebra-konstrueerimine}
    Olgu $\A$ algebra, millel on assotriatiivne korrutustehe
    $\star \col \A \times \A \to \A$. Defineerides kommutaatori valemiga
    \begin{align}\label{eq:naide-lie-sulg}
        [x, y] = x \star y - y \star x,\quad x, y \in \A,
    \end{align}
    saame algebrast $\A$ moodustada Lie algebra $\A_L$. Valemist
    \eqref{eq:naide-lie-sulg} järeldub vahetult, et Lie algebra
    definitsiooni nõue \eqref{id:bracket-antisymm} kehtib. Jacobi samasuse
    kehvivuseks märgime, et
    \begin{align*}
        &\left[x, \left[y, z\right]\right] + 
            \left[y, \left[z, x\right]\right] + 
            \left[z, \left[x, y\right]\right] = \\
        =& \left[x, y \star z - z \star y \right] + 
            \left[y, z \star x - x \star z \right] + 
            \left[z, x \star y - y \star x \right] = \\
        =& \left[x, y \star z\right] - \left[x, z \star y\right] + 
            \left[y, z \star x\right] - \left[x, x \star z\right] + 
            \left[z, x \star y\right] - \left[z, y \star x\right] = \\
        =&\ x \star y \star z - y \star z \star x - x \star z \star y +
            z \star y \star x + y \star z \star x - z \star x \star y\, - \\
         &\ y \star x \star z + x \star z \star y + z \star x \star y -
            x \star y \star z - z \star y \star x + y \star x \star z = 0,
    \end{align*}
    kus $x, y, z \in \A$.
\end{naide}

Niisiis suvalisest assotsiatiivsest algebrast on võimalik konstrueerida
Lie algebra. Arvestades, et maatriksite ja lineaarteisenduste korrutamine
rahuldavad assotsiatiivsuse tingimust, on näites
\ref{naide:lie-algebra-konstrueerimine} esitatud eeskirja abil
võimalik kõiki punktis \ref{subsec:mat-ryhmad-ja-bilinvorm} toodud
rühmi võimalik vaadelda kui Lie algebraid. Märgime, et Lie algebrate
tähistamiseks kasutatakse tavaliselt väikeseid gooti tähti, seega näiteks
pööratavate lineaarteisenduste rühmale $\GL(n)$ vastavaks Lie algebraks on
$\mathfrak{gl(n)}$.

\subsection{Struktuurikonstandid}

Eeldame, et järgnevas on meil antud lõplikumõõtmeline Lie algebra $\g$,
üle korpuse $K$, mille vektorruumil on fikseerutd baas
$\{ e_1, e_2, \dots, e_n \}$. Siinjuures tuletame meelde, et Lie algebra
aluseks oleva vektorruumi baasi elemente nimetatakse sageli selle
Lie algebra \emph{generaatoriteks}. Et Lie sulg
$[\cdot, \cdot]$ on bilineaarne vorm, siis tema väärtused Lie algebral
$\g$ on täielikult määratud, kui me teame millega võrduvad
$[e_\alpha, e_\beta]$, kus $\alpha, \beta \in \{1, 2, \dots, n\}$.
Tõepoolest, suvalise vektori võime esitada baasivektorite
$e_1, e_2, \dots, e_n$ lineaarkombinatsioonina ja kõikide vektorite
$x, y \in \g$ korral leiduvad $a^\alpha, b^\beta \in K$ nii, et
$x = a^\alpha e_\alpha$ ja $y = b^\beta e_\beta$, kus
$\alpha, \beta = 1, 2, \dots, n$. Niisiis saame $[x, y]$ välja arvutada
järgmiselt:
\begin{align*}
    [x, y] = [a^\alpha e_\alpha, b^\beta e_\beta] = 
    a^\alpha b^\beta [e_\alpha, e_\beta].
\end{align*}
Iga $\alpha, \beta \in \{1, 2, \dots, n\}$ korral võime omakorda ka
vektori $[e_\alpha, e_\beta]$ avaldata lineaarkombinatsioonina
baasivektoritest kujul
\begin{align*}
    [e_\alpha, e_\beta] = K_{\alpha \beta}^{\lambda} e_\lambda,
\end{align*}
ja seega jääb meile $[x, y]$ arvutamiseks lõpuks võrdus
\begin{align*}
    [x, y] = a^\alpha b^\beta K_{\alpha \beta}^{\lambda} e_\lambda.
\end{align*}

\begin{dfn}
    Olgu $\g$ lõplikumõõtmeline Lie algebra ning $\{ e_1, e_2, \dots, e_n \}$
    selle Lie algebra vektorruumi baas. Tähistades
    $[e_\alpha, e_\beta] = K_{\alpha \beta}^{\lambda} e_\lambda$, kus
    $\alpha, \beta \in \{1, 2, \dots, n\}$, siis arve
    $K_{\alpha \beta}^{\lambda}$ nimetatakse Lie algebra $\g$
    \emph{struktuurikonstantideks}.
\end{dfn}

Kasutades kommutaatori $[\cdot, \cdot]$ kaldsümmeetrilisust saame
struktuurikonstantide kohta valemi
\begin{align}\label{om:str-konst-kald-symm}
    K_{\alpha \beta}^{\lambda} = -K_{\beta \alpha}^{\lambda}.
\end{align}
Jacobi samasuse abil saame veel teisegi tingimuse, mida struktuurikonstandid
rahuldama peavad.
\begin{align*}
    \left[e_\alpha, \left[ e_\beta, e_\gamma \right] \right] +
    \left[e_\beta, \left[ e_\gamma, e_\alpha \right] \right] +
    \left[e_\gamma, \left[ e_\alpha, e_\beta \right] \right] = 0, \\
    %
    \left[e_\alpha, K_{\beta \gamma}^{\lambda} e_\lambda \right] +
    \left[e_\beta, K_{\gamma \alpha}^{\lambda} e_\lambda \right] +
    \left[e_\gamma, K_{\alpha \beta}^{\lambda} e_\lambda \right] = 0, \\
    %
    K_{\beta \gamma}^{\lambda} \left[e_\alpha, e_\lambda \right] +
    K_{\gamma \alpha}^{\lambda} \left[e_\beta, e_\lambda \right] +
    K_{\alpha \beta}^{\lambda} \left[e_\gamma, e_\lambda \right] = 0, \\
    %
    K_{\beta \gamma}^{\lambda} K_{\alpha \lambda}^{\mu} e_{\mu} +
    K_{\gamma \alpha}^{\lambda} K_{\beta \lambda}^{\mu} e_{\mu} +
    K_{\alpha \beta}^{\lambda} K_{\gamma \lambda}^{\mu} e_{\mu} = 0, \\
    %
    K_{\beta \gamma}^{\lambda} K_{\alpha \lambda}^{\mu} +
    K_{\gamma \alpha}^{\lambda} K_{\beta \lambda}^{\mu} +
    K_{\alpha \beta}^{\lambda} K_{\gamma \lambda}^{\mu} = 0.
\end{align*}

\begin{naide}
    Vaatleme eespool näiteks toodud spetsiaalsete unitaarsete maatriksite
    rühma $\SU(n)$ erijuhul $n = 2$. Rühmale $\SU(2)$ vastab Lie algebra
    $\su(2)$, mille
    element $x \in \su(2)$ rahuldab tingimusi
    \begin{align}
        \Tr x = 0,\label{eq:su2-ting1} \\
        x^\dag + x = 0.\label{eq:su2-ting2}
    \end{align}

    Tingimuste \eqref{eq:su2-ting1} ja \eqref{eq:su2-ting2} põhjal
    on võimalik näidata, et Lie algebra $\su(2)$ generaatoriteks
    on maatriksid
    \begin{align*}
        \rho_1 = \begin{pmatrix}
            0 & i \\
            i & 0
        \end{pmatrix},
        \quad
        \rho_2 = \begin{pmatrix}
            0 & 1 \\
            -1 & 0
        \end{pmatrix},
        \quad
        \rho_3 = \begin{pmatrix}
            i & 0 \\
            0 & -i
        \end{pmatrix},
    \end{align*}
    nagu võib näha bakalaureusetöös \cite{latt2013}.

    Arvutame Lie algebra $\su(2)$ generaatoritel Lie sulu väärtused
    ja leiame seeläbi struktuurikonstandid. Ilmselt kõik suvalise
    $\alpha \in \{1, 2, 3\}$ korral $[\rho_\alpha, \rho_\alpha] = 0$.
    Arvestades veel omadust \eqref{om:str-konst-kald-symm} on meile
    huvi pakkuvad kommutaatorid on vaid kujul
    $[\rho_\alpha, \rho_\beta]$, kus $\alpha < \beta$.
    \begin{align}
        [\rho_1, \rho_2] &= \rho_1 \rho_2 - \rho_2 \rho_1 =
            \begin{pmatrix}
                -2i &  0 \\
                  0 & 2i
            \end{pmatrix} = -2 \rho_3, \label{eq:str-konst-naide-brac-1}\\[0.1cm]
        [\rho_1, \rho_3] &= \rho_1 \rho_3 - \rho_3 \rho_1 =
            \begin{pmatrix}
                 0 & 2 \\
                -2 & 0
            \end{pmatrix} = 2 \rho_2, \label{eq:str-konst-naide-brac-2}\\[0.1cm]
        [\rho_2, \rho_3] &= \rho_2 \rho_3 - \rho_3 \rho_2 =
            \begin{pmatrix}
                  0 & -2i \\
                -2i & 0
            \end{pmatrix} = -2 \rho_1. \label{eq:str-konst-naide-brac-3}
    \end{align}

    Seega võrduste \eqref{eq:str-konst-naide-brac-1},
    \eqref{eq:str-konst-naide-brac-2} ja \eqref{eq:str-konst-naide-brac-3}
    põhjal on antud baasi suhtes nullist erinevad struktuurikonstandid
    \begin{multicols}{3}
        \begin{enumerate}
            \item $K_{12}^{3} = -2$,
            \item $K_{21}^{3} = 2$,
            %
            \item $K_{13}^{2} = 2$,
            \item $K_{31}^{2} = -2$,
            %
            \item $K_{23}^{1} = -2$,
            \item $K_{32}^{1} = 2$.
        \end{enumerate}
    \end{multicols}

    Valides näiteks
        $x = \begin{pmatrix}
              3i & 7+i \\
            -7+i & -3i
        \end{pmatrix}$
        ja
        $y = \begin{pmatrix}
            i & 5 + 2i \\
            -5+2i &     -i
        \end{pmatrix}$,
    siis ilmselt $x, y  \in \su(2)$, ja saame arvutada Lie sulu $[x, y]$.
    Nüüd ühelt poolt vahetu arvutuse tulemusena
    \begin{align*}
        [x, y] &= xy - yx =
            \begin{pmatrix}
                -40+9i &  -5+8i \\
                  5+8i & -40-9i
            \end{pmatrix} - 
            \begin{pmatrix}
                -40-9i &   5-8i \\
                 -5-8i & -40+9i
            \end{pmatrix} = \\[0.1cm]
            &= \begin{pmatrix}
                   18i & -10+16i \\
                10+16i &    -18i
            \end{pmatrix},
    \end{align*}
    kuid teisalt $x = \rho_1 + 7\rho_2 + 3\rho_3$ ja
    $y = 2\rho_1 + 5\rho_2 + \rho_3$, ning saame kasutada
    struktuurikonstante:
    \begin{align*}
        [x, y] &= [\rho_1 + 7\rho_2 + 3\rho_3,
                  2\rho_1 + 5\rho_2 + \rho_3] = \\
        &= 2[\rho_1, \rho_1] + 5[\rho_1, \rho_2] + [\rho_1, \rho_3] +
            14[\rho_2, \rho_1] + 35[\rho_2, \rho_2] +
            7[\rho_2, \rho_3] + \\
        &\phantom{=}\, \ 6[\rho_3, \rho_1] + 15[\rho_3, \rho_2] +
            3[\rho_3, \rho_3] = \\
        &= 5[\rho_1, \rho_2] + [\rho_1, \rho_3] - 14[\rho_1, \rho_2] +
            7[\rho_2, \rho_3] - 6[\rho_1, \rho_3] -
            15[\rho_2, \rho_3] = \\
        &= -9[\rho_1, \rho_2] -5[\rho_1, \rho_3] -
            8[\rho_2, \rho_3] = \\
        &= -9 K_{12}^{\lambda} \rho_\lambda
           -5 K_{13}^{\lambda} \rho_\lambda
           -8 K_{23}^{\lambda} \rho_\lambda = \\
        &= -9 K_{12}^{3} \rho_3
           -5 K_{13}^{2} \rho_2
           -8 K_{23}^{1} \rho_1 = \\
        &= -9 \cdot (-2) \rho_3
           -5 \cdot 2 \rho_2
           -8 \cdot (-2) \rho_1 = \\
        &= 16 \rho_1 - 10 \rho_2 + 18 \rho_3 = \\[0.1cm]
        &= \begin{pmatrix}
               18i & -10+16i \\
            10+16i &    -18i
        \end{pmatrix}.
    \end{align*}

    Ootuspäraselt annavad mõlemad variandid sama tulemuse, kuid teise
    variandi puhul ei soorita me kordagi selle algebra korrutustehet.
\end{naide}

Ilmselt sõltuvad Lie algebra struktuurikonstandid algebra vektorruumi
baasist. Olgu $\{ e_1, e_2, \dots, e_n \}$ ja
$\{ \hat{e}_1, \hat{e}_2, \dots, \hat{e}_n \}$ Lie algebra $\g$
vektorruumi kaks erinevad baasi, ning olgu nad omavahel järgmises
seoses:
\begin{align}\label{eq:str-konst-baasi-yleminek-1}
    \hat{e}_\alpha = a_\alpha^\lambda e_\lambda.
\end{align}

Siis
\begin{align}\label{eq:str-konst-baasi-yleminek-2}
    \hat{K}_{\alpha \beta}^{\xi} \hat{e}_\xi = 
    [\hat{e}_\alpha, \hat{e}_\beta] =
    [a_\alpha^\mu e_\mu, a_\beta^\nu e_\nu] =
    a_\alpha^\mu a_\beta^\nu [e_\mu, e_\nu] =
    a_\alpha^\mu a_\beta^\nu K_{\mu \nu}^{\lambda} e_\lambda.
\end{align}
Kui võrduste ahela \eqref{eq:str-konst-baasi-yleminek-2} kõige
vasakpoolsemas osas kasutada samasust
\eqref{eq:str-konst-baasi-yleminek-1}, siis kehtib
\begin{align*}
    \hat{K}_{\alpha \beta}^{\xi} a_\xi^\lambda e_\lambda =
    a_\alpha^\mu a_\beta^\nu K_{\mu \nu}^{\lambda} e_\lambda,
\end{align*}
millest kokkuvõttes saame valemi
\begin{align*}
    a_\xi^\lambda \hat{K}_{\alpha \beta}^{\xi} =
    a_\alpha^\mu a_\beta^\nu K_{\mu \nu}^{\lambda}.
\end{align*}

\subsection{Esitused}

Olgu $(\g_1, [\cdot, \cdot]_{\g_1})$ ja $(\g_2, [\cdot, \cdot]_{\g_2})$ Lie
algebrad. Lineaarkujutust $\varphi \col \g_1 \to \g_2$ nimetatakse
Lie algebrate \emph{homomorfismiks}, kui ta säilitab kommutaatori, see
tähendab iga $x, y \in \g_1$ korral kehtib võrdus
\begin{align*}
    \varphi([x, y]_{\g_1}) = [\varphi(x), \varphi(y)]_{\g_2}.
\end{align*}
Homomorfismi $\varphi$ nimetatakse Lie algebrate \emph{isomorfismiks},
kui $\varphi$ on ka üks-ühene ning pealekujutus. Klassikalisel viisil on
defineeritud ka Lie algebrate \emph{endomorfismi} ning \emph{automorfismi}
mõisted.

Homomorfismi mõiste viib meid väga tähtsa osani Lie algebrate teoorias,
milleks on \emph{esitused}.

\begin{dfn}
    Olgu $\g$ Lie algebra ja $\V$ vektorruum. Siis nimetatakse homomorfismi
    $\varphi \col \g \to \gl(\V)$ Lie algebra $\g$ \emph{esituseks}
    vektorruumile $\V$.
\end{dfn}

Esiteks paneme tähele, et selline definitsioon omab mõtet, kuna
eelneva arutelu põhjal on selge, et $\gl(\V)$ on tõepoolest Lie algebra.
Niisiis on $\g$ esitus vektorruumil $\V$ lineaarne kujutus $\varphi$
Lie algebrast $\g$ vektorruumi $\V$ endomorfismide ringi nii, et
\begin{align*}
    \varphi([x, y])(v) = \left( \varphi(x)\varphi(y) \right) (v) -
                         \left( \varphi(y)\varphi(x) \right) (v),
\end{align*}
kõikide $x, y \in \g$ ja $v \in \V$ korral. Me ütleme, et esitus $\varphi$
on \emph{täpne}, kui $\varphi(x) = 0$ siis ja ainult siis, kui $x = 0$. Teisi
sõnu on esitus täpne parajasti siis, kui ta on üks-ühene.

\begin{naide}\label{naide:adjoint-repr}
    Olgu $\g$ Lie algebra. Vaatleme lineaarset kujutust
    \begin{align*}
        \ad \col \g \ni x \mapsto \ad_x \in \gl(\g),
    \end{align*}
    mis on $y \in \g$ korral on defineeritud eeskirjaga
    \begin{align}\label{def:ad}
        \ad_x(y) = [x, y].
    \end{align}
    Kujutus $\ad \col \g \to \gl(\g)$ on Lie algebra $\g$ esitus. Seda esitust
    nimetatakse $\g$ \emph{adjungeeritud esituseks}. Märkamaks, et $\ad$ on
    tõepoolest esitus märgime kõigepealt, et arvestades eeskirja
    \eqref{def:ad} ning kommutaatori lineaarsust, on kujutuse $\ad$
    lineaarsus ilmne. Veendumaks, et kujutus $\ad \col \g \to \gl(\g)$ on
    esitus tuleb kontrollida, et iga $x, y, z \in \g$ korral kehtiks võrdus
    $\left[ \ad_x, \ad_y \right](z) = \ad_{[x, y]}(z)$, mille saame Jaboci
    samasusest \eqref{id:jacobi}.
    \begin{align*}
        0 &= \brac{x}{\brac{y}{z}} + \brac{y}{\brac{z}{x}} +
             \brac{z}{\brac{x}{y}} = \\
          &= \brac{x}{\brac{y}{z}} + \brac{y}{-\brac{x}{z}} -
             \brac{\brac{x}{y}}{z} = \\
          &= \brac{x}{\brac{y}{z}} - \brac{y}{\brac{x}{z}} -
             \brac{\brac{x}{y}}{z}
    \end{align*}
    ehk $\brac{x}{\brac{y}{z}} - \brac{y}{\brac{x}{z}} = \brac{\brac{x}{y}}{z}$.
    Nüüd, et Lie algebras $\gl(\g)$ arvutatakse kommutaatori
    väärtusi eeskirja
    $[\ad_x, \ad_y] = \ad_x \circ \ad_y - \ad_y \circ \ad_x$ järgi, siis
    \begin{align*}
        [\ad_x, \ad_y](z) = \brac{x}{\brac{y}{z}} - \brac{y}{\brac{x}{z}} =
        \brac{\brac{x}{y}}{z} = \ad_{\brac{x}{y}}(z).
    \end{align*}
\end{naide}

Esituste teooria üheks väga oluliseks tulemuseks on nõndanimetatud
\emph{Ado teoreem}, mis väidab, et iga lõplikumõõtmeline Lie algebra $\g$
on Lie alebra $\gl(\V)$ alamalgebra, kus $\V$ on mingi lõplikumõõtmeline
vektorruum. Niisiis, Lie algebrat $\g$ on tegelikult võimalik vaadelda kui
maatriksalgebrat.\cite{hall2003lie}

\begin{thm}[Ado, 1935]
    Iga lõplikumõõtmeline Lie algebra üle nullkarakteristikaga korpuse omab
    täpset lõplikumõõtmelist esitust.
\end{thm}

Tegelikult kehtib ka Ado teoreemi oluliselt tugevam variant,
kus on kaotatud eeldus korpuse nullkarakteristika
kohta.\cite{hochschild1966}