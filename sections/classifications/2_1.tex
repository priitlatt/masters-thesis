%!TEX root = ../../thesis.tex

\subsection{\texorpdfstring{$2|1$}{2|1}-dimensionaalsed
    \texorpdfstring{$3$}{3}-Lie superalgebrad}

Olgu $3$-Lie superalgebra $\g$ supervektorruumi dimensioon
$2|1$ ja olgu tema generaatoriteks $\{ e_1, e_2, f_1 \}$. Siis
\begin{align*}
    \begin{matrix*}[l]
        & [e_1, e_1, e_1] = 0, \quad
        & [e_1, f_1, f_1] = \l_1 e_1 + \l_2 e_2, \\
        %
        & [e_1, e_1, e_2] = 0, \quad
        & [e_2, e_2, e_2] = 0, \\
        %
        & [e_1, e_1, f_1] = 0, \quad
        & [e_2, e_2, f_1] = 0, \\
        %
        & [e_1, e_2, e_2] = 0, \quad
        & [e_2, f_1, f_1] = \l_3 e_1 +\l_4 e_2, \\
        %
        & [e_1, e_2, f_1] = \m_1 f_1, \quad
        & [f_1, f_1, f_1] = \m_2 f_1,
    \end{matrix*}
\end{align*}
kus $\l_1, \l_2, \l_3, \l_4, \m_1, \m_2 \in \C$ on suvalised skalaarid.

Algoritmi tulemusel saame $24$ mittetriviaalset kitsendust\footnote{
Need on leitavad veebilehelt
\url{https://www.example.com/2_1_restrictions.pdf}},
mida $3$-Lie superalgebra $\g$ struktuurikonstandid rahuldama peavad.
Kitsenduste mittetriviaalseteks lahenditeks sobivad seejuures
struktuurikonstantide komplektid
\renewcommand\arraystretch{1.3}
\begin{align*}
    \left\{ \begin{matrix*}[l]
        \m_1 = 0 \\
        \m_2 = 0 \\
        \l_1 = c_1 \\
        \l_2 = c_2 \\
        \l_3 = -\frac{c_1^2}{c_2} \\
        \l_4 = -c_1 \\
    \end{matrix*} \right.,
    \quad
    \left\{ \begin{matrix*}
        \m_1 = c_3 \\
        \m_2 = 0 \\
        \l_1 = 0 \\
        \l_2 = 0 \\
        \l_3 = 0 \\
        \l_4 = 0 \\
    \end{matrix*} \right.,
    \quad
    \left\{ \begin{matrix*}
        \m_1 = 0 \\
        \m_2 = 0 \\
        \l_1 = 0 \\
        \l_2 = 0 \\
        \l_3 = c_4 \\
        \l_4 = 0 \\
    \end{matrix*} \right.,
\end{align*}
\renewcommand\arraystretch{1}
kus $c_1, c_2, c_3, c_4 \in \C \setminus \{0\}$ on vabad muutujad.

\begin{thm}
    Olgu $3$-Lie superalgebra $\g$ supervektorruumi dimensioon $2|1$.
    Siis $\g$ on kas Abeli Lie superalgebra või tema mittetriviaalsed
    kommutatsiooniseosed on
    \renewcommand\arraystretch{1.2}
    \begin{align}
        & \left\{
            \begin{matrix*}[l]
                [e_1, f_1, f_1] = c_1 e_1 + c_2 e_2, \\
                [e_2, f_1, f_1] = -\frac{c_1^2}{c_2} e_1 - c_1 e_2, \\
            \end{matrix*}
        \right. \\[0.2cm]
        &\ \ \ [e_1, e_2, f_1] = c f_1, \\[0.2cm]
        &\ \ \ [f_1, f_1, f_1] = c f_1,
    \end{align}
    \renewcommand\arraystretch{1}
    kus $c, c_1, c_2 \in \C \setminus \{ 0 \}$ on vabad muutujad.
\end{thm}
