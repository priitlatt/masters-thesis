%!TEX root = ../thesis.tex

%%%%%%%%%%%%%%%%%%%%
%% n-Lie algebra  %%
%%%%%%%%%%%%%%%%%%%%

\section{\texorpdfstring{$n$}\ -Lie algebra}

Selles peatüki eesmärgiks on klassikalise Lie algebra
üldistame, mille käigus toome sisse $n$-Lie algebra mõiste.
Edasi tutvustame artiklis \cite{AMS:2011} näidatud eeskirja,
mille abil on võimalik Lie algebrast konstrueerida
ternaarne Lie algebra ning jätkame seda teooriaarendust
tuginedes artiklile \cite{AKMS:2014}, kus kirjeldatakse
kuidas $n$-Lie algebrast indutseerida $(n+1)$-Lie algebra, ja
tõestame sellega seonduvad põhilised tulemused.

\subsection{\texorpdfstring{$n$}\ -Lie algebra definitsioon}

Lie algebra definitsioonis on kesksel kohal kaldsümmeetriline
bilineaarne korrutustehe, mis rahuldab Jacobi samasust.
Üheks viisiks Lie algebra mõistet üldistada, ongi just nimelt
tema korrutamise üldistamine. Seda tehes on loomulik nõuda, et
ka üldistatud korrutamistehe rahuldaks kaldsümmeetrilisuse
tingimust ning Jacobi samasust, või vähemalt selle mingit
analoogi, mis annaks klassikalisel juhul täpselt
Jacobi samasuse.

Filippov\footnote{Aleksei Fedorovich Filippov (1923 -- 2006),
vene matemaatik} näitas aastal 1985 artikis \cite{filippov1985}
$n$-Lie algebrate klassi, kus bilineaarne korrutamine on
asendatud $n$-lineaarse kaldsümmeetrilise operatsiooniga,
mis rahuldab teatud samasust.\cite{kasymov1987}
Tänaseks on just see, Nambu mehaanikast välja kasvanud
üldistus osutunud üheks põhiliseks Lie algebrate edasiseks
uurimissuunaks.

\begin{dfn}[$n$-Lie algebra]
    Vektorruumi $\A$ nimetatakse \emph{$n$-Lie algebraks}, kui
    on määratud $n$-lineaarne kaldsümmeetriline kujutus
    $\nbrac{\cdot}{\cdot} \colon \A^n \to \A$, mis
    suvaliste
    \[ x_1, \dots, x_{n-1}, y_1, \dots, y_n \in \A \]
    korral rahuldab tingimust
    \begin{align}\label{id:filippov}
        \left[ x_1, \dots, x_{n-1}, \nbrac{y_1}{y_n} \right] =
        \sum_{i=1}^n \left[
            y_1, \dots, \left[ x_1, \dots, x_{n-1}, y_i \right], \dots, y_n
        \right].
    \end{align}
\end{dfn}

Võrdust \eqref{id:filippov} $n$-Lie algebra definitsiooniks
nimetatakse üldistatud Jacobi samasuseks või ka
\emph{Filippovi samasuseks}. Vahetu kontrolli põhjal
on selge, et valides $n = 2$, saame Filippovi samasusest
\eqref{id:filippov} Jacobi samasuse \eqref{id:jacobi}.
$n$-aarse Lie sulu kaldsümmeetrilisus tähendab, et
suvaliste $x_1, x_2, \dots, x_n \in \A$ korral
\begin{align*}
    [x_1, \dots, x_i, x_{i+1}, \dots, x_n] =
    -[x_1, \dots, x_{i+1}, x_i, \dots, x_n].
\end{align*}

Toome siinkohal $n$-Lie algebra kohta näite, mille
Filippov esitas oma artiklis \cite{filippov1985} vahetult
pärast oma definitsiooni.

\begin{naide}
    Olgu $E$ reaalne $(n+1)$-mõõtmeline Eukleidiline ruum,
    ning tähistame elementide $x_1, x_2, \dots, x_n \in E$
    vektorkorrutise $[x_1, x_2, \dots, x_n]$. Meenutame, 
    et vektorkorrutis on kaldsümmeetriline ning iga teguri
    suhtes lineaarne. Lisaks, kui meil on ruumi $E$ mingi
    baas $\{e_1, e_2, \dots, e_{n+1}\}$, siis avaldub see
    vektorrkorrutis järgmise kui determinant
    \begin{align}\label{eq:n-vec-mult}
        [x_1, x_2, \dots, x_n] = \left|\begin{matrix}
          x_{11}    & x_{12}    & \dots  & x_{1n}    & e_1    \\
          x_{21}    & x_{22}    & \dots  & x_{2n}    & e_2    \\
          \vdots    & \vdots    & \ddots & \vdots    & \vdots \\
          x_{n+1,1} & x_{n+1,2} & \dots  & x_{n+1,n} & e_{n+1}
        \end{matrix}\right|,
    \end{align}
    kus $(x_{1i}, x_{2i}, \dots, x_{n+1,i})$ on vektorite
    $x_i$, $i = 1, 2, \dots, n$, koordinaadid.

    Kui me varustame ruumi $E$ nüüd $n$-aarse vektorkorrutisega
    \eqref{eq:n-vec-mult}, siis saame $(n+1)$-mõõtmelise
    reaalse kaldsümmeetrilise algebra, mida tähistame
    $\A_{n+1}$. Tänu determinandi multilineaarsusele on
    vektorkorrutis täielikult määratud baasivektorite
    korrutustabeliga. Võrdusest \eqref{eq:n-vec-mult} saame
    me baasivektoriteke järgmise korrutustabeli:
    \begin{align}\label{eq:vec-mult-table}
        [e_1, \dots, e_{i-1}, \hat{e_i}, e_{i+1}, \dots, e_{n+1}]
        = (-1)^{n+1+i} e_i,
    \end{align}
    kus $i = 1, 2, \dots, n+1$, ja $\hat{e_i}$ tähistab
    vektori $e_i$ arvutusest välja jätmist. Ülejäänud
    baasivektorite korrutised on kas nullid või kättesaadavad
    võrdusest \eqref{eq:vec-mult-table} ja
    kaldsümmeetrilisusest.

    Selle põhjal ei ole keeruline veenuda, et algebra
    $(\A_{n+1}, [\cdot, \dots, \cdot])$ on $n$-Lie algebra.
\end{naide}

\subsection{Indutseeritud \texorpdfstring{$n$}\ -Lie algebra}

See alapeatükk tugineb artiklile \cite{AKMS:2014}.

\begin{dfn}[Jälg]
    Olgu $\A$ vektorruum ning olgu $\phi \col \A^n \to \A$. Me
    ütleme, et lineaarkujutus $\tau \col \A \to K$ on
    \emph{$\phi$ jälg}, kui suvaliste $x_1, \dots, x_n \in \A$
    korral
    \begin{align*}
        \tau \left(
            \phi \left( x_1, \dots, x_n \right)
        \right) = 0.
    \end{align*}
\end{dfn}

Olgu $\phi \col \A^n \to \A$ $n$-lineaarne ja
$\tau \col \A \to K$ lineaarne kujutus. Toome mugavuse
ja selguse mõttes sisse uue kujutuse $\phi_i \col \A^n \to \A$,
$i = 1, \dots, n+1$, mis on defineeritud kui
\begin{align*}
    \phi_i\left(x_1, \dots, x_i, \dots, x_{n+1}\right) &=
    \phi \left(x_1, \dots, \hat{x_i}, \dots, x_{n+1}\right) = \\
    &= \phi \left(x_1, \dots, x_{i-1}, x_{i+1}, \dots, x_{n+1}\right),
\end{align*}
kus $\hat{x_i}$ tähistab kõrvalejäätavat elementi, see tähendab
$\phi_i(x_1, \dots, x_{n+1})$ arvutatakse elementidel
$x_1, \dots, x_{i-1}, x_{i+1}, \dots, x_{n+1}$.

Defineerime nende
kujutuste abil uue $(n+1)$-lineaarse kujutuse
$\phi_\tau \col \A^{n+1} \to \A$ valemiga
\begin{align}\label{eq:phi_tau}
    \phi_\tau \left( x_1, \dots, x_{n+1} \right) =
    \sum_{i=1}^{n+1} (-1)^{i-1} \tau(x_i)
        \phi_i(x_1, \dots, x_{n+1}),
\end{align}


Seega võttes näiteks $n = 2$ saame valemi \eqref{eq:phi_tau}
põhjal kirjutada
\[
    \phi_\tau (x_1, x_2, x_3) =
        \tau(x_1) \phi(x_2, x_3) -
        \tau(x_2) \phi(x_1, x_3) +
        \tau(x_3) \phi(x_1, x_2).
\]

Osutub, et selliselt defineeritud kujutusel $\phi_\tau$ on
mitmed head omadused.

\begin{lemma}
    Olgu $\A$ vektorruum ning $\phi \col \A^n \to \A$ $n$-lineaarne
    kaldsümmeetriline kujutus ja $\tau \col \A \to K$ lineaarne.
    Siis kujutus $\phi_\tau \col A^{n+1} \to A$ on samuti
    kaldsümmeetriline. Lisaks, kui $\tau$ on $\phi$ jälg, siis
    $\tau$ on ka $\phi_\tau$ jälg.
\end{lemma}

\begin{proof}
    Eeldame, et $\A$ on vektorruum, $\phi \col \A^n \to \A$ on
    $n$-lineaarne ning kaldsümmeetriline ja $\tau \col \A \to K$
    on lineaarne.

    Veendumaks, et $\phi_\tau$ on $(n+1)$-lineaarne ning
    kaldsümmeetriline olgu $j \in \{1, 2, \dots, n+1\}$,
    $x_1, x_2, \dots, x_{n+1}, x_j^1, x_j^2 \in \A$ ja
    $\lambda$ ning $\mu$ skalaarid. Arvestades nii
    $\phi$ kui ka $\tau$ lineaarsust märgime $\phi_\tau$
    lineaarsuseks, et
    \begin{align*}
        &\phi_\tau(
            x_1, \dots, \lambda x_j^1 + \mu x_j^2, \dots, x_{n+1}
        ) = \\
        &= \sum_{i=1}^{n+1} (-1)^{i-1} \tau(x_i) \phi_i(
             x_1, \dots, \lambda x_j^1 + \mu x_j^2, \dots, x_{n+1}
           ) = \\
        &= \sum_{\substack{i=1 \\ i \ne j}}^{n+1} (-1)^{i-1}
           \tau(x_i) \phi_i(
             x_1, \dots, \lambda x_j^1 + \mu x_j^2, \dots, x_{n+1}
           ) + \\
        &\quad\ (-1)^{j-1} \tau(\lambda x_j^1 + \mu x_j^2)
           \phi_j(
             x_1, \dots, \lambda x_j^1 + \mu x_j^2, \dots, x_{n+1}
           ) = \\
        &= \sum_{\substack{i=1 \\ i \ne j}}^{n+1}
           \lambda (-1)^{i-1} \tau(x_i) \phi_i(
             x_1, \dots, x_j^1, \dots, x_{n+1}
           ) + \\
        &\quad\ \sum_{\substack{i=1 \\ i \ne j}}^{n+1} \mu
           (-1)^{i-1} \tau(x_i)
           \phi_i( x_1, \dots, x_j^2, \dots, x_{n+1} ) + \\
        &\quad\ \lambda (-1)^{j-1} \tau(x_j^1) \phi_j(
             x_1, \dots, x_j^1, \dots, x_{n+1}) + \\
        &\quad\ \mu (-1)^{j-1} \tau(x_j^2) \phi_j(
             x_1, \dots, x_j^2, \dots, x_{n+1}) = \\
        &= \lambda \phi_\tau(x_1, \dots, x_j^1, \dots, x_{n+1}) +
           \mu \phi_\tau(x_1, \dots, x_j^2, \dots, x_{n+1}).
    \end{align*}

    Kaldsümmeetrilisus avaldub samuti vahetu arvutuse tulemusena:
    \begin{align*}
        &\phi_\tau(
          x_1, \dots, x_{j-1}, x_j, \dots, x_{n+1}
        ) = \\
        &= \sum_{\substack{i=1\\i \ne j-1, i \ne j}}^{n+1}
          (-1)^{i-1} \tau(x_i) \phi_i(
              x_1, \dots, x_{j-1}, x_j, \dots, x_{n+1}
          ) + \\
        &\quad\ (-1)^{j-1-1} \tau(x_{j-1}) \phi(
              x_1, \dots, x_{j-2}, x_j, x_{j+1}, \dots, x_{n+1}
          ) + \\
        &\quad\ (-1)^{j-1} \tau(x_j) \phi(
              x_1, \dots, x_{j-2}, x_{j-1}, x_{j+1}, \dots, x_{n+1}
          ) = \\
        &= -\sum_{\substack{i=1\\i \ne j-1, i \ne j}}^{n+1}
          (-1)^{i-1} \tau(x_i) \phi_i(
              x_1, \dots, x_j, x_{j-1}, \dots, x_{n+1}
          ) - \\
        &\quad\ (-1)^{j-1} \tau(x_{j-1}) \phi(
              x_1, \dots, x_{j-2}, x_j, x_{j+1}, \dots, x_{n+1}
          ) - \\
        &\quad\ (-1)^{j-1-1} \tau(x_j) \phi(
              x_1, \dots, x_{j-2}, x_{j-1}, x_{j+1}, \dots, x_{n+1}
          ) = \\
        &= - \phi_\tau(
              x_1, \dots, x_j, x_{j-1}, \dots, x_{n+1}
            )
    \end{align*}

    Tõestuse lõpetuseks eeldame, et $\tau$ on $\phi$ jälg ja
    näitame, et sel juhul on $\tau$ ka $\phi_\tau$ jälg.

    Et $\tau$ on $\phi$ jälg, siis iga $x_1, x_2, \dots, x_n \in \A$
    korral $\tau(\phi(x_1, x_2, \dots, x_n)) = 0$ ja seega
    \begin{align*}
        &\tau(\phi_\tau(x_1, x_2, \dots, x_{n+1})) = 
        \sum_{i=1}^{n+1} \tau \left(
            (-1)^{i-1} \tau(x_i) \phi_i(x_1, \dots, x_{n+1})
        \right) = \\
        &= \sum_{i=1}^{n+1} (-1)^{i-1} \tau(x_i) \tau \left(
            \phi(x_1, \dots, x_{i-1}, x_{i+1}, \dots, x_{n+1})
        \right) = \sum_{i=1}^{n+1} 0 = 0.
    \end{align*}
\end{proof}

Kasutades eelmist lemmat saame tõestada järgmise tähtsa teoreemi.

\begin{thm}\label{thm:n+1_lie_alg}
    Olgu $(\A, \phi)$ $n$-Lie algebra ning olgu $\tau$ lineaarkujutuse
    $\phi$ jälg. Siis $(\A, \phi_\tau)$ on $(n+1)$-Lie algebra.
\end{thm}

\begin{proof}
    Eeldame, et paar $(\A, \phi)$ omab $n$-Lie algebra struktuuri,
    see tähendab $\phi \col \A^n \to \A$ on kõikide argumentide
    järgi lineaarne, kaldsümmeetriline ja rahuldab Filippovi
    samasust, ning olgu kujutuse $\phi$ jälg $\tau \col \A \to K$.

    Lemma põhjal on siis $\phi_\tau$ oma argumentide järgi lineaarne
    ning kaldsümmeetriline. Seega peame teoreemi tõestamiseks
    näitama, et $\phi_\tau$ rahuldab Filippovi samasust
    \begin{align}\label{eq:n+1_filippov}
        \begin{split}
            &\phi_\tau(x_1, \dots, x_n,
                \phi_\tau(y_1, \dots, y_{n+1})
            ) = \\
            &= \sum_{i=1}^{n+1} \phi_\tau(
                y_1, \dots, y_{i-1},
                \phi_\tau(x_1, \dots, x_n, y_i),
                y_{i+1}, \dots, y_{n+1}
            )
        \end{split}
    \end{align}

    Olgu selleks $x_1, x_2, \dots, x_n, y_1, y_2, \dots, y_{n+1}
    \in \A$. Siis ühelt poolt
    \begin{align*}
        &\phi_\tau(x_1, \dots, x_n,
            \phi_\tau(y_1, \dots, y_{n+1})
        ) = \\
        &= \phi_\tau\left(
             x_1, \dots, x_n,
             \sum_{i=1}^{n+1} (-1)^{i-1} \tau(y_i)
             \phi_i(y_1, \dots, y_{n+1})
           \right) = \\
        &= \sum_{i=1}^{n+1} (-1)^{i-1} \tau(y_i) \phi_\tau\left(
             x_1, \dots, x_n, \phi_i(y_1, \dots, y_{n+1})
           \right) = \\
        &= \sum_{i=1}^{n+1} (-1)^{i-1} \tau(y_i)
           \sum_{j=1}^{n} (-1)^{j-1} \tau(x_j) \phi_j(
             x_1, \dots, x_n, \phi_i(y_1, \dots, y_{n+1})) + \\
        &\quad\, \sum_{i=1}^{n+1} (-1)^{i-1} \tau(y_i) (-1)^{n+1-1}
           \underbrace{
             \tau\left( \phi_i(y_1, \dots, y_{n+1}) \right)
           }_{0} \phi(x_1, \dots, x_n) = \\
        &= \sum_{i=1}^{n+1} \sum_{j=1}^{n} (-1)^{i-1} (-1)^{j-1}
           \tau(y_i) \tau(x_j) \phi_j \left(
             x_1, \dots, x_n, \phi_i(y_1, \dots, y_{n+1})
           \right) = \\
        &= \sum_{i=1}^{n+1} \sum_{j=1}^{n} (-1)^{i+j}
           \tau(y_i) \tau(x_j) \phi_j \left(
             x_1, \dots, x_n, \phi_i(y_1, \dots, y_{n+1})
           \right).
    \end{align*}
\end{proof}

Teoreemis kirjeldatud viisil saadud $(n+1)$-Lie algebrat $(A, \phi_\tau)$
nimetatakse $n$-Lie algebra $(A, \phi)$ poolt \emph{indutseeritud}
$(n+1)$-Lie algebraks.

Teoreemist~\ref{thm:n+1_lie_alg} saame teha olulise järlduse:

\begin{jar}
    Olgu $(A, \brac{\cdot}{\cdot})$ Lie algebra ning olgu antud
    $\brac{\cdot}{\cdot}$ jälg $\tau \col A \to \K$. Siis ternaarne sulg
    $[ \cdot, \cdot, \cdot ] \col A^3 \to A$, mis on defineeritud
    valemiga
    \[
        [x, y, z] = \tau(x)[y, z] + \tau(y)[z, x] + \tau(z)[x, y],
    \]
    määrab $3$-Lie algebra struktuuri $A_\tau$ vektorruumil $A$.
\end{jar}
