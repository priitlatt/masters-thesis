%!TEX root = ../../thesis.tex

\subsection{\texorpdfstring{$1|1$}{1|1}-dimensionaalsed
    \texorpdfstring{$3$}{3}-Lie superalgebrad}

Olgu meil $3$-Lie superalgebra $\g$, mille supervektorruumi dimensioon
on $1|1$ ja tema vektorruumi baas on $\{ e, f \}$, kus $e$ on paarisvektor
ning $f$ on paaritu. Siis avaldub kommutaator baasielementidel järgmiselt:
\begin{align*}
    \begin{matrix*}[l]
        &[e, e, e] = 0,\quad
        &[e, f, f] = \l e, \\
        &[e, e, f] = 0,\quad
        &[f, f, f] = \m f
    \end{matrix*}
\end{align*}
kus $l, m \in \C$ on suvalised skalaarid.

Rakendades neile kommutaatoritele nüüd meie algoritmi saame kitsendused,
mida Lie superalgebra $\g$ struktuurikonstandid rahuldama peavad:
\begin{align*}
    \left\{
        \begin{matrix*}[l]
            &2\l\m e + \l^2 e =  \l^2 e, \\
            &3\l^2 e = \l\m e, \\
            &3\m^2 f = \m^2 f,
        \end{matrix*}
    \right.
    %
    \quad \iff \quad
    %
    \left\{
        \begin{matrix*}[l]
            &2\l\m + \l^2 - \l^2 = 0, \\
            &3\l^2 - \l\m = 0, \\
            &3\m^2 - \m^2 = 0.
        \end{matrix*}
    \right.
\end{align*}
Neid kitsendusi rahuldab vaid paar
$\begin{pmatrix}\l \\ \m\end{pmatrix} = \begin{pmatrix}0 \\ 0\end{pmatrix}$,
mis tähendab, et kehtib järgmine teoreem.

\begin{thm}
    Kõik $3$-Lie superalgebraid, mille supervektorruumi dimensioon on $1|1$,
    on triviaalsed ehk Abeli Lie superalgebrad.
\end{thm}
