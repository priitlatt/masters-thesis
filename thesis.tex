\documentclass[12pt,a4paper]{article}

\usepackage[T1]{fontenc}
\usepackage[utf8]{inputenc}
\usepackage[estonian]{babel}
\usepackage{geometry}

\usepackage{amsmath}
\usepackage{amsfonts}
\usepackage{amssymb}
\usepackage{amsthm}
\usepackage{amsopn}

\usepackage{hyperref}
\usepackage{multicol}
\usepackage{verbatim}


\everymath{\displaystyle}

\author{Priit Lätt}
\title{3-Lie superalgebrad}

\theoremstyle{plain}
\newtheorem{thm}{Teoreem}[section]
\newtheorem{lau}[thm]{Lause}
\newtheorem{lemma}[thm]{Lemma}
\newtheorem{jar}[thm]{Järeldus}

\theoremstyle{definition}
\newtheorem{dfn}{Definitsioon}[section]
\newtheorem{naide}{Näide}[section]

\newcommand{\K}{\mathbb{K}}
\newcommand{\G}{\mathcal{G}}

\DeclareMathOperator{\gl}{gl}

\newcommand{\col}{\colon}
\newcommand{\arr}{\rightarrow}

\newcommand{\brac}[2]{\ensuremath{\left[ #1, #2 \right]}}
\newcommand{\nbrac}[2]{\ensuremath{\left[ #1, \dots, #2 \right]}}

\begin{document}

%!TEX root = ../thesis.tex

%%%%%%%%%%%%%%%%%%%%%%%%%%%%%%%%%%%%%%%%%%%%%
%%  Klassikaline teooria ehk Lie algebrad  %%
%%%%%%%%%%%%%%%%%%%%%%%%%%%%%%%%%%%%%%%%%%%%%

\section{Lie algebra}

Matemaatika haru, mida me täna tunneme kui \emph{Lie teooriat} kerkis esile
geomeetria ja lineaaralgebra uurimisest. Lie teooria üheks keskseks mõisteks
on \emph{Lie algebra} - vektorruum, mis on varustatud
mitteassotsiatiivse korrutamisega ehk nõndanimetatud \emph{Lie suluga}.
Lie algebrad ja nende uurimine on tihedalt seotud teise Lie teooria keskse
mõistega, milleks on \emph{Lie rühm}. Viimased on
struktuurid, mis on korraga nii algebralised rühmad kui ka topoloogilised
muutkonnad, kusjuures rühma korrutamine ja selle pöördtehe on mõlemad
pidevad. Osutub, et igale Lie rühmale saab vastavusse seada Lie algebra, kuid
üldjuhul kahjuks vastupidine väide ei kehti. Samas on võimalik näidata
pisut nõrgem tulemus: suvalise lõplikumõõtmelise
reaalsele või komplekssele Lie algebra jaoks leidub temale üheselt
vastav sidus Lie rühm.\cite{kirillov2008introduction} Just selle viimase,
nõndanimetatud \emph{Lie kolmanda teoreemi} tõttu on võimalik
Lie rühmasid vaadelda Lie algebrate kontekstis ja see teebki Lie algebrad
äärmiselt oluliseks.

Tähistagu kõikjal järgnevas $K$ nullkarakteristikaga korpust ning $V$
vektorruumi üle korpuse $K$. Ruumi kokkuhoiu ja mugavuse mõttes
kasutame edaspidi vajaduse korral summade tähistamisel
Einsteini summeerimiskokkulepet. Teisi sõnu, kui meil on indeksid $i$ ja $j$,
mis omavad väärtusi $1, \dots, n$, kus $n \in \N$, siis jätame vahel
summeerimisel summamärgi kirjutamata ning säilitame summeerimise tähistamiseks
vaid indeksid. Einsteini summeeruvuskokkulepet arvestades kehtivad näiteks
järgmised võrdused:
\begin{align*}
    x^{i} e_{i} &= \sum_{a=1}^{n} x^{i} e_{i} = 
        x^{1} e_{1} + x^{2} e_{2} + \dots + x^{n} e_{n}, \\
    \lambda{^i_j} x^{j} &= \sum_{j=1}^{n} 
        \lambda{^i_j} x^{j} = \lambda{^i_1} x^{1} + 
        \lambda{^i_2} x^{2} + 
        \dots +\lambda{^i_n} x^{n},\\
    \eta_{ij} u^{i} v^{j} &= \eta_{11} u^{1} v^{1} + 
        \eta_{12} u^{1} v^{2} + \dots + \eta_{1n} u^{1} v^{n} + 
        \eta_{21} u^{2} v^{1} + \dots + \eta_{nn} u^{n} v^{n},
\end{align*}
ja nii edasi.

Järgnevas teeme põgusa sissejuhatuse klassikalisse Lie teooriasse.

\subsection{Maatriksrühmad ja bilineaarvorm}\label{subsec:mat-ryhmad-ja-bilinvorm}

Meenutame, et \emph{lineaarkujutus}
$\phi \col V_1 \to V_2$ vektorruumist $V_1$ vektorruumi $V_2$ säilitab
vektorite liitmise ja skalaariga korrutamise, see tähendab
\[ \phi(x + y) = \phi(x) + \phi(y), \]
ning
\[ \phi(\lambda x) = \lambda \phi(x), \]
kus $x, y \in V_1$ ja $\lambda$ on skalaar. Kui vektorruumid $V_1$ ja $V_2$
langevad kokku, siis ütleme me kujutuse $\phi$ kohta \emph{lineaarteisendus}.

Algebrast on teada, et lineaarteisendusel eksisteerib pöördteisendus siis ja
ainult siis, kui ta on nii üks-ühene kui ka pealeteisendus. Kõigi vektorruumi $V$
pööratavate lineaarteisenduste rühma nimetatakse vektorruumi $V$
\emph{pööratavate lineaarteisenduste rühmaks}\footnote{Inglise keeles
\emph{general linear group}.} ja tähistatakse $\GL(V)$. Selge, et selle
rühma korrutamiseks on tavaline lineaarteisenduste kompositsioon.

Et lõplikumõõtmelise vektorruumi lineaarteisendus on pööratav parajasti siis
kui tema determinant on nullist erinev, siis rühma $\GL(V)$ kuuluvad
need ja ainult need lineaarteisendused, mille determinant pole null.
Kui vaatleme vaid lineaarteisendusi, mille determinant on üks, saame olulise
alamrühma $\SL(V)$, mida nimetatakse vektorruumi $V$
\emph{spetsiaalsete lineaarteisenduste rühmaks}.

Et igal vektorruumil leidub baas, siis võime vektorruumi $V$ jaoks
fikseerida mingi baasi. Sel juhul
saame kõik lineaarteisendused esitada maatriksitena ning nõnda võime
edaspidi lineaarteisenduste rühmade asemel rääkida \emph{maatriksrühmadest}.
Kui $\{e_1, e_2, \dots, e_n\}$ moodustab vektorruumi $V$ baasi ning
$\phi \col V \to V$ on mingi lineaarteisendus, siis talle vastav maatriks
selle baasi suhtes on $(a^i_j)$, mis on määratud valemiga
\[ \phi(e_j) = \sum_{i=1}^{n} a^i_j e_i. \]

Selge, et vaadeldes rühmi $\GL(V)$ ja $\SL(V)$ maatriksrühmadena
on rühma tehteks juba tavaline maatriksite korrutamine. Ilmselt saab
nimetatud maatriksrühmad defineerida suvalise korpuste jaoks, ja nii ka
reaal- ning kompleksarvude korral. Sellest lähtuvalt kasutatakse sageli
nullist erineva determinandiga $n \times n$ maatriksrühmade tähistuseks
$\GL(n, \R)$ või $\GL(n, \C)$, ning neid rühmi nimetame vastavalt
\emph{reaalsete pööratavate lineaarteisenduste rühmaks} ja \emph{komplekssete
pööratavate lineaarteisenduste rühmaks}. Analoogiliselt on kasutusel
tähistused $\SL(n, \R)$ ja $\SL(n, \C)$.

Rühmal $\GL(n, \C)$ on palju tuntud alamrühmi. Klassikaliseks näiteks on
$n \times n$ \emph{ortogonaalsete maatriksite rühm} $\Ort(n, \C)$, kuhu kuuluvad
ortogonaalsed maatriksid, see tähendab sellised maatriksid $A$, mille korral
$A^T = A^{-1}$. Teise näitena võib tuua unitaarsete maatriksite rühma $\U(n)$,
mille elementideks on anti-Hermite'i maatriksid $A$, mis rahuldavad tingimust
$A^\dag = \overline{A}^T = -A$. Edasi on lihtne konstrueerida saadud alamrühmade
\emph{spetsiaalsed} variandid. \emph{Spetsiaalsete komplekssete ortogonaalmaatriksite rühm} on
\[ \SO(n, \C) = \Ort(n, \C) \cap \SL(n, \C), \]
ja \emph{spetsiaalsete unitaarsete maatriksite rühmaks} on
\[ \SU(n) = \U(n) \cap \SL(n, \C). \]

\begin{dfn}
    Olgu $V$ vektorruum üle korpuse $K$. Kujutust $\blinv \col V^2 \to K$
    nimetatakse \emph{bilineaarvormiks}, kui iga $x, y, z \in V$
    ja skalaaride $\lambda, \mu \in K$ korral
    \begin{enumerate}[label=\roman*.]
        \item $(\lambda x + \mu y, z) = \lambda (x, z) + \mu (y, z)$,
        \item $(x, \lambda y + \mu z) = \lambda (x, y) + \mu (x, z)$.
    \end{enumerate}
\end{dfn}

Kui vetorruumis $V$ on antud baas $\{e_1, e_2, \dots, e_n\}$, siis saab
bilineaarvormi $\blinv \col V^2 \to K$ esitada talle vastava maatriksi
$B = (b_{ij})$ abil, kus $b_{ij} = (e_i, e_j)$. Tõepoolest, kui
meil on antud vektorid $x = \sum_i \lambda^i e_i$ ja $y = \sum_j \mu^j e_j$,
siis kasutades $\blinv$ lineaarsust mõlema muutuja järgi võime arvutada
\[ (x, y) = \sum_{i, j} b_{ij} \lambda^i \mu^j. \]

Me ütleme, et bilineaarvorm $\blinv \col V^2 \to K$ on \emph{sümmeetriline} kui
iga $x, y \in V$ korral $(x, y) = (y, x)$. Selge, et sümmeetrilise
bilineaarvormi maatriksi $B$ korral kehtib võrdus $B = B^T$. Vormi $\blinv$
nimetatakse \emph{kaldsümmeetriliseks} kui iga $x, y \in V$ korral kehtib
võrdus $(x, y) = - (y, x)$. Lihtne on veenduda, et kaldsümmeetrilise
bilineaarvormi korral rahuldab talle vastav maatriks $B$ seost
$B^T = -B$.

\begin{dfn}
    Olgu $V$ vektorruum kus on fikseeritud mingi baas, olgu $\phi$
    vektorruumi $V$ lineaarteisendus ning olgu $A = (a^i_j)$ on
    lineaarteisenduse $\phi$ maatriks fikseeritud baasi suhtes.
    Lineaarteisenduse $\phi$ \emph{jäljeks} nimetatakse kujutust
    $\Tr_V \col \GL(V) \to K$, kus
    \[ \Tr_V(A) = \sum_i a^i_i. \]
\end{dfn}

Juhul kui maatriksi $A$ korral $\Tr_V A = 0$, siis ütleme, et
maatriks $A$ on \emph{jäljeta}.

\begin{naide}
    On hästi teada, et vektorruumi $V$ kõigi lineaarteisenduste hulk $\Lin V$
    on ise ka vektorruum, kusjuures kui vektorruumi $V$ dimensioon
    on $\dim(V) = n$, siis ruumi $\Lin V$ dimendsioon on
    $\dim \left(\Lin V\right) = n^2$. Kasutades jälge $\Tr_V$ saame
    defineerida bilineaarvormi
    $\blinv \col \Lin V \times \Lin V \to K$ järgmiselt:
    \[ (A, B) = \Tr_V(AB), \]
    kus $A$ ja $B$ on maatriksid, mis vastavad vektorruumi $\Lin V$
    teisendustele mingi baasi suhtes. Selge, et selliselt defineeritud
    bilineaarvorm sümmeetriline.
\end{naide}

Kasutades bilineaarvormi sümmeetrilisuse või kaldsümmeetrilisuse mõistet
saame sisse tuua \emph{ortogonaalsuse} mõiste. Me ütleme, et vektorid $x$ ja $y$
on bilineaarvormi $\blinv$ suhtes ortogonaalsed, kui $(x, y) = 0$. Selge, et
ortogonaalsuse tingimus ise on sümmeetriline, see tähendab kui $x$ on
ortogonaalne vektoriga $y$, siis kehtib ka vastupidine, $y$ on ortogonaalne
vektoriga $x$. Kui vektor $x \neq 0$ on iseenesega ortogonaalne, see tähendab
$(x, x) = 0$, siis nimetatakse vektorit $x$ \emph{isotroopseks}. Selge, et
Eukleidilises geomeetrias selliseid vektoreid ei leidu, kuid üldisemates
situatsioonides esinevad nad küllaltki sageli, näiteks
\emph{Minkowski aegruumis}.

Edasises vaatleme ortogonaaseid ja sümplektilisi rühmi ning selleks
nõuame, et vaatluse all olevad bilineaarvormid oleksid mittesingulaarsed
ehk regulaarsed, see tähendab kui $(x, y) = 0$ iga $y \in V$ korral, siis
järelikult $x = 0$. Osutub, et bilineaarvorm $\blinv$ on regulaarne parajasti
siis, kui temale vastav maatiks $B = (b^i_j)$ on pööratav, mis tähendab,
et $\det B \neq 0$.

\begin{dfn}
    Me ütleme, et lineaarne operaator $\phi$ on \emph{ortogonaalne}
    regulaarse sümmeetrilise bilineaarvormi $\blinv$ suhtes, kui
    \[ (\phi(x), \phi(y)) = (x, y) \]
    kõikide $x$ ja $y$ korral vektorruumist $V$.
\end{dfn}

Kui $x$ on ortogonaalse lineaarse operaatori $\phi$ tuumast, siis
kehtib $\phi(x) = 0$. Viimane aga tähendab, et iga $y \in V$ korral
$(x, y) = (\phi(x), \phi(y)) = (0, \phi(y)) = 0$. Kokkuvõttes,
et $\blinv$ on regulaarne, siis järelikult $x = 0$ ja $\phi$ on üks-ühene.
Kui nüüd veel $V$ on lõplikumõõtmeline, siis peab $\phi$ olema pööratav.
Seda arutelu silmas pidades võime öelda, et ortogonaalsed lineaarsed
operaatorid moodustavad rühma, mida me nimetame \emph{ortogonaalsete
lineaarteisenduste rühmaks} bilineaarvormi $\blinv$ suhtes.
Võttes tarvitusele vektorruumi $V$ baasi saame konstrueerida ka
\emph{ortogonaalsete maatriksite rühma}, mida tähistatakse komplekssel juhul
kui $\Ort(n, \C)$, kus $n \in \N$ märgib, et tegu on $n \times n$ maatriksitega.

Sümplektiliste teisenduste tarvis tuleb vaadelda kaldsümmeetrilisi
bilineaarvorme.

\begin{dfn}
    Me ütleme, et lineaarne operaator $\phi$ on \emph{sümplektiline}
    regulaarse kaldsümmeetrilise bilineaarvormi $\blinv$ suhtes, kui
    \[ (\phi(x), \phi(y)) = (x, y) \]
    kõikide $x$ ja $y$ korral vektorruumist $V$.
\end{dfn}

Märgime, et sümplektilised lineaarteisendused leiduvad ainult sellistes
vektorruumides, mille dimensioon on paarisarvuline, see tähendab
$\dim V = 2n$, kus $n \in \N$. Sümplektilised teisendused moodustavad
\emph{sümplektiliste rühma}, mida tähistatakse kompleksel juhul
$\Sp(n, \C)$. Reaalsete sümplektiliste
teisenduste rühma saame kui vaatleme ühisosa rühmaga $\GL(2n, \R)$:
\[ \Sp(n, \R) = \Sp(n, \C) \cap \GL(2n, \R). \]


\subsection{Eksponentsiaalkujutus}

Kõikide seni käsitluse all olnud maatriksrühmade esindajad peavad
vastavatesse rühmadesse kuulumiseks rahuldama mingeid algebralisi tingimusi.
Need tingimused võib kirja panna maatriksite elementide kaudu, mille
tulemusel saaame me mittelineaarseid võrrandeid, mis määravad rühma kuulumise.
Osutub, et need tingimused on võimalik asendada mingi hulga ekvivalentsete
lineaarsete võrranditega ja selline üleminek mittelineaarselt süsteemilt
lineaarsele ongi võtmetähtsusega idee üleminekul Lie rühmadest Lie
algebratele.\cite{johan1989survey}

Klassikaliseks viisiks kuidas sellist üleminekut realiseeritakse on
\emph{eksponentsiaalkujutuse} kasutuselevõtt. Nagu nimigi viitab, on
tegu analüüsist tuttava kujutuse üldistusega. Et meil oli siiani tegemist
vaid maatriksrühmadega, siis läheme siin ka edasi vaid eksponentsiaalkujutuse
ühe tähtsa erijuhuga, \emph{maatrikseksponentsiaaliga}, kuid olgu öeldud,
et järgnevad väited kehtivad tegelikult ka üldisemas seades,
nagu võib näha raamatus \cite{kirillov2008introduction}.

Olgu $A$ mingi $n \times n$ maatriks, $k \in \N$ ning olgu $I$
ühikmaatriksit. Tähistame $A^0 = I$ ning
$A^k = \underbrace{A \cdot A \cdot \ldots \cdot A}_{k\text{-korda}}$.

\begin{dfn}
    Olgu $X$ reaalne või kompleksne $n \times n$ maatriks. Maatriksi
    $X$ \emph{eksponendiks}, mida tähistatakse $e^X$ või $\exp X$, nimetatakse
    astmerida
    \begin{align}\label{eq:mat-exp}
        e^X = \sum_{k=0}^{\infty} \frac{X^k}{k!}.
    \end{align}
\end{dfn}

Ilmselt tuleks definitsiooni korrektsuses veendumaks näidata, et suvalise
maatriksi $X$ korral rida \eqref{eq:mat-exp} koondub. Selleks meenutame,
et $n \times n$ maatriksi $X = (X_{ij})$ normi arvutatakse valemi
\begin{align}\label{eq:mat-norm}
    \| X \| = \left( \sum_{i,j=1}^n |X_{ij}|^2 \right)^{\frac{1}{2}}
\end{align}
järgi. Arvestades, et $\| XY \| \le \|X\| \|Y\|$, siis $\|X^k\| \le \|X\|^k$.
Rakendades nüüd normi \eqref{eq:mat-norm} rea \eqref{eq:mat-exp} liikmetele
saame
\begin{align*}
    \sum_{k=0}^\infty \left\lVert \frac{X^k}{k!} \right\rVert \le
    \sum_{k=0}^\infty \frac{\|X\|^k}{k!} = e^{\|X\|} < \infty,
\end{align*}
mis tähendab, et rida \eqref{eq:mat-exp} koondub absoluutselt ja seega
ta ka koondub. Märkamaks, et $e^X$ on pidev funktsioon märgime esiteks,
et $X^k$ on argumendi $X$ suhtes pidev funktsioon ja seega on rea
\eqref{eq:mat-exp} osasummad pidevad. Teisalt paneme tähele, et
\eqref{eq:mat-exp} koondub ühtlaselt hulkadel, mis on kujul
$\{ \|X\| \le R \}$, ja seega on rida kokkuvõttes pidev.

Seega on maatrikseksponentsiaal korrektselt defineeritud ning ka pidev.
Järgmises lauses on toodud rida eksponentsiaalkujutuse põhilisi omadusi,
mille võrdlemisi lihtsad tõestused võib huvi korral võib leida näiteks
teosest \cite{hall2003lie}.

\begin{lau}\label{lau:mat-exp-om}
    Olgu $X$ ja $Y$ suvalised $n \times n$ maatriksid. Siis kehtivad järgmised
    väited:
    \begin{enumerate}[label=\arabic*)]
        \item\label{om:mat-exp-1} $e^0 = I$,
        \item\label{om:mat-exp-2} $\left(e^X\right)^T = e^{X^T}$,
        \item\label{om:mat-exp-3} $e^X$ on pööratav ning kehtib
            $\left(e^X\right)^{-1} = e^{X^{-1}}$,
        \item\label{om:mat-exp-4} $e^{(\lambda + \mu)X} = e^{\lambda X} e^{\mu X}$ suvaliste
            $\lambda, \mu \in \C$ korral,
        \item\label{om:mat-exp-5} kui $XY = YX$, siis $e^{X+Y} = e^X e^Y = e^Y e^X$,
        \item\label{om:mat-exp-6}  kui $C$ on pööratav, siis $e^{CXC^{-1}} = C e^X C^{-1}$,
        \item\label{om:mat-exp-7} $\det e^X = e^{\Tr_V X}$.
    \end{enumerate}
\end{lau}

Ostutub, et rühma $\GL(n, \C)$ ühikelemendi mingis ümbruses on võimalik
suvaline maatriks esitada kujul $e^A$, kus $A$ on mingi $n \times n$
maatriks. Rühma $\GL(n, \R)$ korral on maatriks $A$ reaalne. Oluline on
tähele panna, et vaadeldes rühma $\GL(n, \R)$, ei ole eksponentfunktsiooni
kujutis terve rühm. Selles veenumiseks piisab võtta $n = 1$ ning näha, et
$\exp \left(\GL(1, \R)\right) = \R^+$, ehk kujutiseks on reaaltelje positiivne
osa, samas kui $\GL(1, \R) = \R \setminus {0}$ ehk reaaltelg ilma nullpunktita.

Niisiis eksponentkujutust kasutades on oht kaotada rühma globaalne struktuur,
samas kui lokaalne struktuur säilib.

Et maatriksi $A$ korral kuuluks maatriks $e^A$ mõnda puntis
\ref{subsec:mat-ryhmad-ja-bilinvorm}. \nameref{subsec:mat-ryhmad-ja-bilinvorm} nimetatud rühma tuleb maatriksile
$A$ seada mingid lineaarsed kitsendused. Näiteks spetsiaalse lineaarse rühma
$\SL(n)$ korral võime mittelineaarse tingimuse $e^A$ determinandi kohta
asendada lineaarse tingimusega maatriksi $A$ jälje kohta kasutades lause
\ref{lau:mat-exp-om} punkti \ref{om:mat-exp-7}. Nii on näiteks
$\det e^A = 1$ parajasti siis, kui $\Tr_V A = 0$.

Kokkuvõttes nägime, et eksponentsiaalkujutuse abil on võimalik asendada
klassikalised maatriksrühmad maatrikshulkadega, millele on seatud
teatud lineaarsed kitsendused. Selge, et need hulgad on kinnised
lineaarkombinatsioonide suhtes ja nii võib neid vaadelda kui vektorruume.
Tavalise maatriksite korrutamise osas kahjuks kinnisus säilida ei
pruugi. Samas kui meil on $n \times n$ maatriksid $A$ ja $B$, mis on vastavalt
kas kaldsümmeetrilised, rahuldavad anti-Hermite'i tingimust või neil
puudub jälg, siis maatriksil $C = AB - BA$ on samuti selline omadus.
Niisiis saadud maatrikshulgad ei moodusta ainuüksi vektorruumi,
vaid on kinnised ka teatud binaarse tehte suhtes.

\subsection{Lie algebra definitsioon}

Enne kui Lie algebra definitsiooni anname tuletame meelde, et \emph{algebraks}
üle korpuse $K$ nimetatakse vektorruumi $V$ üle korpuse $K$, millel on
defineeritud bilineaarne korrutamine $V \times V \to V$. Kui algebra
tehe rahuldab assotsiatiivsuse tingimust, siis nimetatakse seda algebrat
assotriatiivsekt ning vastasel korral mitteassotsiatiivseks. Nii on näiteks
vektorruumi $V$ lineaarteisenduste vektorruum $\Lin V$ assotsiatiivne algebra,
mille tehteks on teisenduste kompositsioon: $f \circ g$. Samas võime
vektorruumi $\Lin V$ varustada ka teistsuguse korrutamisega ning saada uue
algebralise struktuuri, kui võtame tehteks näiteks $f \circ g - g \circ f$.
Üldiselt selline korrutamine aga enam kommutatiivne ei ole.

\begin{dfn}
    Algebrat $\g$ üle korpuse $K$ nimetatakse \emph{Lie algebraks}, kui
    tema korrutamine $[\cdot, \cdot] \col V \times V \to V$ rahuldab
    kõikide $x, y, z \in \g$ tingimusi
    \begin{gather}
        [x, x] = 0,\label{id:bracket-antisymm} \\
        \brac{x}{\brac{y}{z}} + \brac{y}{\brac{z}{x}} +
            \brac{z}{\brac{x}{y}} = 0.\label{id:jacobi}
    \end{gather}
\end{dfn}

Me ütleme definitsioonis toodud korrutise $[x, y]$ kohta elementide $x$ ja
$y$ \emph{Lie sulg}, definitsioonis toodud samasust \eqref{id:jacobi}
nimetatakse \emph{Jacobi samasuseks}. Piltlikult öeldes mõõdab Lie
sulg algebra elementide mittekommuteeruvust ja seda asjaolu kirjeldavat
võrdust \eqref{id:bracket-antisymm} võime kirja panna ka kujul
\begin{equation}\label{eq:bracket-antisymm}
    [x, y] = -[y, x].
\end{equation}
Tõepoolest, \eqref{id:bracket-antisymm} järgi kehtib $[x+y, x+y] = 0$,
millest saame bilineaarsuse abil $[x, x] + [x, y] + [y, x] + [y, y] = 0$,
ehk kehtibki $[x, y] = -[y, x]$.

Rakendades võrdust \eqref{eq:bracket-antisymm} saame Jacobi samasuse
kirjutada kui
\begin{equation}
    \brac{x}{\brac{y}{z}} = \brac{\brac{x}{y}}{z} +
            \brac{y}{\brac{x}{z}}.
\end{equation}
%!TEX root = ../thesis.tex

%%%%%%%%%%%%%%%%%%%%%%%%%%%%%%%%%%%
%%  Indutseeritud n-Lie algebra  %%
%%%%%%%%%%%%%%%%%%%%%%%%%%%%%%%%%%%

\section{Indutseeritud \texorpdfstring{$n$}\ -Lie algebra}

See peatükk tugineb artiklile \cite{AKMS:2014}

Edasises eeldame, et kõik vektoruumid on üle vaadeldud üle
$0$-karakteristikaga korpuse $\K$.

\begin{dfn}[Lie algebra]\label{def:lie_algebra}
    Vektorruumi $A$ nimetatakse \emph{Lie algebraks}, kui on
    määratud bilineaarvorm
    $\brac{\cdot}{\cdot} \colon A \times A \arr A$, mis
    suvaliste $x, y, z \in A$ korral rahuldab tingimusi
    \begin{itemize}
        \item $\brac{x}{y} = -\brac{y}{x}$,
        \item $\brac{x}{\brac{y}{z}} + 
               \brac{z}{\brac{x}{y}} +
               \brac{y}{\brac{z}{x}} = 0$.
    \end{itemize}
\end{dfn}

Bilineaarvormi $\brac{\cdot}{\cdot}$ Lie algebra definistioonis
nimetatakse selle Lie algebra suluks. Edaspidi tähistame konkreetsuse
mõttes sageli Lie suluga $\brac{\cdot}{\cdot}$ varustatud vektorruumi
$A$ paarina $(A, \brac{\cdot}{\cdot})$.

\begin{dfn}[$n$-Lie algebra]
    Vektorruumi $A$ nimetatakse \emph{$n$-Lie algebraks}, kui on
    määratud $n$-lineaarne kaldsümmeetriline kujutus
    $\nbrac{\cdot}{\cdot} \colon A^n \times A \arr A$, mis
    suvaliste
    \[ x_1, \dots, x_{n-1}, y_1, \dots, y_n \in A \]
    korral rahuldab tingimust
    \[
        \left[ x_1, \dots, x_{n-1}, \nbrac{y_1}{y_n} \right] =
        \sum_{i=1}^n \left[
            y_1, \dots, \left[ x_1, \dots, x_{n-1}, y_i \right], \dots, y_n
        \right].
    \]
\end{dfn}

\begin{dfn}[Jälg]
    Olgu $A$ vektorruum ning olgu $\phi \col A^n \arr A$. Me
    ütleme, et lineaarkujutus $\tau \col A \arr \K$ on
    \emph{$\phi$-jälg}, kui suvaliste $x_1, \dots, x_n \in A$ korral
    $\tau \left( \phi \left( x_1, \dots, x_n \right) \right) = 0$.
\end{dfn}

Olgu $\phi \col A^n \arr A$ $n$-lineaarne ja
$\tau \col A \arr \K$ lineaarne kujutus. Defineerime nende
kujutuste abil uue $(n+1)$-lineaarse kujutuse
$\phi_\tau \col A^{n+1} \arr A$ valemiga
\begin{align}\label{eq:phi_tau}
    \phi_\tau \left( x_1, \dots, x_{n+1} \right) =
    \sum_{i=1}^{n+1} (-1)^{i-1} \tau(x_i)
        \phi(x_1, \dots, \hat{x_i}, \dots, x_{n+1}),
\end{align}
kus $\hat{x_i}$ tähistab kõrvalejäätavat elementi, see tähendab
$\phi(x_1, \dots, \hat{x_i}, \dots, x_{n+1})$ arvutatakse elementidel
$x_1, \dots, x_{i-1}, x_{i+1}, \dots, x_{n+1}$.

Rikastame defineeritud kujutust ühe näitega. Võttes $n = 2$ saame
valemi~\ref{eq:phi_tau} põhjal kirjutada
\[
    \phi_\tau (x_1, x_2, x_3) =
        \tau(x_1) \phi(x_2, x_3) -
        \tau(x_2) \phi(x_1, x_3) +
        \tau(x_3) \phi(x_1, x_2).
\]

Edasises toome ära mõningad kujutuse $\phi_\tau$ tähtsamad omadused.

\begin{lemma}
    Olgu $A$ vektorruum ning $\phi \col A^n \arr A$ $n$-lineaarne kaldsümmeetriline kujutus
    ja $\tau \col A \arr \K$ lineaarne. Siis kujutus
    $\phi_\tau \col A^{n+1} \arr A$ on samuti kaldsümmeetriline. Lisaks,
    kui $\tau$ on $\phi$-jälg, siis $\tau$ on ka $\phi_\tau$-jälg.
\end{lemma}

\begin{thm}\label{thm:n+1_lie_alg}
    Olgu $(A, \phi)$ $n$-Lie algebra ning olgu $\tau$ lineaarkujutuse
    $\phi$-jälg. Siis $(A, \phi_\tau)$ on $(n+1)$-Lie algebra.
\end{thm}

Teoreemis kirjeldatud viisil saadud $(n+1)$-Lie algebrat $(A, \phi_\tau)$
nimetatakse $n$-Lie algebra $(A, \phi)$ poolt \emph{indutseeritud}
$(n+1)$-Lie algebraks.

Teoreemist~\ref{thm:n+1_lie_alg} saame teha olulise järlduse:

\begin{jar}
    Olgu $(A, \brac{\cdot}{\cdot})$ Lie algebra ning olgu antud
    $\brac{\cdot}{\cdot}$ jälg $\tau \col A \arr \K$. Siis ternaarne sulg
    $[ \cdot, \cdot, \cdot ] \col A^3 \arr A$, mis on defineeritud
    valemiga
    \[
        [x, y, z] = \tau(x)[y, z] + \tau(y)[z, x] + \tau(z)[x, y],
    \]
    määrab $3$-Lie algebra struktuuri $A_\tau$ vektorruumil $A$.
\end{jar}

\input{sections/3_lie_superalgebra.tex}
%!TEX root = ../thesis.tex

%%%%%%%%%%%%%%%%%%%%%%%%%
%% n-Lie superalgebra  %%
%%%%%%%%%%%%%%%%%%%%%%%%%

\section{\texorpdfstring{$n$}\ -Lie superalgebra}

Järgnevas toome sisse \emph{supermatemaatika} põhimõsted ning defineerime
nende abil \emph{Lie superalgebra}. Edasi ühendame Lie superalgebra
konstruktsiooni Filippovi $n$-Lie algebraga, ning konstrueerime
\emph{$n$-Lie superalgebra}.

\subsection{Lie superalgebra}

Eelmises peatükis nägime, kuidas Filippov leidis viisi Lie algebra
üldistamiseks kasutades binaarse operatsiooni asemel $n$-aarset
operatsiooni, kuid jättes aluseks oleva algebra samaks. Teine viis
Lie algebra mõistet üldistada on vaadelda algebra asemel \emph{superalgebrat}.
Tuletame selleks kõigepealt meelde mõned baasdefinitsioonid.

\begin{dfn}
    Vektorruumi $\V$ nimetatakse \emph{$\Z_2$-gradueeritud vektorruumiks}
    ehk \emph{supervektorruumiks}, kui ta on esitatav vektorruumide
    otsesummana $\V = \V_{\overline{0}} \oplus \V_{\overline{1}}$.
    Vektorruumi $\V_{\overline{0}}$ elemente nimetatakse seejuures
    \emph{paarisvektoriteks} ja $\V_{\overline{1}}$ elemente
    \emph{paarituteks} vektoriteks.
\end{dfn}

Kui supervektorruumi element $x \in \V$ korral
$x \in \V_{\overline{0}}$ või $x \in \V_{\overline{1}}$, siis ütleme, et
vektor $x$ on \emph{homogeenne}. Homogeensete vektorite korral on
otstarbekas vaadelda kujutust $|\cdot| \col \V \to \Z_2$, mis on defineeritud
võrdusega
\begin{align*}
    |x| = \begin{cases}
        \overline{0}, &\text{kui}\ x \in \V_{\overline{0}}, \\
        \overline{1}, &\text{kui}\ x \in \V_{\overline{1}}.
    \end{cases}
\end{align*}
Homogeense elemendi $x$ korral nimetame suurust $|x| \in \Z_2$ tema
paarsuseks. Märgime, et kasutades edaspidises kirjutist $|x|$ eeldame
vaikimisi, et element $x \in \V$ on homogeenne.

Supervektorruumi $\V = \V_{\overline{0}} \oplus \V_{\overline{1}}$
\emph{alamruumiks} nimetatakse supervektorruumi
$\W = \W_{\overline{0}} \oplus \W_{\overline{1}} \subseteq \V$, kui
vastavate otseliidetavate gradueeringud ühtivad, see tähendab
$\W_i \subset \V_i$, kus $i \in \Z_2$.

\begin{dfn}
    Olgu supervektorruumil $\A$ antud bilineaarne algebraline tehe
    $\phi \col \A \times \A \to \A$. Supervektorruumi $\A$ nimetatakse
    \emph{superalgebraks}, kui tehe $\phi$ rahuldab suvaliste homogeensete
    vektorite $x, y \in \A$ korral tingimust
    \begin{align}\label{def:superalgebra-parity}
        \left|\phi(x, y)\right| = |x| + |y|.
    \end{align}
\end{dfn}

Paneme tähele, et superalgebra definitsioonis nõutav kujutus
$\phi \col \A \times \A \to \A$ nõuab taustal, et supervektorruum
$\A$ oleks tegelikult algebra. Supervektorruumi ühikelement ja
assotsiatiivsus defineeritakse tavapärasel viisil.

% TODO näide naide:lie-algebra-konstrueerimine


\subsection{\texorpdfstring{$n$}\ -Lie superalgebra}

See peatükk tugineb artiklile \cite{Abramov:2014}

Järgnevas eeldame, et meil on antud supervektorruum
ehk supervektorruum $\G = \G_{\overline{0}} \oplus \G_{\overline{1}}$
ning $n$-lineaarne kujutus $\phi \col \G^n \to \G$, mis rahuldab
tingimusi
\begin{itemize}
    \item $| \phi(x_1, \dots, x_n) | = \sum_{i=1}^n |x_i|$,
    \item $ \phi \left( x_1, \dots, x_i, x_{i+1}, \dots, x_n \right) =
            -(-1)^{ |x_i| |x_{i+1}| } \phi \left(
                x_1, \dots, x_{i+1}, x_i, \dots, x_n \right), $
\end{itemize}
kus $|x| \in \left\{ \overline{0}, \overline{1} \right\}$
tähistab elemendi $x$ paartust. Samuti eeldame, et $S \col \G \to \K$
on lineaarne kujutus, mis rahuldab
\begin{itemize}
    \item $S \left( \phi \left( x_1, \dots, x_n \right) \right) = 0$,
    \item $S(x) = 0$ iga $x \in \G_{\overline{1}}$.
\end{itemize}

Selge, et siin sisse toodud kujutused $\phi$ ja $S$ on eelnevas
kirjeldatu analoogid supervektorruumis. Seejuures kujutust $S \col \G \to \K$
nimetatakse \emph{superjäljeks}.

Kasutades kujutusi $\phi$ ja $S$ defineerime analoogiliselt
vektorruumide situatsioonile, kuid nüüd juba supervektorruumi iseärasusi
arvesse võttes, see tähendab paarsusi arvestades, uue kujutuse
$\phi_S \col \G^{n+1} \to G$ valemiga
\[
    \phi_S (x_1, \dots, x_{n+1}) =
    \sum_{i=1}^{n+1} (-1)^{i-1}(-1)^{|x_i| \sum_{j=1}^{i-1} |x_j| }
        S(x_i) \phi \left(
            x_1, \dots, \hat{x_i}, \dots, x_{n+1}
        \right).
\]

Saadud kujutuse tähtsamad omadused võtab kokku järgmine oluline lemma:

\begin{lemma}
    $(n+1)$-lineaarne kujutus $\phi_S \col \G^{n+1} \to \G$
    rahuldab tingimusi
    \begin{enumerate}
        \item $ | \phi_S \left(x_1, \dots, x_{n+1} \right) | =
               \sum_{i=1}^{n+1} |x_i| $,
        \item $ \phi_S \left(x_1, \dots, x_i, x_{i+1}, \dots, x_{n+1} \right) =
               -(-1)^{|x_i| |x_{i+1}|} \phi_S \left(
                    x_1, \dots, x_{i+1}, x_i, \dots, x_{n+1}
                \right) $,
        \item $S \left( \phi_S \left( x_1, \dots, x_{n+1} \right) \right)$.
    \end{enumerate}
\end{lemma}

Üldistame nüüd definitsiooni~\ref{def:lie-algebra} supervektorruumi jaoks
ning defineerime \emph{$n$-Lie superalgebra}.

\begin{dfn}[$n$-Lie superalgebra]
    Olgu $\G = \G_{\overline{0}} \oplus \G_{\overline{1}}$ 
    supervektorruum. Me ütleme, et $\G$ on
    \emph{$n$-Lie superalgebra}, kui $\G$ on varustatud
    gradueeritud $n$-Lie suluga $\nbrac{\cdot}{\cdot} \col \G^n \to \G$,
    mis rahuldab tingimusi
    \begin{enumerate}
        \item $\left| \nbrac{x_1}{x_n} \right| = \sum_{i=1}^n |x_i| $,
        \item $\left[ x_1, \dots, x_i, x_{i+1}, \dots, x_n \right] =
            -(-1)^{|x_i| |x_{i+1}|} \left[
                x_1, \dots, x_{i+1}, x_i, \dots, x_n
            \right]$,
        \item $\left[ y_1, \dots, y_{n-1}, \nbrac{x_1}{x_n} \right] = $ \\
            $ = \sum_{i=1}^n (-1)^{\tau_x (i-1) \tau_y (n-1)}
            \left[
                x_1, \dots, x_{i-1},
                \left[ y_1, \dots, y_{n-1}, x_i \right],
                x_{i+1}, \dots, x_n
            \right] $,
    \end{enumerate}
    kus $x = (x_1, \dots, x_n)$ ja $y = (y_1, \dots, y_{n-1})$ ning
    $\tau_x (k) = \sum_{j=1}^{k-1} |x_i|$.
\end{dfn}

Võttes arvesse $n$-Lie superalgebra definitsiooni saame sõnastada 
teoreemi \ref{thm:n+1_lie_alg} superanaloogi järgmiselt:

\begin{thm}
    Olgu $\G = \G_{\overline{0}} \oplus \G_{\overline{1}}$
    $n$-Lie superalgebra suluga $\nbrac{\cdot}{\cdot} \col \G^n \to \G$,
    ning $V$ lõplikumõõtmeline vektorruum ja olgu
    antud $\G$ esitus $\phi \col \G \to \gl V$. Defineerides
    $\nbrac{\cdot}{\cdot} \col \G^{n+1} \to \G$ valemiga
    \[
        \nbrac{x_1}{x_{n+1}} = \sum_{i=1}{n+1}
        (-1)^{i-1} (-1)^{|x_i| \tau_x (i-1)} S(\phi(x_i))
        \left[ x_1, \dots, \hat{x_i}, \dots, x_{n+1} \right],
    \]
    on supervektorruum $\G$, varustatuna suluga
    $\nbrac{\cdot}{\cdot} \col \G^{n+1} \to \G$ $(n+1)$-Lie
    superalgebra.
\end{thm}

%!TEX root = ../thesis.tex

%%%%%%%%%%%%%%%%%%%%%%%
%% Klassifikatsioon  %%
%%%%%%%%%%%%%%%%%%%%%%%

\section{Madaladimensionaalsete \texorpdfstring{$n$}\ -Lie superalgebrate klassifikatsioon}

%!TEX root = ../thesis.tex

\subsection{\texorpdfstring{$(2, 1)\ 3$}\ -Lie superalgebrate klassifikatsioon}

Olgu meil antud $(2, 1)$ 3-Lie superalgebra, millel on fikseeritud baas
$\left\{ e_1, e_2, f_1 \right\}$, kus $e_1$ ja $e_2$ on paaris- ja
$f_1$ on paaritu baasivektor.

Siis avaldub kommutaator baasielementidel järgmiselt:

\begin{align}\label{samasused:2-1}
    \begin{split}
        & \left[e_1, e_1, e_1\right] = 0, \\
        & \left[e_1, e_1, e_2\right] = 0, \\
        & \left[e_1, e_1, f_1\right] = 0, \\
        & \left[e_1, e_2, e_2\right] = 0, \\
        & \left[e_1, e_2, f_1\right] = m_1 \cdot f_1, \\
        & \left[e_1, f_1, f_1\right] = l_1 \cdot e_1 + l_2 \cdot e_2, \\
        & \left[e_2, e_2, e_2\right] = 0, \\
        & \left[e_2, e_2, f_1\right] = 0, \\
        & \left[e_2, f_1, f_1\right] = l_3 \cdot e_1 + l_4 \cdot e_2, \\
        & \left[f_1, f_1, f_1\right] = m_2 \cdot f_1, \\
    \end{split}
\end{align}
kus $m_1, m_2, l_1, l_2, l_3, l_4 \in \K$ on mingid konstandid.

Rakendades nullist erinevatele kommutaatoritele Filippovi samasuse analoogi
supervektorruumis, saame järgmised võrrandid:
\begin{enumerate}
    \item $-m_1 \cdot m_1 + m_1 \cdot m_1 = 0$,
    \item $l_1 \cdot m_1 - l_1 \cdot m_1 = 0$,
    \item $l_2 \cdot m_1 - l_2 \cdot m_1 = 0$,
    \item $l_2 \cdot m_1 - l_2 \cdot m_1 = 0$,
    \item $m_1 \cdot m_1 - m_1 \cdot m_1 = 0$,
    \item $m_1 \cdot m_1 - m_1 \cdot m_1 = 0$,
    \item $l_1 \cdot m_1 - l_1 \cdot m_1 = 0$,
    \item $l_2 \cdot m_1 - l_2 \cdot m_1 = 0$,
    \item $l_3 \cdot m_1 + l_3 \cdot m_1 = 0$,
    \item $l_4 \cdot m_1 + l_4 \cdot m_1 = 0$,
    \item $m_1 \cdot m_2 - l_1 \cdot m_1 - l_4 \cdot m_1 - m_1 \cdot m_2 = 0$,
    \item $-l_1 \cdot m_1 - l_1 \cdot m_1 = 0$,
    \item $-l_2 \cdot m_1 - l_2 \cdot m_1 = 0$,
    \item $-l_2 \cdot m_1 - l_2 \cdot m_1 + l_2 \cdot m_1 = 0$,
    \item $l_1 \cdot m_1 - m_1 \cdot m_2 - l_4 \cdot m_1 + l_4 \cdot m_1 = 0$,
    \item $l_1 \cdot l_1 + l_2 \cdot l_3 - l_1 \cdot l_1 - l_2 \cdot l_3 - l_1 \cdot m_2 - l_1 \cdot m_2 = 0$,
    \item $l_1 \cdot l_2 + l_2 \cdot l_4 - l_1 \cdot l_2 - l_2 \cdot l_4 - l_2 \cdot m_2 - l_2 \cdot m_2 = 0$,
    \item $l_3 \cdot m_1 - l_3 \cdot m_1 = 0$,
    \item $l_4 \cdot m_1 - l_4 \cdot m_1 = 0$,
    \item $-l_3 \cdot m_1 + l_3 \cdot m_1 = 0$,
    \item $-l_3 \cdot m_1 - l_3 \cdot m_1 = 0$,
    \item $-l_4 \cdot m_1 - l_4 \cdot m_1 = 0$,
    \item $-l_4 \cdot m_1 - m_1 \cdot m_2 + l_1 \cdot m_1 - l_1 \cdot m_1 = 0$,
    \item $l_3 \cdot m_1 + l_3 \cdot m_1 - l_3 \cdot m_1 = 0$,
    \item $l_1 \cdot l_3 + l_3 \cdot l_4 - l_1 \cdot l_3 - l_3 \cdot l_4 - l_3 \cdot m_2 - l_3 \cdot m_2 = 0$,
    \item $l_2 \cdot l_3 + l_4 \cdot l_4 - l_2 \cdot l_3 - l_4 \cdot l_4 - l_4 \cdot m_2 - l_4 \cdot m_2 = 0$,
    \item $m_1 \cdot m_2 - m_1 \cdot m_2 - m_1 \cdot m_2 - m_1 \cdot m_2 = 0$,
    \item $l_1 \cdot m_2 - l_1 \cdot l_1 - l_2 \cdot l_3 + l_1 \cdot l_1 + l_2 \cdot l_3 - l_1 \cdot l_1 - l_2 \cdot l_3 = 0$,
    \item $l_2 \cdot m_2 - l_1 \cdot l_2 - l_2 \cdot l_4 + l_1 \cdot l_2 + l_2 \cdot l_4 - l_1 \cdot l_2 - l_2 \cdot l_4 = 0$,
    \item $l_3 \cdot m_2 - l_1 \cdot l_3 - l_3 \cdot l_4 + l_1 \cdot l_3 + l_3 \cdot l_4 - l_1 \cdot l_3 - l_3 \cdot l_4 = 0$,
    \item $l_4 \cdot m_2 - l_2 \cdot l_3 - l_4 \cdot l_4 + l_2 \cdot l_3 + l_4 \cdot l_4 - l_2 \cdot l_3 - l_4 \cdot l_4 = 0$,
    \item $m_2 \cdot m_2 - m_2 \cdot m_2 - m_2 \cdot m_2 - m_2 \cdot m_2 = 0$.
\end{enumerate}

Kui koondame võrrandites samasugused kuid erinevate märkidega liikmed, siis
jäävad järele järgmised võrrandid:
\begin{multicols}{2}
\begin{enumerate}
    \item $l_3 \cdot m_1 = 0$,
    \item $l_4 \cdot m_1 = 0$,
    \item $l_1 \cdot m_1 + l_4 \cdot m_1 = 0$,
    \item $l_1 \cdot m_1 = 0$,
    \item $l_2 \cdot m_1 = 0$,
    \item $l_2 \cdot m_1 = 0$,
    \item $l_1 \cdot m_1 - m_1 \cdot m_2 = 0$,
    \item $l_1 \cdot m_2 = 0$,
    \item $l_2 \cdot m_2 = 0$,
    \item $l_3 \cdot m_1 = 0$,
    \item $l_4 \cdot m_1 = 0$,
    \item $l_4 \cdot m_1 + m1 \cdot m2 = 0$,
    \item $l_3 \cdot m_1 = 0$,
    \item $l_3 \cdot m_2 = 0$,
    \item $l_4 \cdot m_2 = 0$,
    \item $m_1 \cdot m_2 = 0$,
    \item $l_1 \cdot m_2 - l_1 \cdot l1 - l_2 \cdot l_3 = 0$,
    \item $l_2 \cdot m_2 - l_1 \cdot l2 - l_2 \cdot l_4 = 0$,
    \item $l_3 \cdot m_2 - l_1 \cdot l3 - l_3 \cdot l_4 = 0$,
    \item $l_4 \cdot m_2 - l_2 \cdot l3 - l_4 \cdot l_4 = 0$,
    \item $m_2 \cdot m_2 = 0$.
\end{enumerate}
\end{multicols}

Sellel võrrandisüsteemil on kolm mittetriviaalset lahendit:

\begin{align*}
    \begin{pmatrix}
        m_1 \\ m_2 \\ l_1 \\ l_2 \\ l_3 \vphantom{-\frac{c_1^2}{c_2}} \\ l_4
    \end{pmatrix} = 
    \begin{pmatrix}
        0 \\ 0 \\ c_1 \\ c_2 \\ -\frac{c_1^2}{c_2} \\ -c_1
    \end{pmatrix},
    %
    \quad\quad
    %
    \begin{pmatrix}
        m_1 \\ m_2 \\ l_1 \\ l_2 \\ l_3 \\ l_4
    \end{pmatrix} = 
    \begin{pmatrix}
        c \\ 0 \\ 0 \\ 0 \\ 0 \\ 0
    \end{pmatrix},
    %
    \quad\quad
    %
    \begin{pmatrix}
        m_1 \\ m_2 \\ l_1 \\ l_2 \\ l_3 \\ l_4
    \end{pmatrix} = 
    \begin{pmatrix}
        0 \\ 0 \\ 0 \\ 0 \\ c \\ 0
    \end{pmatrix},
\end{align*}
kus $c, c_1, c_2 \in \K$ on suvalised parameetrid.
Võttes arvesse samasusi \eqref{samasused:2-1} saame kommutatsiooniseosed:
\begin{itemize}
    \item $\begin{cases}
              \left[ e_1, f_1, f_1 \right] = c_1 e_1 + c_2 e_2,\\
              \left[ e_2, f_1, f_1 \right] = -\frac{c_1^2}{c_2} e_1 - c_1 e_2
          \end{cases}$,
    \item $\left[ e_1, e_2, f_1 \right] = c f_1$,
    \item $\left[ e_2, f_1, f_1 \right] = c e_1$.
\end{itemize}
\input{sections/5.2_(1,2)_3-lie_superalgebra_klassifikatsioon.tex}

\newpage
\bibliographystyle{plain}
\bibliography{references}

\end{document}