%!TEX root = ../../thesis.tex

\subsection{\texorpdfstring{$1|2$}{1|2}-dimensionaalsed
    \texorpdfstring{$3$}{3}-Lie superalgebrad}

Olgu $3$-Lie superalgebra $\g$ supervektorruumi dimensioon
$1|2$ ja olgu tema generaatoriteks $\{ e_1, f_1, f_2 \}$. Siis
\begin{align*}
    \begin{matrix*}[l]
        & [e_1, e_1, e_1] = 0, \quad
        & [e_1, f_2, f_2] = \l_3 e_1, \\
        %
        & [e_1, e_1, f_1] = 0, \quad
        & [f_1, f_1, f_1] = \m_1 f_1 + \m_2 f_2, \\
        %
        & [e_1, e_1, f_2] = 0, \quad
        & [f_1, f_1, f_2] = \m_3 f_1 + \m_4 f_2, \\
        %
        & [e_1, f_1, f_1] = \l_1 e_1, \quad
        & [f_1, f_2, f_2] = \m_5 f_1 + \m_6 f_2, \\
        %
        & [e_1, f_1, f_2] = \l_2 e_1, \quad
        & [f_2, f_2, f_2] = \m_7 f_1 + \m_8 f_2,
    \end{matrix*}
\end{align*}
kus $\l_1, \l_2, \l_3, \m_1, \m_2, \dots, \m_8 \in \C$ on suvalised skalaarid.

Rakendades neile kommutaatoritele meie algoritmi saame $41$
mittetriviaalset kitsendust\footnote{Need on leitavad veebilehelt
\url{http://priitlatt.github.io/masters-thesis/kitsendused/1_2.txt}},
mida $\g$ struktuurikonstandid rahuldama peavad. Kitsenduste
mittetriviaalseteks lahenditeks sobivad struktuurikonstantide komplektid
\renewcommand\arraystretch{1.3}
\begin{align*}
    \left\{ \begin{matrix*}[l]
        \l_1 = 0 \\
        \l_2 = 0 \\
        \l_3 = 0 \\
        \m_1 = -c_1 \\
        \m_2 = \frac{c_1^2}{c_2} \\
        \m_3 = -c_2 \\
        \m_4 = c_1 \\
        \m_5 = -\frac{c_2^2}{c_1} \\
        \m_6 = c_2 \\
        \m_7 = -\frac{c_2^3}{c_1^2} \\
        \m_8 = \frac{c_2^2}{c_1} \\
    \end{matrix*} \right.,
    \quad
    \left\{ \begin{matrix*}[l]
        \l_1 = 0 \\
        \l_2 = 0 \\
        \l_3 = 0 \\
        \m_1 = 0 \\
        \m_2 = c_3 \\
        \m_3 = 0 \\
        \m_4 = 0 \\
        \m_5 = 0 \\
        \m_6 = 0 \\
        \m_7 = 0 \\
        \m_8 = 0 \\
    \end{matrix*} \right.,
    \quad
    \left\{ \begin{matrix*}[l]
        \l_1 = 0 \\
        \l_2 = 0 \\
        \l_3 = 0 \\
        \m_1 = 0 \\
        \m_2 = 0 \\
        \m_3 = 0 \\
        \m_4 = 0 \\
        \m_5 = 0 \\
        \m_6 = 0 \\
        \m_7 = c_4 \\
        \m_8 = 0 \\
    \end{matrix*} \right.,
\end{align*}
\renewcommand\arraystretch{1}
kus $c_1, c_2, c_3, c_4 \in \C \setminus \{0\}$ on vabad muutujad.

\begin{lau} \label{lause:1|2 seosed}
    $3$-Lie superalgebra, mille supervektorruumi dimensioon on $1|2$, on Abeli
    Lie superalgebra, või tema mittetriviaalsed kommutatsiooniseosed on
    \renewcommand\arraystretch{1.2}
    \begin{align}
        & \left\{
            \begin{matrix*}[l]
                [f_1, f_1, f_1] = -c_1 f_1 + \frac{c_1^2}{c_2} f_2, \\
                [f_1, f_1, f_2] = -c_2 f_1 + c_1 f_2, \\
                [f_1, f_2, f_2] = -\frac{c_2^2}{c_1} f_1 + c_2 f_2, \\
                [f_2, f_2, f_2] = -\frac{c_2^3}{c_1^2} f_1 +
                    \frac{c_2^2}{c_1} f_2, \\
            \end{matrix*}
        \right. \\[0.2cm]
        %
        &\ \ \ [f_1, f_1, f_1] = c f_2, \\[0.2cm]
        %
        &\ \ \ [f_2, f_2, f_2] = c f_1,
    \end{align}
    \renewcommand\arraystretch{1}
    kus $c, c_1, c_2 \in \C \setminus \{ 0 \}$ on vabad muutujad.
\end{lau}

Kuna lauses \ref{lause:1|2 seosed} on täpselt samad seosed nagu lauses
\ref{lause:0|2 seosed}, siis saame analoogiliselt eelnevas punktis kirjeldatule
järgneva teoreemi.

\begin{thm}
    Olgu $3$-Lie superalgebra $\g$ supervektorruumi dimensioon on $1|2$.
    Siis $\g$ on kas Abeli Lie superalgebra, või ta on isomorfne $3$-Lie
    superalgebraga $\h$, mille mittetriviaalsed kommutatsiooniseosed on kas
    \[ \left\{
            \begin{matrix*}[l]
                [f_1, f_1, f_1] = -f_1 + f_2, \\
                [f_1, f_1, f_2] = -f_1 + f_2, \\
                [f_1, f_2, f_2] = -f_1 + f_2, \\
                [f_2, f_2, f_2] = -f_1 + f_2, \\
            \end{matrix*}
    \right. \]
    või
    \[ [f_1, f_1, f_1]  = f_2, \]
    kus $f_1$ ja $f_2$ on $\h$ paaritud generaatorid.
\end{thm}

Võttes arvesse eelmise punkti tulemusi näeme, et iga $3$-Lie superalgebra
jaoks, mille supervektorruumi dimensioon on $1|2$, leidub
$3$-Lie superalgebra vektorruumi dimensiooniga $0|2$ nii, et nende
kommutatsiooniseosed ühtivad. Seega on võimalik $0|2$ dimensioonilt
triviaalselt jätkata seda algebrat dimensioonile $1|2$.
