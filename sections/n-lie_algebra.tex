%!TEX root = ../thesis.tex

%%%%%%%%%%%%%%%%%%%%
%% n-Lie algebra  %%
%%%%%%%%%%%%%%%%%%%%

\section{\texorpdfstring{$n$}{n}-Lie algebra}

Selle peatüki eesmärgiks on klassikalise Lie algebra
üldistamine, mille käigus toome sisse $n$-Lie algebra mõiste.
Edasi tutvustame esmalt artiklis \cite{AMS:2011} näidatud eeskirja,
mille abil on võimalik Lie algebrast konstrueerida
ternaarne Lie algebra, ning jätkame seda teooriaarendust
tuginedes artiklile \cite{AKMS:2014}, kus kirjeldatakse
üldisemalt kuidas $n$-Lie algebrast indutseerida $(n+1)$-Lie
algebra.

\subsection{\texorpdfstring{$n$}{n}-Lie algebra definitsioon}

Lie algebra definitsioonis on kesksel kohal kaldsümmeetriline
bilineaarne korrutustehe, mis rahuldab Jacobi samasust.
Üheks viisiks Lie algebra mõistet üldistada, ongi just nimelt
tema korrutamise üldistamine. Seda tehes on loomulik nõuda, et
ka üldistatud korrutamistehe rahuldaks kaldsümmeetrilisuse
tingimust ning Jacobi samasust, või vähemalt selle mingit
analoogi, mis annaks juhul $n=2$ täpselt Jacobi samasuse.

Filippov\footnote{Aleksei Fedorovich Filippov (1923--2006),
vene matemaatik} tutvustas aastal 1985 artikis \cite{filippov1985}
$n$-Lie algebrate klassi, kus bilineaarne korrutamine on
asendatud $n$-lineaarse kaldsümmeetrilise operatsiooniga,
mis rahuldab teatud samasust. \cite{kasymov1987}
Tänaseks on just see, Nambu mehaanikast välja kasvanud
üldistus osutunud üheks põhiliseks Lie algebrate edasiseks
uurimissuunaks.

\begin{dfn}
    Vektorruumi $\g$ nimetatakse \emph{$n$-Lie algebraks}, kui
    on määratud $n$-lineaarne kaldsümmeetriline kujutus
    $\nbrac{\cdot}{\cdot} \colon \g^n \to \g$, mis
    suvaliste
    \[ x_1, \dots, x_{n-1}, y_1, \dots, y_n \in \g \]
    korral rahuldab tingimust
    \begin{align}\label{id:filippov}
        \left[ x_1, \dots, x_{n-1}, \nbrac{y_1}{y_n} \right] =
        \sum_{i=1}^n \left[
            y_1, \dots, \left[ x_1, \dots, x_{n-1}, y_i \right], \dots, y_n
        \right].
    \end{align}
\end{dfn}

Võrdust \eqref{id:filippov} $n$-Lie algebra definitsioonis
nimetatakse üldistatud Jacobi samasuseks või ka
\emph{Filippovi samasuseks}. Vahetu kontrolli põhjal
on selge, et valides $n = 2$, saame Filippovi samasusest
\eqref{id:filippov} Jacobi samasuse \eqref{id:jacobi}. Seejuures
$n$-aarse Lie sulu kaldsümmeetrilisus tähendab, et
suvaliste $x_1, x_2, \dots, x_n \in \g$ korral
\begin{align}\label{def:n-lie-brac-antisymm}
    [x_1, \dots, x_i, x_{i+1}, \dots, x_n] =
    -[x_1, \dots, x_{i+1}, x_i, \dots, x_n].
\end{align}

Toome siinkohal $n$-Lie algebra kohta näite, mille
Filippov esitas artiklis \cite{filippov1985} vahetult
pärast oma definitsiooni.

\begin{naide}
    Olgu $E$ reaalne $(n+1)$-mõõtmeline Eukleidiline ruum,
    ning tähistame elementide $x_1, x_2, \dots, x_n \in E$
    vektorkorrutise $[x_1, x_2, \dots, x_n]$. Meenutame, 
    et vektorkorrutis on kaldsümmeetriline ning iga teguri
    suhtes lineaarne. Lisaks, kui meil on ruumi $E$ mingi
    baas $\{e_1, e_2, \dots, e_{n+1}\}$, siis avaldub see
    vektorrkorrutis determinandina
    \begin{align}\label{eq:n-vec-mult}
        [x_1, x_2, \dots, x_n] = \left|\begin{matrix}
          x_{11}    & x_{12}    & \dots  & x_{1n}    & e_1    \\
          x_{21}    & x_{22}    & \dots  & x_{2n}    & e_2    \\
          \vdots    & \vdots    & \ddots & \vdots    & \vdots \\
          x_{n+1,1} & x_{n+1,2} & \dots  & x_{n+1,n} & e_{n+1}
        \end{matrix}\right|,
    \end{align}
    kus $(x_{1i}, x_{2i}, \dots, x_{n+1,i})$ on vektorite
    $x_i$, $i = 1, 2, \dots, n$, koordinaadid.

    Kui me varustame ruumi $E$ nüüd $n$-aarse vektorkorrutisega
    \eqref{eq:n-vec-mult}, siis saame $(n+1)$-mõõtmelise
    reaalse kaldsümmeetrilise algebra, mida tähistame
    $\E_{n+1}$. Tänu determinandi multilineaarsusele on
    vektorkorrutis täielikult määratud baasivektorite
    korrutustabeliga. Võrdusest \eqref{eq:n-vec-mult} saame
    me baasivektoritele järgmise korrutustabeli:
    \begin{align}\label{eq:vec-mult-table}
        [e_1, \dots, e_{i-1}, \hat{e_i}, e_{i+1}, \dots, e_{n+1}]
        = (-1)^{n+1+i} e_i,
    \end{align}
    kus $i = 1, 2, \dots, n+1$, ja $\hat{e_i}$ tähistab
    vektori $e_i$ arvutusest välja jätmist. Ülejäänud
    baasivektorite korrutised on kas nullid või kättesaadavad
    võrdusest \eqref{eq:vec-mult-table} ja
    kaldsümmeetrilisusest.

    Selle põhjal saab näidata, et algebra
    $(\E_{n+1}, [\cdot, \dots, \cdot])$ on $n$-Lie
    algebra. \cite{filippov1985}
\end{naide}

Punktis \ref{subsec:algebralised-struktuurid} toodud konstruktsioonid
on loomulikul viisil võmalik esitada ka üldisemal juhul.

\begin{dfn}\label{def:n-lie-alamalgebra}
    Me ütleme, et $\h$ on $n$-Lie algebra $(\g, [\cdot, \dots, \cdot])$
    alamalgebra, kui $\h \subset \g$ on alamruum, ning suvaliste
    $x_1, x_2, \dots, x_n \in \h$ korral $[x_1, x_2, \dots, x_n] \in \h$.
\end{dfn}

\begin{dfn}\label{def:n-lie-algebra-ideaal}
    Olgu $(\g, [\cdot, \dots, \cdot])$ $n$-Lie algebra ning olgu
    $\h \subset \g$ alamruum. Me ütleme, et $\h$ on $\g$ \emph{ideaal}, kui
    iga $h \in \h$ ja $x_1, x_2, \dots, x_{n-1} \in \g$ korral
    \begin{align*}
        [h, x_1, x_2, \dots, x_{n-1}] \in \h.
    \end{align*}
\end{dfn}

Analoogiliselt klassikalisele juhule defineeritakse ka
\emph{$n$-Lie faktoralgebra}.
Kui meil on antud $n$-Lie algebra $(\g, [\cdot, \dots, \cdot])$ ja tema
ideaal $\h$, siis faktoralgebra $\g/\h$ kõrvalklassideks on
$\overline{x} = x + \h$, $x \in \g$, ja kommutaatoriks on
\begin{align*}
    [x_1 + \h, x_2 + \h, \dots, x_n + \h]_{\g/\h} =
    [x_1, x_2, \dots, x_n] + \h,\quad
    x_1, x_2, \dots, x_n \in \g.
\end{align*}

\subsection{Indutseeritud \texorpdfstring{$n$}{n}-Lie algebra
  }\label{subsec:indutseeritud-n-lie-alg}

Artiklis \cite{AKMS:2014} on uuritud põhjalikult konstruktsiooni, mille
abil on võimalik etteantud $n$-Lie algebrast teatud tingimustel indutseerida
$(n+1)$-Lie algebra. Toome järgnevas ära selle konstruktsiooniga seotud
põhilised tulemused.

\begin{dfn}
    Olgu $\A$ vektorruum ning olgu $\phi \col \A^n \to \A$. Me
    ütleme, et lineaarkujutus $\tau \col \A \to K$ on
    \emph{$\phi$ jälg}, kui suvaliste $x_1, \dots, x_n \in \A$
    korral
    \begin{align*}
        \tau \left(
            \phi \left( x_1, \dots, x_n \right)
        \right) = 0.
    \end{align*}
\end{dfn}

Olgu $\phi \col \A^n \to \A$ kõigi argumentide järgi lineaarne ja
olgu $\tau \col \A \to K$ lineaarne kujutus. Defineerime mugavuse
ja selguse mõttes sisse uue kujutuse $\phi_i \col \A^{n+1} \to \A$,
$i = 1, \dots, n+1$, valemiga
\begin{align*}
    \phi_i\left(x_1, \dots, x_i, \dots, x_{n+1}\right) &=
    \phi \left(x_1, \dots, \hat{x_i}, \dots, x_{n+1}\right) = \\
    &= \phi \left(x_1, \dots, x_{i-1}, x_{i+1}, \dots, x_{n+1}\right),
\end{align*}
kus $\hat{x_i}$ tähistab kõrvalejäätavat elementi, see tähendab
$\phi$ arvutatakse elementidel $x_1, \dots, x_{i-1}, x_{i+1}, \dots, x_{n+1}$.

Defineerime nende
kujutuste abil uue $(n+1)$-lineaarse kujutuse
$\phi_\tau \col \A^{n+1} \to \A$ valemiga
\begin{align}\label{eq:phi_tau}
    \phi_\tau \left( x_1, \dots, x_{n+1} \right) =
    \sum_{i=1}^{n+1} (-1)^{i-1} \tau(x_i)
        \phi_i(x_1, \dots, x_{n+1}),
\end{align}


Seega võttes näiteks $n = 2$, saame valemi \eqref{eq:phi_tau}
põhjal kirjutada
\[
    \phi_\tau (x_1, x_2, x_3) =
        \tau(x_1) \phi(x_2, x_3) -
        \tau(x_2) \phi(x_1, x_3) +
        \tau(x_3) \phi(x_1, x_2).
\]

Osutub, et selliselt defineeritud kujutusel $\phi_\tau$ on
mitmed head omadused, nagu võib lugeda artiklitest \cite{AKMS:2014,AMS:2011}.

\begin{lemma}
    Olgu $\A$ vektorruum ning $\phi \col \A^n \to \A$ $n$-lineaarne
    kaldsümmeetriline kujutus ja $\tau \col \A \to K$ lineaarne.
    Siis kujutus $\phi_\tau \col A^{n+1} \to A$ on samuti
    kaldsümmeetriline. Lisaks, kui $\tau$ on $\phi$ jälg, siis
    $\tau$ on ka $\phi_\tau$ jälg.
\end{lemma}

\begin{proof}
    Eeldame, et $\A$ on vektorruum, $\phi \col \A^n \to \A$ on
    $n$-lineaarne ning kaldsümmeetriline ja $\tau \col \A \to K$
    on lineaarne.

    Veendumaks, et $\phi_\tau$ on $(n+1)$-lineaarne ning
    kaldsümmeetriline olgu
    \[
        x_1, x_2, \dots, x_{n+1}, x_j^1, x_j^2 \in \A,
        \quad
        j \in \{1, 2, \dots, n+1\},
    \]
    ning olgu $\lambda, \mu$ skalaarid. Arvestades nii
    $\phi$ kui ka $\tau$ lineaarsust, märgime $\phi_\tau$
    lineaarsuseks, et
    \begin{align*}
        &\phi_\tau(
            x_1, \dots, \lambda x_j^1 + \mu x_j^2, \dots, x_{n+1}
        ) = \\
        &= \sum_{i=1}^{n+1} (-1)^{i-1} \tau(x_i) \phi_i(
             x_1, \dots, \lambda x_j^1 + \mu x_j^2, \dots, x_{n+1}
           ) = \\
        &= \sum_{\mathclap{i=1,\ i \ne j}}^{n+1} (-1)^{i-1}
           \tau(x_i) \phi_i(
             x_1, \dots, \lambda x_j^1 + \mu x_j^2, \dots, x_{n+1}
           ) + \\
        &\quad\ (-1)^{j-1} \tau(\lambda x_j^1 + \mu x_j^2)
           \phi_j(
             x_1, \dots, \lambda x_j^1 + \mu x_j^2, \dots, x_{n+1}
           ) = \\
        &= \sum_{\mathclap{i=1,\ i \ne j}}^{n+1}
           \lambda (-1)^{i-1} \tau(x_i) \phi_i(
             x_1, \dots, x_j^1, \dots, x_{n+1}
           ) + \\
        &\quad\ \sum_{\mathclap{i=1,\ i \ne j}}^{n+1} \mu
           (-1)^{i-1} \tau(x_i)
           \phi_i( x_1, \dots, x_j^2, \dots, x_{n+1} ) + \\
        &\quad\ \lambda (-1)^{j-1} \tau(x_j^1) \phi_j(
             x_1, \dots, x_j^1, \dots, x_{n+1}) + \\
        &\quad\ \mu (-1)^{j-1} \tau(x_j^2) \phi_j(
             x_1, \dots, x_j^2, \dots, x_{n+1}) = \\
        &= \lambda \phi_\tau(x_1, \dots, x_j^1, \dots, x_{n+1}) +
           \mu \phi_\tau(x_1, \dots, x_j^2, \dots, x_{n+1}).
    \end{align*}

    Kaldsümmeetrilisus avaldub samuti vahetu arvutuse tulemusena:
    \begin{align*}
        &\phi_\tau(
          x_1, \dots, x_{j-1}, x_j, \dots, x_{n+1}
        ) = \\
        &= \sum_{\substack{i=1\\i \ne j-1,\ i \ne j}}^{n+1}
          (-1)^{i-1} \tau(x_i) \phi_i(
              x_1, \dots, x_{j-1}, x_j, \dots, x_{n+1}
          ) + \\
        &\quad\ (-1)^{j-1-1} \tau(x_{j-1}) \phi(
              x_1, \dots, x_{j-2}, x_j, x_{j+1}, \dots, x_{n+1}
          ) + \\
        &\quad\ (-1)^{j-1} \tau(x_j) \phi(
              x_1, \dots, x_{j-2}, x_{j-1}, x_{j+1}, \dots, x_{n+1}
          ) = \\
        &= -\sum_{\substack{i=1\\i \ne j-1,\ i \ne j}}^{n+1}
          (-1)^{i-1} \tau(x_i) \phi_i(
              x_1, \dots, x_j, x_{j-1}, \dots, x_{n+1}
          ) - \\
        &\quad\ (-1)^{j-1} \tau(x_{j-1}) \phi(
              x_1, \dots, x_{j-2}, x_j, x_{j+1}, \dots, x_{n+1}
          ) - \\
        &\quad\ (-1)^{j-1-1} \tau(x_j) \phi(
              x_1, \dots, x_{j-2}, x_{j-1}, x_{j+1}, \dots, x_{n+1}
          ) = \\
        &= - \phi_\tau(
              x_1, \dots, x_j, x_{j-1}, \dots, x_{n+1}
            )
    \end{align*}

    Tõestuse lõpetuseks eeldame, et $\tau$ on $\phi$ jälg ja
    näitame, et sel juhul on $\tau$ ka $\phi_\tau$ jälg.

    Et $\tau$ on $\phi$ jälg, siis iga $x_1, x_2, \dots, x_n \in \A$
    korral $\tau(\phi(x_1, x_2, \dots, x_n)) = 0$ ja seega
    \begin{align*}
        &\tau(\phi_\tau(x_1, x_2, \dots, x_{n+1})) = \\
        &= \sum_{i=1}^{n+1} \tau \left(
            (-1)^{i-1} \tau(x_i) \phi_i(x_1, \dots, x_{n+1})
        \right) = \\
        &= \sum_{i=1}^{n+1} (-1)^{i-1} \tau(x_i) \tau \left(
            \phi(x_1, \dots, x_{i-1}, x_{i+1}, \dots, x_{n+1})
        \right) = \\
        &= \sum_{i=1}^{n+1} 0 = 0,
    \end{align*}
    mida oligi tarvis.
\end{proof}

Kasutades lemmat, on võimalik tõestada järgmine teoreem. \cite{AKMS:2014}

\begin{thm}\label{thm:n+1_lie_alg}
    Olgu $(\g, [\cdot, \dots, \cdot])$ $n$-Lie algebra ning olgu $\tau$
    lineaarkujutuse $[\cdot, \dots, \cdot]$ jälg. Siis
    $(\g, [\cdot, \dots, \cdot]_\tau)$ on $(n+1)$-Lie algebra.
\end{thm}

Teoreemis kirjeldatud viisil saadud $(n+1)$-Lie algebrat
$(\g, [\cdot, \dots, \cdot]_\tau)$ nimetatakse $n$-Lie algebra
$(\g, [\cdot, \dots, \cdot])$ poolt \emph{indutseeritud}
$(n+1)$-Lie algebraks.

Teoreemist~\ref{thm:n+1_lie_alg} saame teha olulise järlduse:

\begin{jar}
    Olgu $(\g, \brac{\cdot}{\cdot})$ Lie algebra ning olgu antud
    $\brac{\cdot}{\cdot}$ jälg $\tau \col \g \to K$. Siis ternaarne sulg
    $[ \cdot, \cdot, \cdot ] \col \g^3 \to \g$, mis on defineeritud
    valemiga
    \[
        [x, y, z] = \tau(x)[y, z] + \tau(y)[z, x] + \tau(z)[x, y],
    \]
    määrab $3$-Lie algebra struktuuri $\g_\tau$ vektorruumil $\g$.
    \hfill \qed
\end{jar}

\begin{lau}\label{lause:indutseeritud-alamalgebra}
    Olgu $(\g, [\cdot, \dots, \cdot])$ $n$-Lie algebra ning olgu
    $\h \subset \g$ alamalgebra. Kui $\tau$ on $[\cdot, \dots, \cdot]$ jälg,
    siis $\h$ on ka $(\g, [\cdot, \dots, \cdot]_\tau)$ alamalgebra.
\end{lau}

\begin{proof}
    Olgu $\h$ $n$-Lie algebra $(\g, [\cdot, \dots, \cdot])$ alamalgebra,
    $x_1, x_2, \dots, x_{n+1} \in \h$ ning olgu $\tau$ sulu
    $[\cdot, \dots, \cdot]$ jälg. Siis
    \begin{align*}
        [x_1, x_2, \dots, x_{n+1}]_\tau =  \sum_{i=1}^{n+1} (-1)^i \tau(x_i)
        [x_1, x_2, \dots, x_{i-1}, x_{i+1}, \dots, x_{n+1}],
    \end{align*}
    mis on $\h$ elementide lineaarkombinatsioon, kuna iga
    $i = 1, 2, \dots, n+1$ korral
    $[x_1, x_2, \dots, x_{i-1}, x_{i+1}, \dots, x_{n+1}] \in \h$.
\end{proof}

\begin{lau}\label{lause:indutseeritud-ideaal}
    Olgu $(\g, [\cdot, \dots, \cdot])$ ideaal $\h$. Siis $\h$ on
    $(\g, [\cdot, \dots, \cdot]_\tau)$ ideaal parajasti siis, kui
    $[\g, \g, \dots, \g] \subseteq \h$ või $\h \subseteq \ker \tau$.
\end{lau}

\begin{proof}
    Olgu $h \in \h$ ja $x_1, x_2, \dots, x_n \in \g$ suvalised. Siis
    \begin{align*}
        [x_1, x_2, \dots, x_n, h]_\tau =  &\sum_{i=1}^{n} (-1)^i \tau(x_i)
        [x_1, x_2, \dots, x_{i-1}, x_{i+1}, \dots, x_n, h] + \\
        &(-1)^{n+1} \tau(h) [x_1, x_2, \dots, x_n].
    \end{align*}
    Et $\h$ on $(\g, [\cdot, \dots, \cdot])$ ideaal, siis ilmselt
    \begin{align*}
        \sum_{i=1}^{n} (-1)^i \tau(x_i)
        [x_1, x_2, \dots, x_{i-1}, x_{i+1}, \dots, x_n, h] \in \h.
    \end{align*}
    Niisiis, tingimus $[x_1, x_2, \dots, x_n, h]_\tau \in \h$ on
    samaväärne tingimusega
    \[ (-1)^{n+1} \tau(h) [x_1, x_2, \dots, x_n] \in \h. \]
    Viimase võime aga lahti kirjutada kujul
    $\tau(h) = 0$ või $[x_1, x_2, \dots, x_n] \in \h$.
\end{proof}

\subsection{Nambu mehaanika. Nambu-Poissoni sulg}

Filippovile olid $n$-Lie algebra konstrueerimisel peamiseks
inspiratsiooniks Jaapani füüsiku ja Nobeli preemia laureaadi
Yoichiro Nambu tööd Hamiltoni mehaanika
üldistuste vallas, mida täna tuntakse kui Nambu mehaanikat. Osutub, et
seal esinev \emph{Nambu-Poissoni} sulg on $n$-Lie algebra sulu erijuht,
mida on põhjalikult uurinud Takhtajan. Tuginedes Takhtajani artiklile
\cite{takhtajan1994}, toome Nambu mehaanika abil veel ühe
näite $n$-Lie algebrast.

Alustame klassikalisest situatsioonist ehk tavalisest Hamiltoni mehaanikast.

\begin{dfn}
    \emph{Poissoni muutkonnaks} nimetatakse siledat muutkonda $M$ koos
    funktsioonide algebraga $A = C^\infty(M)$, kui on antud binaarne
    kujutus
    \[ \{\cdot, \cdot\} \col A \otimes A \to A, \]
    mis rahuldab järgmisi tingimusi:
    \begin{enumerate}
      \item kaldsümmeetrilisus, iga $f_1, f_2 \in A$ korral
          \begin{align}\label{eq:poisson-antisymm}
              \{f_1, f_2\} = -\{f_2, f_1\},
          \end{align}
      \item kehtib Leibnizi tingimus, ehk iga $f_1, f_2, f_3 \in A$ korral
          \begin{align}
              \{f_1 f_2, f_3\} = f_1 \{f_2, f_3\} + f_2 \{f_1, f_3\},
          \end{align}
      \item kehtib Jacobi samasus, kui $f_1, f_2, f_3 \in A$, siis
          \begin{align}\label{eq:poisson-jacobi-id}
              \{f_1, \{f_2, f_3\}\} + \{f_2, \{f_3, f_1\}\} +
              \{f_3, \{f_1, f_2\}\} = 0.
          \end{align}
    \end{enumerate}
\end{dfn}

Poissoni muutkonna definitsioonis esinevat kujutust $\{\cdot, \cdot\}$
nimetatakse \emph{Poissoni suluks}, ja funktsioonide algebrat $A$ nimetatakse
\emph{vaadeldavate algebraks}. Kuna definitsioonis on nõutud tingimused
\eqref{eq:poisson-antisymm} ja \eqref{eq:poisson-jacobi-id}, siis on selge,
et Poissoni muutkonnal on olemas Lie algebra struktuur.
Selgub, et Poissoni sulg mängib klassikalises mehaanikas olulist rolli.
Nimelt, klassikalises mehaanikas on teada, et dünaamilise süsteemi ajas
muutumist kirjeldab Poissoni sulg, mille abil on võimalik formuleerida
Hamiltoni liikumisvõrrand
\begin{align}\label{eq:hamilton-equation}
    \frac{\mathrm{d} f}{\mathrm{d} t} = \{H, f\}, \quad f \in A,
\end{align}
kus $H \in A$ on fikseeritud operaator, mis vastab süsteemi koguenergiale,
ja mida nimetatakse \emph{Hamiltoniaaniks}. Saadud konstruktsioon
kirjeldab süsteemi muutumist ajas, ehk selle süsteemi dünaamikat, ja on
seega olulisel kohal paljude kvantteooriate kirjeldustes.

Lihtsaima Poissoni muutkonna näitena võime vaadelda $M = \R^2$
koordinaatidega $x$ ja $y$, ning Poissoni suluga
\begin{align*}
    \{f_1, f_2\} =
        \frac{\partial f_1}{\partial x} \frac{\partial f_2}{\partial y} -
        \frac{\partial f_1}{\partial y} \frac{\partial f_2}{\partial x} =
        \frac{\partial(f_1, f_2)}{\partial(x, y)},
\end{align*}
või üldisemalt $M = \R^{2n}$, $n \in \N$, koordinaatidega
$x_1, x_2, \dots, x_n, y_1, y_2, \dots, y_n$, kus Poissoni muutkonna
struktuur on antud suluga
\begin{align}\label{eq:R^n-poisson-sulg}
    \{f_1, f_2\} = \sum_{i=1}^{n} \left(
        \frac{\partial f_1}{\partial x_i} \frac{\partial f_2}{\partial y_i} -
        \frac{\partial f_1}{\partial y_i} \frac{\partial f_2}{\partial x_i}
    \right).
\end{align}

Nii defineeritud Poissoni sulg rahuldab Poissoni muutkonna definitsioonis
seatud tingimusi. Tõepoolest, võrdusest \eqref{eq:R^n-poisson-sulg}
järeldub kaldsümmeetrilisus vahetult. Leibnizi reegli kehtivuseks
märgime, et
\begin{align*}
    \{f_1 f_2, f_3\} &= \sum_{i=1}^{n} \left(
        \frac{\partial (f_1 f_2)}{\partial x_i}
        \frac{\partial f_3}{\partial y_i} -
        \frac{\partial (f_1 f_2)}{\partial y_i}
        \frac{\partial f_3}{\partial x_i}
    \right) = \\
    &= \sum_{i=1}^{n} \left[
        \left(
            f_1 \frac{\partial f_2}{\partial x_i} +
            f_2 \frac{\partial f_1}{\partial x_i} +
        \right) \frac{\partial f_3}{\partial y_i} -
        \left(
            f_1 \frac{\partial f_2}{\partial y_i} +
            f_2 \frac{\partial f_1}{\partial y_i} +
        \right) \frac{\partial f_3}{\partial x_i}
    \right] = \\
    &= f_1 \sum_{i=1}^{n} \left(
            \frac{\partial f_2}{\partial x_i}
            \frac{\partial f_3}{\partial y_i} -
            \frac{\partial f_2}{\partial y_i}
            \frac{\partial f_3}{\partial x_i}
        \right) +
        f_2 \sum_{i=1}^{n} \left(
                \frac{\partial f_1}{\partial x_i}
                \frac{\partial f_3}{\partial y_i} -
                \frac{\partial f_1}{\partial y_i}
                \frac{\partial f_3}{\partial x_i}
        \right) = \\
    &= f_1 \{f_2, f_3\} + f_2 \{f_1, f_3\}.
\end{align*}

Veendumaks, et kehtib ka Jacobi samasus, piisab vahetult arvutada
\begin{align*}
    \{f_1, \{f_2, f_3\}\}, \quad
    \{f_2, \{f_3, f_1\}\}, \quad
    \{f_3, \{f_1, f_2\}\}.
\end{align*}
Arvestades, et funktsioonide korrutamine on defineeritud punktiviisi ning
võttes arvesse korrutise diferentseerimise eeskirja, võime saadud tulemusi
liites näha, et nende summa on tõepoolest null, ehk tingimus
\eqref{eq:poisson-jacobi-id} on täidetud.

Muutkonna $M$ Poissoni struktuuri on valemi \eqref{eq:R^n-poisson-sulg}
asemel võimalik kirjeldada ka kaasaegse diferentsiaalgeomeetria
aparatuuri abil. Selleks oletame, et muutkonnal $M$ on antud
kaldsümmeetriline vorm
\[ \eta \col D(M) \otimes_{C^\infty(M)} D(M) \to C^\infty(M), \]
kus sümboliga $D(M)$ on tähistatud muutkonna $M$ vektorväljade vektorruum.
Poissoni sulu võib siis defineerida valemiga
\begin{align}\label{eq:poisson-brac-nabla}
    \{ f_1, f_2 \} = \eta(\nabla f_1, \nabla f_2),
\end{align}
kus $f_1, f_2 \in C^\infty(M)$ ja vektorväli $\nabla f$ on
funktsiooni $f$ gradient. Kui $x_1, x_2, \dots, x_n$ on muutkonna
lokaalsed koordinaadid, siis vektorväljad
\[
    \frac{\partial}{\partial x_1},
    \frac{\partial}{\partial x_2},
    \dots,
    \frac{\partial}{\partial x_n}
\]
moodustavad baasi vektorväljade moodulile üle funktsioonide algebra,
ning vorm $\eta$ tekitab kaldsümmeetrilise kaks korda kovariantse
tensorvälja
\[
    \eta_{ij} = \eta\left(
        \frac{\partial}{\partial x_i}, \frac{\partial}{\partial x_j}
    \right),
\]
mida nimetatakse \emph{Poissoni tensorväljaks}.

Seega saame arvutada
\begin{align*}
    \{f_1, f_2\} &= \eta(\nabla f_1, \nabla f_2)
    = \eta\left(
        \sum_{i=1}^{n}
            \frac{\partial f_1}{\partial x_i}
            \frac{\partial}{\partial x_i},
        \sum_{j=1}^{n}
            \frac{\partial f_2}{\partial x_j}
            \frac{\partial}{\partial x_j}
    \right) = \\
    &= \sum_{i=1}^{n} \sum_{j=1}^{n}
        \frac{\partial f_1}{\partial x_i}
        \frac{\partial f_2}{\partial x_j}
        \eta\left(
            \frac{\partial}{\partial x_i},
            \frac{\partial}{\partial x_j}
        \right)
    = \sum_{i=1}^{n} \sum_{j=1}^{n} \eta_{ij}
        \frac{\partial f_1}{\partial x_i}
        \frac{\partial f_2}{\partial x_j} = \\
    &= \sum_{i < j} \eta_{ij}
        \frac{\partial f_1}{\partial x_i}
        \frac{\partial f_2}{\partial x_j} +
    \sum_{i > j} \eta_{ij}
        \frac{\partial f_1}{\partial x_i}
        \frac{\partial f_2}{\partial x_j} +
    \sum_{i = 1}^{n} \eta_{ii}
        \frac{\partial f_1}{\partial x_i}
        \frac{\partial f_2}{\partial x_i} = \\
    &= \sum_{i < j} \eta_{ij}
        \frac{\partial f_1}{\partial x_i}
        \frac{\partial f_2}{\partial x_j} +
    \sum_{j > i} \eta_{ji}
        \frac{\partial f_1}{\partial x_j}
        \frac{\partial f_2}{\partial x_i} = \\
    &= \sum_{i < j} \eta_{ij}
        \frac{\partial f_1}{\partial x_i}
        \frac{\partial f_2}{\partial x_j} -
    \sum_{i < j} \eta_{ij}
        \frac{\partial f_1}{\partial x_j}
        \frac{\partial f_2}{\partial x_i} = \\
    &= \sum_{i < j} \eta_{ij} \left(
        \frac{\partial f_1}{\partial x_i}
        \frac{\partial f_2}{\partial x_j} -
        \frac{\partial f_1}{\partial x_j}
        \frac{\partial f_2}{\partial x_i}
    \right) = \\
    &= \frac{1}{2} \left[
        \sum_{i < j}
            \eta_{ij} \left(
                \frac{\partial f_1}{\partial x_i}
                \frac{\partial f_2}{\partial x_j} -
                \frac{\partial f_1}{\partial x_j}
                \frac{\partial f_2}{\partial x_i}
            \right)
        + \sum_{j < i}
            \eta_{ji} \left(
                \frac{\partial f_1}{\partial x_j}
                \frac{\partial f_2}{\partial x_i} -
                \frac{\partial f_1}{\partial x_i}
                \frac{\partial f_2}{\partial x_j}
            \right)
    \right] = \\
    &= \frac{1}{2} \left[
        \sum_{i < j}
            \eta_{ij} \left(
                \frac{\partial f_1}{\partial x_i}
                \frac{\partial f_2}{\partial x_j} -
                \frac{\partial f_1}{\partial x_j}
                \frac{\partial f_2}{\partial x_i}
            \right)
        - \sum_{j < i}
            \eta_{ij} \left(
                \frac{\partial f_1}{\partial x_j}
                \frac{\partial f_2}{\partial x_i} -
                \frac{\partial f_1}{\partial x_i}
                \frac{\partial f_2}{\partial x_j}
            \right)
    \right] = \\
    &= \frac{1}{2} \left[
        \sum_{i < j}
            \eta_{ij} \left(
                \frac{\partial f_1}{\partial x_i}
                \frac{\partial f_2}{\partial x_j} -
                \frac{\partial f_1}{\partial x_j}
                \frac{\partial f_2}{\partial x_i}
            \right)
        + \sum_{j < i}
            \eta_{ij} \left(
                \frac{\partial f_1}{\partial x_i}
                \frac{\partial f_2}{\partial x_j} -
                \frac{\partial f_1}{\partial x_j}
                \frac{\partial f_2}{\partial x_i}
            \right)
    \right] = \\
    &= \sum_{i=1}^{n} \sum_{j=1}^{n} \eta_{ij} \frac{1}{2}
        \left(
            \frac{\partial f_1}{\partial x_i}
            \frac{\partial f_2}{\partial x_j} -
            \frac{\partial f_1}{\partial x_j}
            \frac{\partial f_2}{\partial x_i}
        \right).
\end{align*}

Tuues sisse väliskorrutise
\[
    \frac{\partial}{\partial x_i} \wedge
    \frac{\partial}{\partial x_j} (f_1, f_2) =
    \frac{1}{2} \left(
        \frac{\partial f_1}{\partial x_i}
        \frac{\partial f_2}{\partial x_j} -
        \frac{\partial f_1}{\partial x_j}
        \frac{\partial f_2}{\partial x_i}
    \right),
\]
siis võime Poissoni sulu \eqref{eq:poisson-brac-nabla} panna kirja kujul
\begin{align}\label{eq:poisson-brac-nabla2}
    \{f_1, f_2\} = \sum_{i=1}^{n} \sum_{j=1}^{n} \eta_{ij}
    \frac{\partial}{\partial x_i} \wedge
    \frac{\partial}{\partial x_j} (f_1, f_2).
\end{align}

Et sulg \eqref{eq:poisson-brac-nabla2} määraks Poissoni muutkonna struktuuri,
peab ta rahuldama ka Jacobi samasust. Osutub, et Jacobi samasuse kehtimine
sulu \eqref{eq:poisson-brac-nabla2} korral on samaväärne tingimusega,
et Schouteni sulg tensorvärljast $\eta$ iseenesega on null. \cite{takhtajan1994}

Nambu asendas oma käsitluses binaarse Poissoni sulu ternaarse või koguni
$n$-aarse operatsiooniga muutkonna $M$ vaadeldavate algebral $A$.
Dünaamilise süsteemi kirjeldamiseks ehk Hamiltoni liikumisvõrrandi
\eqref{eq:hamilton-equation} formuleerimise jaoks on
sellisel juhul tarvis ühe Hamiltoniaani asemel vaadata vastavalt kas kahte
või $n-1$ Hamiltoniaani $H_1, H_2, \dots, H_{n-1}$.
Sellisel juhul näitavad need Hamiltoniaanid süsteemi
dünaamikat määravate sõltumatute parameetrite maksimaalset arvu.

\begin{dfn}
    Muutkonda $M$ nimetatakse $n$-järku \emph{Nambu-Poissoni muutkonnaks},
    kui on määratud kujutus
    $\{\cdot, \dots, \cdot\} \col A^{\otimes^n} \to A$, mis on
    \begin{enumerate}
        \item kaldsümmeetriline, see tähendab iga $f_1, f_2, \dots, f_n \in A$
            korral
            \begin{align}\label{def:nambu-brac-antisymm}
                \{f_1, \dots, f_n\} = (-1)^{|\sigma|}
                \{f_{\sigma(1)}, \dots, f_{\sigma(n)}\},
            \end{align}
        \item rahuldab Leibnizi tingimust, ehk iga
            $f_1, f_2, \dots, f_{n+1} \in A$ korral
            \begin{align}
                \{f_1 f_2, f_3, \dots, f_n\} =
                f_1 \{f_2, f_3, \dots, f_n\} +
                f_2 \{f_1, f_3, \dots, f_n\},
            \end{align}
        \item iga $f_1, f_2, \dots, f_{n-1}, g_1, g_2, \dots, g_n,$ korral on
        täidetud samasus
            \begin{align}\label{def:nambu-manifold-fund-id}
                \{f_1, \dots, f_{n-1}, \{g_1, \dots, g_n\}\} =
                \sum_{i=1}^{n} \{
                    g_1, \dots, \{f_1, \dots, f_{n-1}, g_i\}, \dots, g_n
                \}.
            \end{align}
    \end{enumerate}
\end{dfn}

Definitsioonis nõutavat $n$-aarset operatsiooni nimetatakse seejuures
\emph{Nambu suluks}, ja kaldsümmeetrilisuse tingimuses tähistab
$\sigma$ indeksite $i = 1, 2, \dots, n$ permutatsiooni ning
$|\sigma|$ selle permutatsiooni paarsust. Tegelikult on
aga võrdusega \eqref{def:nambu-brac-antisymm} antud kaldsümmeetrilisus
täpselt sama, mis $n$-Lie algebra definitsioonis nõutud kaldsümmeetrilisus
\eqref{def:n-lie-brac-antisymm}, sest kahe järjestikuse elemendi vahetamise
tulemusel on võimalik saada mistahes permutatsioon. Teisalt Nambu
muutkonna definitsioonis nõutud samasus \eqref{def:nambu-manifold-fund-id}
ei ole mitte midagi muud, kui juba meile tuttav Filippovi samasus
\eqref{id:filippov}. Seega kokkuvõttes tekib meil Nambu muutkonnal
loomulikul viisil $n$-Lie algebra struktuur.

Kokkuvõttes on Nambu-Poissoni muutkonnal dünaamiline süsteem määratud
$n-1$ funktsiooniga $H_1, H_2, \dots, H_{n-1}$, mille abil on võimalik
formuleerida \emph{Nambu-Hamiltoni liikumisvõrrandi}
\begin{align*}
    \frac{\mathrm{d} f}{\mathrm{d} t} = \{H_1, \dots, H_{n-1}, f\},
    \quad f \in A.
\end{align*}

Sellise süsteemi saab analoogiliselt klassikalisele juhule kirjeldada
kasutades diferentsiaalgeomeetria vahendeid, nagu on näha
Takhtajani artiklis \cite{takhtajan1994}. Selleks defineerime Nambu
sulu valemiga
\begin{align*}
    \{f_1, \dots, f_n\} = \eta\left( \nabla f_1, \dots, \nabla f_n \right),
\end{align*}
kus $\eta \col D(M)^{\otimes^n} \to C^\infty(M)$ on kaldsümmeetriline vorm ja
$D(M)$ tähistab muutkonna $M$ vektorväljade vektorruumi. Kasutades
muutkonnal $M$ lokaalseid koordinaate $(x_1, x_2, \dots, x_n)$, saame
$\eta$ kirjutada kaldsümmeetriliste $n$-korda kovariantse
\emph{Nambu tensorväljana} $\eta_{i_1 i_2 \dots i_n}$:
\begin{align*}
    \eta = \sum_{i_1, \dots, i_n = 1}^{n} \eta_{i_1 i_2 \dots i_n}
        \frac{\partial}{\partial x_{i_1}}
        \wedge \dots \wedge
        \frac{\partial}{\partial x_{i_n}}.
\end{align*}
Kokkuvõttes saab siis Nambu sulg kuju
\begin{align*}
    \{f_1, \dots, f_n\} =
    \sum_{i_1, \dots, i_n = 1}^{n} \eta_{i_1 i_2 \dots i_n}
        \frac{\partial}{\partial x_{i_1}}
        \wedge \dots \wedge
        \frac{\partial}{\partial x_{i_n}} (f_1, \dots, f_n).
\end{align*}
