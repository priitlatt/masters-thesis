%!TEX root = ../../thesis.tex

\subsection{\texorpdfstring{$2|1$}{2|1}-dimensionaalsed
    \texorpdfstring{$3$}{3}-Lie superalgebrad}

Olgu $3$-Lie superalgebra $\g$ supervektorruumi dimensioon
$2|1$ ja olgu tema generaatoriteks $\{ e_1, e_2, f_1 \}$. Siis
\begin{align*}
    \begin{matrix*}[l]
        & [e_1, e_1, e_1] = 0, \quad
        & [e_1, f_1, f_1] = \l_1 e_1 + \l_2 e_2, \\
        %
        & [e_1, e_1, e_2] = 0, \quad
        & [e_2, e_2, e_2] = 0, \\
        %
        & [e_1, e_1, f_1] = 0, \quad
        & [e_2, e_2, f_1] = 0, \\
        %
        & [e_1, e_2, e_2] = 0, \quad
        & [e_2, f_1, f_1] = \l_3 e_1 +\l_4 e_2, \\
        %
        & [e_1, e_2, f_1] = \m_1 f_1, \quad
        & [f_1, f_1, f_1] = \m_2 f_1,
    \end{matrix*}
\end{align*}
kus $\l_1, \l_2, \l_3, \l_4, \m_1, \m_2 \in \C$ on suvalised skalaarid.

Algoritmi tulemusel saame $24$ mittetriviaalset
kitsendust\footnote{Need on leitavad veebilehelt
\url{http://priitlatt.github.io/masters-thesis/kitsendused/2_1.txt}},
mida $3$-Lie superalgebra $\g$ struktuurikonstandid rahuldama peavad.
Kitsenduste mittetriviaalseteks lahenditeks sobivad seejuures
struktuurikonstantide komplektid
\renewcommand\arraystretch{1.3}
\begin{align*}
    \left\{ \begin{matrix*}[l]
        \m_1 = 0 \\
        \m_2 = 0 \\
        \l_1 = c_1 \\
        \l_2 = c_2 \\
        \l_3 = -\frac{c_1^2}{c_2} \\
        \l_4 = -c_1 \\
    \end{matrix*} \right.,
    \quad
    \left\{ \begin{matrix*}
        \m_1 = c_3 \\
        \m_2 = 0 \\
        \l_1 = 0 \\
        \l_2 = 0 \\
        \l_3 = 0 \\
        \l_4 = 0 \\
    \end{matrix*} \right.,
    \quad
    \left\{ \begin{matrix*}
        \m_1 = 0 \\
        \m_2 = 0 \\
        \l_1 = 0 \\
        \l_2 = 0 \\
        \l_3 = c_4 \\
        \l_4 = 0 \\
    \end{matrix*} \right.,
\end{align*}
\renewcommand\arraystretch{1}
kus $c_1, c_2, c_3, c_4 \in \C \setminus \{0\}$ on vabad muutujad.

\begin{lau}
    Olgu $3$-Lie superalgebra $\g$ supervektorruumi dimensioon $2|1$.
    Siis $\g$ on kas Abeli Lie superalgebra või tema mittetriviaalsed
    kommutatsiooniseosed avalduvad kujul
    \renewcommand\arraystretch{1.2}
    \begin{align}
        & \left\{
            \begin{matrix*}[l]
                [e_1, f_1, f_1] = c_1 e_1 + c_2 e_2, \\
                [e_2, f_1, f_1] = -\frac{c_1^2}{c_2} e_1 - c_1 e_2, \\
            \end{matrix*}
        \right. \label{samasus:2|1 - 1} \\[0.2cm]
        &\ \ \ [e_1, e_2, f_1] = c f_1, \label{samasus:2|1 - 2} \\[0.2cm]
        &\ \ \ [f_1, f_1, f_1] = c f_1, \label{samasus:2|1 - 3}
    \end{align}
    \renewcommand\arraystretch{1}
    kus $c, c_1, c_2 \in \C \setminus \{ 0 \}$ on vabad muutujad.
\end{lau}

Paneme tähele, et sooritades muutujavahetuse $e_1 = c e_1'$, saame samasuse
\eqref{samasus:2|1 - 2} teisendada kujule
\[ [e_1', e_2, f_1] = f_1. \]
Analoogiliselt saame teisenduse $f_1 = \sqrt{c} f_1'$ abil samasuse
\eqref{samasus:2|1 - 3} asemel
\[ [f_1', f_1', f_1'] = f_1'. \]
Lõpetuseks, kui vaatleme muutujavahetusi $e_1 = c_2 e_1'$, $e_2 = c_1 e_2'$
ja $f_1 = \sqrt{c_1} f_1'$, siis on vahetult kontrollitav, et seosed
\eqref{samasus:2|1 - 1} saavad kuju
\[ \left\{
    \begin{matrix*}[l]
        [e_1', f_1', f_1'] = e_1' + e_2', \\
        [e_2', f_1', f_1'] = -e_1' - e_2', \\
    \end{matrix*}
\right. \]
mis kokkuvõttes tähendab, et me oleme tõestanud järgmise teoreemi.

\begin{thm}
    Olgu $3$-Lie superalgebra $\g$ supervektorruumi dimensioon on $2|1$.
    Siis $\g$ on kas Abeli Lie superalgebra, või ta on isomorfne $3$-Lie
    superalgebraga $\h$, mille mittetriviaalsed kommutatsiooniseosed on kas
    \[ \left\{
        \begin{matrix*}[l]
            [e_1, f_1, f_1] = e_1 + e_2, \\
            [e_2, f_1, f_1] = -e_1 - e_2, \\
        \end{matrix*}
    \right. \]
    \[ [e_1, e_1, f_1] = f_1, \]
    või
    \[ [f_1, f_1, f_1] = f_1, \]
    kus $e_1$ ja $e_2$ on $\h$ paarisgeneraatorid ning $f_1$ on $\h$ paaritu
    generaator.
\end{thm}

